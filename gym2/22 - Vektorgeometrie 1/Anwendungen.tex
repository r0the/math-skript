\newpage
\section{Anwendungen}

Mit Vektoren können geometrische Probleme algebraisch gelöst werden. Hier werden erste solche Probleme betrachtet.

% ------------------------------------------------------------------------------
\subsection{Strecke}
\textbf{Aufgabe:} Teilen Sie die Strecke $\overline{AB}$ in drei gleiche Teile.
\[
  A(2\mid 4) \qquad B(11\mid 1)
\]
Zunächst wird der Vektor $\vv{a} = \vv{AB}$ bestimmt. Der erste Teilpunkt $T_{1}$ erhält man, indem ein Drittel von $\vv{a}$ zum Ortsvektor von $A$ addiert wird. Der zweite Teilpunkt erhält man durch Addieren von zwei Dritteln von $\vv{a}$.
\[
  \vv{a} := \vv{AB} \qquad\qquad \vv{OT_{1}} = \vv{OA} + \frac{1}{3}\vv{a} \qquad\qquad \vv{OT_{2}} = \vv{OA} + \frac{2}{3}\vv{a}
\]

\begin{center}
  \begin{tikzpicture}
    \tkzInit[xmin=0,xmax=12,ymin=0,ymax=5]
    \tkzAxeXY[]
    \tkzGrid[color=lightgray]
    \tkzDefPoint(0,0){O}
    \tkzDefPoint(2,4){A}
    \tkzDefPoint(11,1){B}
    \tkzDefPoint(5,3){T1}
    \tkzDefPoint(8,2){T2}
    \tkzDrawSegment[](A,B)
    \tkzDrawSegment[-LaTeX](O,A)
    \tkzDrawSegment[-LaTeX](O,T1)
    \tkzDrawSegment[-LaTeX](O,T2)
    \tkzDrawSegment[red,thick,-LaTeX](A,T1)
    \tkzDrawSegment[red,thick,-LaTeX](T1,T2)
    \tkzDrawSegment[red,thick,-LaTeX](T2,B)

    \tkzLabelSegment[above left](O,A){$\vv{OA}$}
    \tkzLabelSegment[above left](O,T1){$\vv{OT_{1}}$}
    \tkzLabelSegment[below right](O,T2){$\vv{OT_{2}}$}
    \tkzLabelSegment[red,below left](A,T1){$\frac{1}{3}\vv{a}$}
    \tkzLabelSegment[red,below left](T1,T2){$\frac{1}{3}\vv{a}$}
    \tkzLabelSegment[red,below left](T2,B){$\frac{1}{3}\vv{a}$}

    \tkzDrawPoints(A,B)
    \tkzDrawPoints[red](T1,T2)
    \tkzLabelPoint[above right](A){$A$}
    \tkzLabelPoint[above right](B){$B$}
    \tkzLabelPoint[red,above right](T1){$T_{1}$}
    \tkzLabelPoint[red,above right](T2){$T_{2}$}
  \end{tikzpicture}
\end{center}
Damit ergeben sich folgende Ortsvektoren und Koordinaten für $T_{1}$ und $T_{2}$:
\begin{align*}
       \vv{a} &= \vv{AB} = \vxy{11}{1} - \vxy{2}{4} = \vxy{11-2}{1-4} = \vxy{9}{-3} \\[2mm]
  \vv{OT_{1}} &= \vxy{2}{4}+\frac{1}{3}\vxy{9}{-3} = \vxy{2+\frac{1}{3}9}{4+\frac{1}{3}(-3)} = \vxy{5}{3} \\[2mm]
  \vv{OT_{2}} &= \vxy{2}{4}+\frac{2}{3}\vxy{9}{-3} = \vxy{2+\frac{2}{3}9}{4+\frac{2}{3}(-3)} = \vxy{8}{2}
\end{align*}
Die Strecke wird also durch die Punkte $T_{1}(5\mid 3)$ und $T_{2}(8\mid 2)$ in drei gleiche Teile geteilt.

% ------------------------------------------------------------------------------
\subsection{Dreieck}
\textbf{Aufgabe:} Bestimmen Sie den Umfang des Dreiecks $\triangle ABC$ mit
\[
  A(3\mid 2) \qquad B(8\mid 1) \qquad C(7\mid 4)
\]
Zwischen den Punkten $A$, $B$ und $C$ können die Vektoren $\vv{AB}$, $\vv{BC}$ und $\vv{CA}$ definiert werden.
\begin{center}
  \begin{tikzpicture}
    \tkzInit[xmin=0,xmax=12,ymin=0,ymax=5]
    \tkzAxeXY[]
    \tkzGrid[color=lightgray]
    \tkzDefPoint(3,2){A}
    \tkzDefPoint(8,1){B}
    \tkzDefPoint(7,4){C}
    \tkzDrawSegment[red,thick,-LaTeX](A,B)
    \tkzDrawSegment[red,thick,-LaTeX](B,C)
    \tkzDrawSegment[red,thick,-LaTeX](C,A)
    \tkzLabelSegment[red,below left](A,B){$\vv{a}$}
    \tkzLabelSegment[red,above right](B,C){$\vv{b}$}
    \tkzLabelSegment[red,above left](C,A){$\vv{c}$}

    \tkzDrawPoints(A,B,C)
    \tkzLabelPoint[below left](A){$A$}
    \tkzLabelPoint[below right](B){$B$}
    \tkzLabelPoint[above](C){$C$}
  \end{tikzpicture}
\end{center}
Anschliessend werden die Längen der Vektoren ermittelt und addiert, um den Umfang des Dreiecks zu erhalten.
\begin{align*}
  \vv{a} &= \vv{AB} = \vxy{8-3}{1-2} = \vxy{5}{-1} & |\vv{a}| &= \sqrt{5^{2}+(-1)^{2}} = \sqrt{26} \approx 5.099 \\[2mm]
  \vv{b} &= \vv{BC} = \vxy{7-8}{4-1} = \vxy{-1}{3} & |\vv{b}| &= \sqrt{(-1)^{2}+3^{2}} = \sqrt{10} \approx 3.162 \\[2mm]
  \vv{c} &= \vv{CA} = \vxy{3-7}{2-4} = \vxy{-4}{-2} & |\vv{c}| &= \sqrt{(-4)^{2}+(-2)^{2}} = \sqrt{20} \approx 4.472
\end{align*}
Damit ergibt sich für den Umfang des Dreiecks:
\[
  U = |\vv{a}|+|\vv{b}|+|\vv{c}| \approx 5.099 + 3.162 + 4.472 \approx 12.733
\]

% ------------------------------------------------------------------------------
\subsection{Parallelogramm}

\textbf{Aufgabe:} Ermitteln Sie den fehlenden Eckpunkt $C$ und den Umfang des Parallelogramms $ABCD$ mit
\[
  A(3\mid 2) \qquad B(8\mid 1) \qquad D(6\mid 4)
\]
Zuerst werden die Seitenvektoren des Parallelogramms bestimmt:
\begin{align*}
  \vv{a} &= \vv{OB}-\vv{OA} = \vxy{8}{1}-\vxy{3}{2} = \vxy{8-3}{1-2} = \vxy{5}{-1} \\[2mm]
  \vv{b} &= \vv{OD}-\vv{OA} = \vxy{6}{4}-\vxy{3}{2} = \vxy{6-3}{4-2} = \vxy{3}{2}
\end{align*}
Der Punkt $C$ erhält man, indem der Punkt $B$ um den Vektor $\vv{b}$ verschiebt. Dazu wird $\vv{b}$ zum Ortsvektor $\vv{OB}$ addiert:
\[
  \vv{OC} = \vv{OB} + \vv{b} = \vxy{8}{1} + \vxy{3}{2} = \vxy{8+3}{1+2} = \vxy{11}{3}
\]
Also ist der vierte Punkt des Parallelogramms $C(11\mid 3)$.

\begin{center}
  \begin{tikzpicture}
    \tkzInit[xmin=0,xmax=12,ymin=0,ymax=5]
    \tkzAxeXY[]
    \tkzGrid[color=lightgray]
    \tkzDefPoint(0,0){O}
    \tkzDefPoint(3,2){A}
    \tkzDefPoint(8,1){B}
    \tkzDefPoint(11,3){C}
    \tkzDefPoint(6,4){D}

    \tkzDrawSegment[-LaTeX](O,A)
    \tkzLabelSegment[above left](O,A){$\vv{OA}$}

    \tkzDrawSegment[-LaTeX](O,B)
    \tkzLabelSegment[below right](O,B){$\vv{OB}$}

    \tkzDrawSegment[thick,-LaTeX](A,B)
    \tkzLabelSegment[below left](A,B){$\vv{a}$}

    \tkzDrawSegment[thick,-LaTeX](A,D)
    \tkzLabelSegment[above left](A,D){$\vv{b}$}

    \tkzDrawSegment[red,thick,-LaTeX](B,C)
    \tkzLabelSegment[red,below right](B,C){$\vv{b}$}

    \tkzDrawSegment[red,thick,dotted](C,D)

    \tkzDrawPoints(A,B,D)
    \tkzDrawPoints[red](C)
    \tkzLabelPoint[above left](A){$A$}
    \tkzLabelPoint[below right](B){$B$}
    \tkzLabelPoint[red,right](C){$C$}
    \tkzLabelPoint[above](D){$D$}
  \end{tikzpicture}
\end{center}

Der Umfang des Parallelogramms ergibt sich, indem die Längen von $\vv{a}$ und $\vv{b}$ addiert und verdoppelt werden:
\[
  U = 2\left(|\vv{a}| + |\vv{b}|\right) = 2\left(\sqrt{5^{2}+(-1)^2}+\sqrt{3^{2}+2^{2}}\right) \approx 2\cdot 8.704 \approx 12.310
\]

% ------------------------------------------------------------------------------
\subsection{Quadrat}

\textbf{Aufgabe:} Ermitteln Sie die fehlenden Eckpunkte $C$ und $D$ sowie die Fläche des Quadrats $ABCD$ mit
\[
  A(3\mid 2) \qquad B(6\mid 1)
\]
\begin{center}
  \begin{tikzpicture}
    \tkzInit[xmin=0,xmax=10,ymin=0,ymax=5]
    \tkzAxeXY[]
    \tkzGrid[color=lightgray]
    \tkzDefPoint(3,2){A}
    \tkzDefPoint(6,1){B}
    \tkzDefPoint(7,4){C}
    \tkzDefPoint(4,5){D}
    \tkzDrawSegment[thick,-LaTeX](A,B)
    \tkzLabelSegment[below left](A,B){$\vv{a}$}

    \tkzDrawSegment[red,thick,-LaTeX](A,D)
    \tkzLabelSegment[red,above left](A,D){$\vv{b}$}
    \tkzDrawSegment[red,thick,-LaTeX](B,C)
    \tkzLabelSegment[red,below right](B,C){$\vv{b}$}
    \tkzDrawSegment[red,thick,dotted](C,D)

    \tkzDrawPoints(A,B,D)
    \tkzDrawPoints[red](C)
    \tkzLabelPoint[left](A){$A$}
    \tkzLabelPoint[below](B){$B$}
    \tkzLabelPoint[red,right](C){$C$}
    \tkzLabelPoint[red,above](D){$D$}
    \tkzMarkRightAngle[red,german,size=0.5](B,A,D)
  \end{tikzpicture}
\end{center}
Zunächst wird der Vektor $\vv{a} = \vv{AB}$ ermittelt:
\[
  \vv{a} = \vxy{6}{1}-\vxy{3}{2} = \vxy{6-3}{1-2} = \vxy{3}{-1}
\]
Nun wird der zu $\vv{a}$ senkrechte Vektor $\vv{b}$ ermittelt:
\[
  \vv{b} = \vxy{1}{3}
\]
Nun werden $C$ und $D$ ermittelt:
\begin{align*}
  \vv{OC} &= \vv{OB}+\vv{b} = \vxy{6}{1} + \vxy{1}{3} = \vxy{6+1}{1+3} = \vxy{7}{4} \\[2mm]
  \vv{OD} &= \vv{OA}+\vv{b} = \vxy{3}{2} + \vxy{1}{3} = \vxy{3+1}{2+3} = \vxy{4}{5}
\end{align*}
Die gesuchten Punkte sind also $C(7\mid 4)$ und $D(4\mid 5)$. Die Fläche des Quadrats beträgt
\[
  A = |\vv{a}|^{2} = \left(\sqrt{3^{2}+1^{2}}\right)^{2} = 9+1 = 10
\]


\newpage
% ------------------------------------------------------------------------------
\subsection{Kreis}

\textbf{Aufgabe:} Berechnen Sie den Mittelpunkt und den Radius des Kreises, auf welchem die folgenden drei Punkte liegen:
\[
  A(3\mid 2) \qquad B(6\mid 5) \qquad C(7\mid 2)
\]
\begin{center}
  \begin{tikzpicture}
    \tkzInit[xmin=0,xmax=10,ymin=0,ymax=6]
    \tkzAxeXY[]
    \tkzGrid[color=lightgray]
    \tkzDefPoint(3,2){A}
    \tkzDefPoint(6,5){B}
    \tkzDefPoint(7,2){C}
    \tkzDefPoint(5,3){M}

    \tkzDrawCircle[thick,red](M,C)

    \tkzDrawSegment[red,thick,-LaTeX](A,M)
    \tkzLabelSegment[red,below right](A,M){$\vv{AM}$}

    \tkzDrawPoints(A,B,C)
    \tkzDrawPoints[red](M)
    \tkzLabelPoints[below left](A)
    \tkzLabelPoints[above right](B)
    \tkzLabelPoints[below right](C)
    \tkzLabelPoints[red,below right](M)
  \end{tikzpicture}
\end{center}
Sei $\vv{OA}$ der Ortsvektor des Punkts $A$ auf dem Kreis und $\vv{OM}$ der Ortsvektor des Mittelpunkts $M$ mit den folgenden Komponenten:
\[
  \vv{OA} = \vxy{3}{2} \qquad \vv{OM} = \vxy{x}{y}
\]
Der Vektor $\vv{AM} = \vv{OM}-\vv{OA}$ zeigt vom Punkt $P$ zum Kreismittelpunkt $M$. Seine Länge ist der Radius des Kreises $r$:
\[
  \left|\vv{OM}-\vv{OA}\right| = r
\]
In Komponentenschreibweise ergibt sich die folgende Gleichung:
\[
  \sqrt{\left(x-3\right)^{2}+\left(y-2\right)^{2}} = r
\]
Um die Wurzel zu eliminieren wird die Gleichung quadriert.
\[
  \left(x-3\right)^{2}+\left(y-2\right)^{2} = r^{2}
\]
Für die Punkte $B$ und $C$ ergeben sich zwei analoge Gleichungen. Diese ergeben zusammen ein Gleichungssystem mit drei Gleichungen und drei Unbekannten:
\begin{align*}
  (x-3)^{2}+(y-2)^{2} &= r^{2} && \circled{1} \\
  (x-6)^{2}+(y-5)^{2} &= r^{2} && \circled{2} \\
  (x-7)^{2}+(y-2)^{2} &= r^{2} && \circled{3}
  \intertext{Die Gleichungen werden ausmultipliziert:}
  x^{2}-6x+y^{2}-4y+13 &= r^{2} && \circled{1} \\
  x^{2}-12x+y^{2}-10y+61 &= r^{2} && \circled{2} \\
  x^{2}-14x+y^{2}-4y+53 &= r^{2} && \circled{3} \\
\end{align*}
Durch Gleichsetzen der Gleichungen werden $r$ und die Quadrate eliminiert und es entsteht ein lineares Gleichungssystem:
\begin{align*}
  -6x-4y+13 &= -12x-10y+61 && \circled{1}=\circled{2} \\
  -6x-4y+13 &= -14x-4y+53 && \circled{1}=\circled{3}
  \intertext{Nach Zusammenfassen ergibt sich:}
  6x+6y &= 48 \\
  8x &= 40
\end{align*}
Die Lösung des Gleichungssystems ist $x = 5$ und $y = 3$. Also hat der Mittelpunkt des Kreises die Koordinaten $M(5\mid 3)$.

Den Radius des Kreises ist die Länge des Vektors $\vv{AM}$:
\[
  r = \left|\vv{AM}\right| = \left|\vxy{5-2}{3-2}\right| = \left|\vxy{3}{1}\right| = \sqrt{3^{2}+1^{2}} = \sqrt{10} \approx 3.162
\]
