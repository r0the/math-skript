\documentclass[parskip=half]{scrartcl}
% ------------------------------------------------------------------------------
% LaTeX-Grundkonfiguration von Stefan Rothe
% ------------------------------------------------------------------------------

% 1 Konfiguration der Schriftarten
% --------------------------------
% um \ifXeTeX verwenden zu können
\usepackage{iftex}

\ifXeTeX
  % Setze PDF-Version auf 1.7
  \special{pdf:minorversion 7}

  % 1.1 Konfiguration der Schriftarten für XeTeX (empfohlen)
  % ----------------------------------------------------------
  \usepackage{fontspec}

  \IfFontExistsTF{Helvetica}{\setmainfont{Helvetica}}{%
    \IfFontExistsTF{Arial}{\setmainfont{Arial}}{}%
  }

  \IfFontExistsTF{Helvetica}{\setsansfont{Helvetica}}{%
    \IfFontExistsTF{Arial}{\setsansfont{Arial}}{}%
  }

  \IfFontExistsTF{Menlo}{\setmonofont[SizeFeatures={Size=10}]{Menlo}}{%
    \IfFontExistsTF{Courier New}{\setmonofont[SizeFeatures={Size=10}]{Courier New}}{}%
  }
\else
  % Setze PDF-Version auf 1.7
  \pdfminorversion=7

  % 1.2 Konfiguration der Schriftarten für PdfTeX
  % -----------------------------------------------

  % Unterstützung von UTF-8 (Unicode)
  \usepackage[utf8]{inputenc}

  % Unterstützung der modernen Zeichencodierung
  \usepackage[T1]{fontenc}

  % Moderne Schriftart verwenden
  \usepackage{lmodern}

  % Schriftart Helvetica verwenden
  \usepackage{helvet}

  % serifenfreie Schriftvariante verwenden
  \renewcommand{\familydefault}{\sfdefault}
\fi

% 2 wichtige Pakete laden und konfigurieren
% -----------------------------------------

% sprachspezifische Anpassungen
\usepackage[ngerman]{babel}

% Absatzabstände kontrollieren für nicht-KOMA-Klassen
\makeatletter
\@ifclassloaded{scrartcl}{}{\usepackage{parskip}}
\makeatother

% Aufzählungen
\usepackage{enumitem}

% mehrere Spalten
\usepackage{multicol}

% Rahmen
\usepackage{tcolorbox}

% Tabellen
\usepackage{booktabs}
\usepackage{tabularx}

% Grafiken (JPG, PNG, PDF)
\usepackage{graphicx}

% Links
\usepackage{hyperref}
\hypersetup{colorlinks=true,urlcolor=blue,linkcolor=black}

% 3 Mathematik
% ------------

% 3.1 Zahlen und Einheiten schön darstellen
% -----------------------------------------
\usepackage{siunitx}
\sisetup{%
  mode=match,
  exponent-product=\cdot,
  group-digits=integer,
  group-separator={\text{\textquotesingle}}
}

% 3.2 AMS-Mathematik
% ------------------

\usepackage[fleqn]{amsmath}
\usepackage{amssymb}
\usepackage{amsthm}

% eigene Operatoren
\DeclareMathOperator{\ggT}{ggT}
\DeclareMathOperator{\kgV}{kgV}
\DeclareMathOperator{\lb}{lb}

% 3.3 Punkte und Vektoren
% -----------------------
\usepackage[b]{esvect}

\makeatletter
% Komponentenschreibweise für Punkte
\NewDocumentCommand\coord@internal{mmmm}{%
\IfNoValueTF{#3}{\mathopen{}\left(#1/\mathopen{}#2\mathclose{}\right)\mathclose{}}
{\mathopen{}\left(#1/\mathopen{}#2\mathclose{}/\mathopen{}#3\mathclose{}\right)\mathclose{}}}
\NewDocumentCommand\coord{o>{\SplitArgument{3}{,}}m}{\IfNoValueF{#1}{#1}\coord@internal #2}

% Komponentenschreibweise für Vektoren für inline-Math
\NewDocumentCommand\tvec@internal{mmmm}{%
\IfNoValueTF{#3}{\left(\begin{smallmatrix}\scriptstyle #1\\\scriptstyle #2\end{smallmatrix}\right)}
{\left(\begin{smallmatrix}\scriptstyle #1\\\scriptstyle #2\\\scriptstyle #3\end{smallmatrix}\right)}}
\NewDocumentCommand\tvec{>{\SplitArgument{3}{,}}m}{\tvec@internal #1}

% Komponentenschreibweise für Vektoren für display-Math
\NewDocumentCommand\dvec@internal{mmmm}{%
\IfNoValueTF{#3}{\left(\begin{matrix}#1\\ #2\end{matrix}\right)}
{\left(\begin{matrix} #1\\ #2\\ #3\end{matrix}\right)}}
\NewDocumentCommand\dvec{>{\SplitArgument{3}{,}}m}{\dvec@internal #1}
\makeatother

% 3.4 Umgebung eqt für Äquivalenzumformungen von Gleichungen
% ----------------------------------------------------------
\makeatletter
\newcounter{@eqtrownumber}
\def\eqtequiv{\stepcounter{@eqtrownumber}\ifnum\value{@eqtrownumber}=1\else\Leftrightarrow\qquad\qquad\fi}
\makeatother
\NewDocumentEnvironment{eqt}{}{\begin{equation*}\begin{array}{@{\eqtequiv}>{\displaystyle}r@{\hspace{2mm}}>{\displaystyle}l@{\hspace{1cm}}|l}}{\end{array}\end{equation*}}
\preto\eqt{\setcounter{@eqtrownumber}{0}}

% 3.5 Lineare Gleichungssysteme
% -----------------------------
\usepackage{systeme}
\sysdelim||

% 3.6 Differentiale
% -----------------
\usepackage{derivative}

% 3.7 Umgebung für Beispiele
% --------------------------
% Definition eines neuen Theorem-Stils für Beispiele
\newtheoremstyle{example}
  {3pt}% Platz oberhalb des Beispiels
  {3pt}% Platz unterhalb des Beispiels
  {\itshape}% Schriftart des Beispiel-Textes
  {}% Abstand nach dem Beispiel-Kopf
  {\bfseries\itshape}% Schriftart des Beispiel-Kopfes
  {.}% Punktierung nach dem Beispiel-Kopf
  {.5em}% Abstand nach der Punktierung
  {\thmname{#1}\thmnumber{ #2}\thmnote{ (#3)}}% Kopfname spezifizieren

\theoremstyle{example}
\newtheorem*{example}{Beispiel}

% 4 Diagramme
% -----------

% 4.1 TikZ
% --------
% muss wegen der Konfiguration vor tikz eingebunden werden
% Mit table können Tabellenzellen eingefärbt werden
\usepackage[table]{xcolor}

% für Geometrie (TikZ-Erweiterung)
% damit wird auch tikz eingebunden
\usepackage{tkz-base}

% für Funktionsplots
\usepackage{tkz-fct}

% Konfiguration Darstellung von Tangenten in tkz-fct
\tkzfctset{tan style/.style={red,thick,>=}}

% für Geometrie
\usepackage{tkz-euclide}

% Kürzen bei Brüchen zeigen
\usepackage{cancel}

% Schriftliche Division
\usepackage{longdivision}
\longdivisionkeys{style=german}

% Bäume mit tikz
\usepackage{forest}

% 5 Informatik
% ------------

% 5.1 Quellcode
% -------------
\usepackage{listings}

\definecolor{pythonCommentColor}{HTML}{9F9F9F}
\definecolor{pythonIdentifierColor}{HTML}{02007C}
\definecolor{pythonKeywordColor}{HTML}{6B134A}
\definecolor{pythonNumberColor}{HTML}{9E4A1A}
\definecolor{pythonStringColor}{HTML}{215912}
\lstset{
  basicstyle=\ttfamily\small,
  breaklines,
  commentstyle={\color{pythonCommentColor}\itshape},
  frame=single,
  keywordstyle={\color{pythonKeywordColor}\bfseries},
  language=Python,
  numbers=left,
  numbersep=5pt,
  numberstyle=\scriptsize\color{black!70},
  showstringspaces=false,
  stringstyle={\color{pythonStringColor}},
}

% Material Design-Farben
% ----------------------
\definecolor{lightblue}{HTML}{BBDEFB} % MD Blue 100
\definecolor{lightred}{HTML}{FFCDD2} % MD Red 100
\definecolor{lightgrey}{HTML}{F5F5F5} % MD Gray 100
\definecolor{lightgreen}{HTML}{C8E6C9} % MD Green 100
\definecolor{lightorange}{HTML}{FFE0B2} % MD Orange 100

\definecolor{theoremcolor}{HTML}{FFEBEE} % MD Red 50
\definecolor{notecolor}{HTML}{FFECB3} % MD Amber 100

\definecolor{red}{HTML}{D50000} % MD Red A700
\definecolor{green}{HTML}{31843F} % MD Green A700
\definecolor{blue}{HTML}{2962FF} % MD Blue A700
\definecolor{teal}{HTML}{00BFA5} % MD Teal A700
\definecolor{cyan}{HTML}{00B8D4} % MD Cyan A700
\definecolor{yellow}{HTML}{EE8E0D} % WordPress Colors Yellow Fire 40%

\tikzset{dim style/.append style={purple,dashed}}
\tikzset{dim fence style/.append style={purple}}
\tikzset{mark angle style/.append style={german}}

% eigene Befehle
% --------------

\tikzset{circled/.style={shape=circle,draw,inner sep=2pt}}
\NewDocumentCommand\circled{m}{\tikz[baseline=(char.base)]{\node[circled] (char) {#1};}}

% Makro \result, um das Resultat von Berechnungen hervorzuheben.
\NewDocumentCommand\result{m}{\textcolor{red}{#1}}

% Makro \extra, um schwierige Zusatzaufgaben zu markieren
\NewDocumentCommand\extra{}{$\bigstar\quad$}


\NewTColorBox{note}{}{
  parbox=false,
  colback=notecolor,
  colframe=black,
  arc=0mm,
  before skip=2mm,
  left=1mm,
  right=1mm,
  boxrule=1pt,
}
\NewTColorBox{theorem}{}{
  parbox=false,
  colback=theoremcolor,
  colframe=black,
  arc=0mm,
  before skip=2mm,
  left=1mm,
  right=1mm,
  boxrule=1pt,
}
\NewTColorBox{instructions}{}{
  parbox=false,
  colback=notecolor,
  colframe=black,
  arc=0mm,
  before skip=2mm,
  left=1mm,
  right=1mm,
  boxrule=1pt,
}

\usepackage{fontawesome5}
\def\digital{\faLaptop{} }
\def\present{\faTrophy{} }

\def\rosconfig{1}


\usepackage{scrlayer-scrpage}
\pagestyle{scrheadings}




\KOMAoption{DIV}{12}
\KOMAoption{toc}{listof}
\DeclareTOCStyleEntry[entryformat=\bfseries,beforeskip=2pt,linefill=\TOCLineLeaderFill]{tocline}{section}
\setcounter{tocdepth}{1}

\title{Vektorgeometrie 2D}
\author{Stefan Rothe}
\date{26.12.2024}

\newpairofpagestyles{firstpage}{%
  \cofoot{\textcopyright{} Gymnasium Kirchenfeld\\Dieses Skript steht unter einer Creative Commons Attribution 4.0 International-Lizenz.\\(CC BY 4.0)}
}

\makeatletter
\lohead{\@title}
\rohead{\@date}
\cofoot{\thepage}
\rofoot{}
\makeatother

\begin{document}
  \maketitle
  \thispagestyle{firstpage}
  \begin{center}
    \begin{tikzpicture}
      \tkzInit[xmin=0,xmax=12,ymin=0,ymax=5]
      \tkzDefPoint(0,0){O}
      \tkzDefPoint(2,2){A}
      \tkzDefPoint(9,1){B}
      \tkzDefPoint(7,4){C}
      \tkzDefMidPoint(B,C) \tkzGetPoint{M}
      \tkzDefTriangleCenter[centroid](A,B,C)
      \tkzGetPoint{S}

      \tkzDrawSegment[thick](A,B)
      \tkzDrawSegment[thick](C,A)
      \tkzDrawSegment[vector style](O,A)
      \tkzDrawSegment[vector style](O,B)
      \tkzDrawSegment[vector style](O,C)
      \tkzDrawSegment[red,thick,vector style](B,C)
      \tkzDrawSegment[red,thick,vector style](A,M)

      \tkzLabelSegment[above left](O,A){$\vv{OA}$}
      \tkzLabelSegment[below right,pos=0.6](O,B){$\vv{OB}$}
      \tkzLabelSegment[below right,pos=0.3](O,C){$\vv{OC}$}
      \tkzLabelSegment[red,below right](A,M){$\vv{AM}$}
      \tkzLabelSegment[red,above right,pos=0.3](B,C){$\vv{a}$}

      \tkzDrawPoints(A,B,C,M,S)
      \tkzLabelPoint[above left](A){$A$}
      \tkzLabelPoint[below right](B){$B$}
      \tkzLabelPoint[above](C){$C$}
      \tkzLabelPoint[above right](M){$M$}
      \tkzLabelPoint[above right](S){$S$}
    \end{tikzpicture}

    \vspace{5mm}
    $\displaystyle\vv{OS} = \tfrac{2}{3}\left(\tfrac{1}{2}\left(\vv{OC}-\vv{OB}\right) -\vv{OA}\right)$
  \end{center}
  \tableofcontents
  \newpage
  \newpage
\section{Einführung}

% ------------------------------------------------------------------------------
\subsection{Vektorbegriff}

Vektoren sind eine neues Grundelement der Mathematik, welche die Zahlen ergänzen. Im Gegensatz zu Zahlen besitzen Vektoren auch eine \textbf{Richtung}.

Ein Vektor kann als Verschiebung interpretiert werden.

% ------------------------------------------------------------------------------
\subsection{Darstellung}

Ein Vektor $\vv{a}$ wird als Pfeil dargestellt und mit einem Kleinbuchstaben mit Pfeil beschriftet.
\begin{center}
  \begin{tikzpicture}
    \tkzDefPoint(0,0){A1}
    \tkzDefShiftPoint[A1](3,0.8){A2}
    \tkzDrawSegment[thick,-LaTeX](A1,A2)
    \tkzLabelSegment[above left](A1,A2){$\vv{a}$}
  \end{tikzpicture}
\end{center}

% ------------------------------------------------------------------------------
\subsection{Definition durch Endpunkte}

Ein Vektor $\vv{a}$ kann durch einen Anfangspunkt $A$ und einen Endpunkt $B$ definiert werden. Dazu wird die folgende Schreibweise verwendet:
\[
  \vv{a} = \vv{AB}
\]
\begin{center}
  \begin{tikzpicture}
    \tkzDefPoint(0,0){A}
    \tkzDefShiftPoint[A](3,0.8){B}
    \tkzDrawSegment[thick,-LaTeX](A,B)
    \tkzLabelSegment[above left](A,B){$\vv{a}$}
    \tkzDrawPoints(A,B)
    \tkzLabelPoints[left](A)
    \tkzLabelPoints[right](B)
  \end{tikzpicture}
\end{center}
Da der Vektor gerichtet ist, ist die Reihenfolge der Punkte wichtig. Der Vektor $\vv{BA}$ ist ein anderer Vektor als $\vv{AB}$.

\newpage
% ------------------------------------------------------------------------------
\subsection{Addition}

Die Addition von Vektoren ist die Aneinanderreihung von Verschiebungen. Wird ein Punkt $A$ zunächst um den Vektor $\vv{a}$ verschoben und anschliessend um den Vektor $\vv{b}$, so kommt der Punkt bei $C$ zu liegen. Die gleiche Verschiebung entsteht, wenn der Punkt $A$ um den Vektor
\[
  \vv{c} = \vv{a} + \vv{b}
\]
verschoben wird:
\begin{center}
  \begin{tikzpicture}
    \tkzDefPoint(0,0){A}
    \tkzDefShiftPoint[A](2,2){D}
    \tkzDefShiftPoint[D](4,-1){C}
    \tkzDrawPoints(A,C)
    \tkzLabelPoints[left](A)
    \tkzLabelPoints[right](C)

    \tkzDrawSegment[thick,-LaTeX](A,D)
    \tkzLabelSegment[above left](A,D){$\vv{a}$}

    \tkzDrawSegment[thick,-LaTeX](D,C)
    \tkzLabelSegment[above right](D,C){$\vv{b}$}

    \tkzDrawSegment[red,thick,-LaTeX](A,C)
    \tkzLabelSegment[red,above left](A,C){$\vv{c}$}
  \end{tikzpicture}
\end{center}

Die Addition von Vektoren ist \textbf{kommutativ}. Es spielt keine Rolle, ob der Vektor $\vv{b}$ an den Vektor $\vv{a}$ angehängt wird oder umgekehrt. Als Summe entsteht in beiden Fällen der gleiche Vektor $\vv{c}$. Die beiden Varianten der Addition bilden ein Parallelogramm.
\[
  \vv{c} = \vv{a} + \vv{b} = \vv{b} + \vv{a}
\]
\begin{center}
  \begin{tikzpicture}
    \tkzDefPoint(0,0){A}
    \tkzDefShiftPoint[A](2,2){D}
    \tkzDefShiftPoint[D](4,-1){C}
    \tkzDefShiftPoint[A](4,-1){B}
    \tkzDrawPoints(A,C)
    \tkzLabelPoints[left](A)
    \tkzLabelPoints[right](C)

    \tkzDrawSegment[thick,-LaTeX](A,B)
    \tkzLabelSegment[below left](A,B){$\vv{a}$}

    \tkzDrawSegment[thick,-LaTeX](B,C)
    \tkzLabelSegment[below right](B,C){$\vv{a}$}

    \tkzDrawSegment[thick,-LaTeX](A,D)
    \tkzLabelSegment[above left](A,D){$\vv{b}$}

    \tkzDrawSegment[thick,-LaTeX](D,C)
    \tkzLabelSegment[above right](D,C){$\vv{b}$}

    \tkzDrawSegment[red,thick,-LaTeX](A,C)
    \tkzLabelSegment[red,above left](A,C){$\vv{c}$}
  \end{tikzpicture}
\end{center}

% ------------------------------------------------------------------------------
\subsection{Skalierung}
Wird ein Vektor $\vv{a}$ zu sich selbst addiert, so entsteht ein paralleler Vektor zu $\vv{a}$ mit doppelter Länge. Der Vektor wird also zwei Mal genommen, geschrieben:
\[
  \vv{a} + \vv{a} = 2\cdot\vv{a} = 2\vv{a}
\]
\begin{center}
  \begin{tikzpicture}
    \tkzDefPoint(0,0){A1}
    \tkzDefShiftPoint[A1](3,0.8){A2}
    \tkzDefShiftPoint[A2](3,0.8){A3}
    \tkzDefShiftPoint[A1](0.1,-0.3){B1}
    \tkzDefShiftPoint[A3](0.1,-0.3){B2}
    \tkzDrawSegment[thick,-LaTeX](A1,A2)
    \tkzLabelSegment[above left](A1,A2){$\vv{a}$}

    \tkzDrawSegment[thick,-LaTeX](A2,A3)
    \tkzLabelSegment[above left](A2,A3){$\vv{a}$}

    \tkzDrawSegment[thick,red,-LaTeX](B1,B2)
    \tkzLabelSegment[red,below right](B1,B2){$2\vv{a}$}
  \end{tikzpicture}
\end{center}
Allgemein kann das mehrfache Addieren des gleichen Vektors als Multiplikation mit einer natürlichen Zahl geschrieben werden:
\[
  n\cdot\vv{a} = n\vv{a} := \underbrace{\vv{a} + \vv{a} + \vv{a} + \cdots + \vv{a}}_{n-\text{mal}}
\]


% ------------------------------------------------------------------------------
\subsection{Gegenvektor}

Der Gegenvektor eines Vektors $\vv{a}$ ist der Vektor, der gleich lang ist wie $\vv{a}$ und in die umgekehrte Richtung zeigt. Der Gegenvektor wird mit $-\vv{a}$ bezeichnet.

\begin{center}
  \begin{tikzpicture}
    \tkzDefPoint(0,0){A1}
    \tkzDefShiftPoint[A1](3,0.8){A2}
    \tkzDefShiftPoint[A1](0.1,-0.3){B2}
    \tkzDefShiftPoint[A2](0.1,-0.3){B1}
    \tkzDrawSegment[thick,-LaTeX](A1,A2)
    \tkzLabelSegment[above left](A1,A2){$\vv{a}$}

    \tkzDrawSegment[thick,red,-LaTeX](B1,B2)
    \tkzLabelSegment[red,below right](B1,B2){$-\vv{a}$}
  \end{tikzpicture}
\end{center}



% ------------------------------------------------------------------------------
\subsection{Subtraktion}

Die Subtraktion von Vektoren kann auf zwei Arten definiert werden. Erstens kann die Subtraktion eines Vektors $\vv{b}$ vom Vektor $\vv{a}$ als Addition des Gegenvektors $-\vv{b}$ betrachtet werden:
\[
  \vv{c} = \vv{a} - \vv{b} = \vv{a} + \left(-\vv{b}\right)
\]
\begin{center}
  \begin{tikzpicture}
    \tkzDefPoint(0,0){A}
    \tkzDefPoint(2,2){B}
    \tkzDefPoint(-2,3){Z1}
    \tkzDefPoint(6,1){Z}
    \tkzDrawPoints(A,Z)

    \tkzDrawPoint(B)

    \tkzDrawSegment[thick,-LaTeX](A,B)
    \tkzLabelSegment[above left](A,B){$\vv{a}$}

    \tkzDrawSegment[thick,-LaTeX](B,Z1)
    \tkzLabelSegment[above right](B,Z1){$-\vv{b}$}

    \tkzDrawSegment[thick,-LaTeX](B,Z)
    \tkzLabelSegment[above right](B,Z){$\vv{b}$}

    \tkzDrawSegment[red,thick,-LaTeX](A,Z1)
    \tkzLabelSegment[red,below left](A,Z1){$\vv{c}$}
  \end{tikzpicture}
\end{center}

Zweitens kann die Subtraktionsgleichung durch addieren von $\vv{b}$ zu einer Addition umgeformt werden:
\[
  \vv{c} = \vv{a} - \vv{b} \qquad\Rightarrow\qquad \vv{c} + \vv{b} = \vv{a}
\]
\begin{center}
  \begin{tikzpicture}
    \tkzDefPoint(0,0){A}
    \tkzDefPoint(2,2){B}
    \tkzDefPoint(-2,3){Z1}

    \tkzDrawSegment[thick,-LaTeX](A,B)
    \tkzLabelSegment[above left](A,B){$\vv{a}$}

    \tkzDrawSegment[thick,-LaTeX](Z1,B)
    \tkzLabelSegment[above right](Z1,B){$\vv{b}$}

    \tkzDrawSegment[red,thick,-LaTeX](A,Z1)
    \tkzLabelSegment[red,below left](A,Z1){$\vv{c}$}
  \end{tikzpicture}
\end{center}

  \newpage
\section{Komponentendarstellung}

% ------------------------------------------------------------------------------
\subsection{Komponenten}

Wird ein Vektor $\vv{v}$ in das kartesische Koordinatensystem eingezeichnet, so kann der Vektor beschrieben werden durch die
\[
  \vv{v} = \vxy{v_{x},v_{y}}
\]
Dabei werden $v_{x}$ und $v_{y}$ die Komponenten des Vektors $\vv{v}$ genannt. $v_{x}$ ist die horizontale Komponente, $v_{y}$ die vertikale Komponente. Die horizontale Komponente steht oben, die vertikale Komponente unten.

% ------------------------------------------------------------------------------
\subsection{Addition und Subtraktion}
Die Addition und Subtraktion von Vektoren erfolgt komponentenweise.

\begin{theorem}
  \textbf{Addition.} Zwei Vektoren $\vv{a} = \vxy{a_{x},a_{y}}$ und $\vv{b} = \vxy{b_{x},b_{y}}$ werden addiert, indem ihre Komponenten addiert werden:
  \[
    \vv{a}+\vv{b} = \vxy{a_{x},a_{y}}+\vxy{b_{x},b_{y}} = \vxy{a_{a}+b_{x},a_{y}+b_{y}}
  \]
  \textbf{Subtraktion.} $\vv{a}$ und $\vv{b}$ werden subtrahiert, indem ihre Komponenten subtrahiert werden:
  \[
    \vv{a}-\vv{b} = \vxy{a_{x},a_{y}}-\vxy{b_{x},b_{y}} = \vxy{a_{a}-b_{x},a_{y}-b_{y}}
  \]

\end{theorem}

% ------------------------------------------------------------------------------
\subsection{Skalierung}

Die Skalierung eines Vektors erfolgt komponentenweise.

\begin{theorem}
  \textbf{Skalierung.} Ein Vektor  $\vv{a} = \vxy{a_{x},a_{y}}$ wird skaliert, indem seine Komponenten mit dem Faktor multipliziert werden:
  \[
    k\cdot\vv{a} = k\cdot\vxy{a_{x},a_{y}} = \vxy{k\cdot a_{x},k\cdot a_{y}}
  \]
\end{theorem}

% ------------------------------------------------------------------------------
\subsection{Länge}

Die Länge eines Vektors $\vv{a}$ wird mit $|\vv{a}|$ bezeichnet. Die Komponenten bilden zusammen mit dem Vektor ein rechtwinkliges Dreieck. Deshalb kann die Länge mit Hilfe des Satzes von Pythagoras aus dem Komponenten berechnet werden.

\begin{center}
  \begin{tikzpicture}
    \tkzInit[xmin=-4,xmax=8,ymin=-1,ymax=4]
    \tkzGrid[color=lightgray]
    \tkzDefPoint(0,0){S}
    \tkzDefPoint(5,3){Z}
    \tkzDefPoint(5,0){H}
    \tkzDrawSegment[thick,vector style](S,Z)
    \tkzDrawPolySeg[thick,red](S,H,Z)
    \tkzLabelSegment[above left](S,Z){$\vv{a}$}
    \tkzLabelSegment[red,below](S,H){$a_{x}$}
    \tkzLabelSegment[red,right](H,Z){$a_{y}$}
    \tkzMarkRightAngle[red,german,size=0.5](Z,H,S)
  \end{tikzpicture}
\end{center}

\begin{theorem}
  \textbf{Länge eines Vektors.} Die Länge $|\vv{a}|$ eines Vektors $\vv{a} = \vxy{a_{x},a_{y}}$ wird wie folgt berechnet:
  \[
    |\vv{a}| = \sqrt{a_{x}^{2}+a_{y}^{2}}
  \]
\end{theorem}

\begin{example}
  \textbf{Beispiel:} In der oben stehenden Abbildung hat der Vektor $\vv{a}$ die Komponenten $\vxy{5,3}$. Also ist seine Länge
  \[
    |\vv{a}| = \sqrt{5^{2}+3^{2}} = \sqrt{34} \approx 5.831
  \]
\end{example}

% ------------------------------------------------------------------------------
\subsection{Nullvektor}

Ein spezieller Vektor ist der Nullvektor. Er stellt einen Pfeil der Länge Null dar. Im Gegensatz zu allen anderen Vektoren hat der Nullvektor keine Richtung. Der Nullvektor wird als Zahl Null mit einem Vektorpfeil geschrieben: $\vv{0}$. Alle seine Komponenten sind Null.
\[
  \vv{0} = \vxy{0,0}
\]

% ------------------------------------------------------------------------------
\subsection{Gleichheit}

Zwei Vektoren sind genau dann gleich, wenn ihre Komponenten übereinstimmen:
\[
  \vxy{a_{x},a_{y}} = \vxy{b_{x},b_{y}} \qquad\Leftrightarrow\qquad a_{x} = b_{x} \quad\text{und}\quad a_{y} = b_{y}
\]
  \newpage
\section{Vektoren und Punkte}

\subsection{Unterschiede}
Vektoren und Punkte sind unterschiedliche Objekte mit unterschiedlichen Eigenschaften.

Ein \textbf{Punkt} befindet sich an einem bestimmten Ort und er hat keine Richtung. Ein Punkt $P$ mit den \textbf{Koordinaten} $x$ und $y$ wird wie folgt geschrieben:
\[
  \pxy[P]{x,y}
\]
Ein \textbf{Vektor} kann beliebig verschoben werden, er hat also keine feste Position, aber er zeigt in eine bestimmte Richtung. Ein Vektor $\vv{v}$ mit den \textbf{Komponenten} $v_{x}$ und $v_{y}$ wird folgendermassen geschrieben:
\[
  \vv{v} = \vxy{v_{x},v_{y}}
\]

% ------------------------------------------------------------------------------
\subsection{Ortsvektor}

Wird ein Vektor $\vv{OP}$ mit dem Ursprung $O$ als Anfangspunkt und dem Punkt $\pxy[P]{x,y}$ als Endpunkt definiert wird, so entsprechen die Komponenten des Vektors gerade den Koordinaten des Punkts $P$:
\[
  \vv{OP} = \vxy{x-0,y-0} = \vxy{x,y}
\]
\begin{center}
  \begin{tikzpicture}
    \tkzInit[xmin=-0.5,xmax=7,ymin=-0.5,ymax=3]
    \tkzDrawXY
    \tkzDefPoint(0,0){O}
    \tkzDefPoint(6,2){P}

    \tkzDrawSegment[red,thick,vector style](O,P)
    \tkzLabelSegment[red,above left](O,P){$\vv{OP}=\vxy{x,y}$}

    \tkzDrawPoints(O,P)
    \tkzLabelPoint[above left](O){$O$}
    \tkzLabelPoint[above right](P){$P\pxy{x,y}$}
  \end{tikzpicture}
\end{center}

Dieser Vektor wird \textbf{Ortsvektor} des Punkts $P$ genannt. Der Ortsvektor von $P$ zeigt vom Ursprung $O$ zum Punkt $P$.

Werden Vektoren als Verschiebungen aufgefasst, so verschiebt der Ortsvektor den Ursprung $O$ in den Punkt $P$.
\begin{example}
  \textbf{Beispiel:} Der Ortsvektor des Punkts $\pxy[A]{-3,5}$ lautet $\vv{OA} = \vxy{-3,5}$.
\end{example}

% ------------------------------------------------------------------------------
\subsection{Vektor zwischen Punkten}

Ein Vektor kann durch einen Anfangspunkt $A$ und einen Endpunkt $B$ bestimmt werden.
\begin{center}
  \begin{tikzpicture}
    \tkzInit[xmin=-0.5,xmax=10,ymin=-0.5,ymax=4]
    \tkzDrawXY
    \tkzDefPoint(0,0){O}
    \tkzDefPoint(2,3){A}
    \tkzDefPoint(9,1){B}

    \tkzDrawSegment[vector style](O,A)
    \tkzLabelSegment[above left](O,A){$\vv{OA}$}

    \tkzDrawSegment[vector style](O,B)
    \tkzLabelSegment[above left](O,B){$\vv{OB}$}

    \tkzDrawSegment[red,thick,vector style](A,B)
    \tkzLabelSegment[red,above right](A,B){$\vv{AB}$}

    \tkzDrawPoints(A,B)
    \tkzLabelPoints[above left](O)
    \tkzLabelPoint[above](A){$A\pxy{x_{A},y_{A}}$}
    \tkzLabelPoint[right](B){$B\pxy{x_{B},y_{B}}$}
  \end{tikzpicture}
\end{center}
Der Vektor $\vv{AB}$, welcher von Punkt $A$ nach Punkt $B$ zeigt, kann als Subtraktion der Ortsvektoren von $A$ und $B$ ausgedrückt werden.
\[
  \vv{AB} = \vv{OB} - \vv{OA} = \vxy{x_{B},y_{B}} - \vxy{x_{A},y_{A}} = \vxy{x_{B}-x_{A},y_{B}-y_{A}}
\]
Daraus lässt sich schliessen, dass sich die Komponenten des Vektors $\vv{AB}$ durch Subtraktion der Koordinaten von Punkt $B$ und Punkt $A$ ergeben.
\begin{theorem}
  \textbf{Vektor zwischen Punkten.} Hat ein Vektor $\vv{AB}$ den Anfangspunkt $\pxy[A]{x_{A},y_{A}}$ und den Endpunkt $\pxy[B]{x_{B},y_{B}}$ besitzt, so lautet seine Komponentendarstellung
  \[
    \vv{AB} = \vxy{x_{B}-x_{A},y_{B}-y_{A}}
  \]
\end{theorem}
\begin{example}
  \textbf{Beispiel:} Im der oben stehenden Abbildung ist der Vektor $\vv{v}$ durch den Anfangspunkt $\pxy[A]{2,3}$ und den Endpunkt $\pxy[B]{9,1}$ definiert. Also lautet seine Komponentendarstellung
  \[
    \vv{AB} = \vxy{9-2,1-3} = \vxy{7,-2}
  \]
\end{example}

  \newpage
\section{Kollinearität und Orthogonalität}

Eine wichtige Frage ist es, ob zwei Vektoren parallel zueinander sind oder senkrecht aufeinander stehen.

\subsection{Kollinearität (parallel)}

Sind zwei Vektoren parallel, so werden sie als kollinear bezeichnet. Zwei Vektoren $\vv{a}$ und $\vv{b}$ sind parallel, wenn Sie in die gleiche oder umgekehrte Richtung zeigen. Das ist genau dann der Fall, wenn der eine Vektor ein vielfaches des anderen Vektors ist.

\begin{theorem}
  \textbf{Kollinearität.} Zwei Vektoren $\vv{a} = \vxy{a_{x}}{a_{y}}$ und $\vv{b} = \vxy{b_{x}}{b_{y}}$ sind kollinear (parallel), wenn es eine Zahl $\lambda$ gibt, sodass
  \[
    \vv{a} = \lambda\cdot\vv{b} \qquad\Leftrightarrow\qquad \vxy{a_{x}}{a_{y}} = \lambda\cdot\vxy{b_{x}}{b_{y}}
  \]
\end{theorem}

\begin{center}
  \begin{tikzpicture}
    \tkzInit[xmin=0,xmax=12,ymin=0,ymax=4]
    \tkzAxeXY[]
    \tkzGrid[color=lightgray]

    \tkzDefPoint(1,3){A1}
    \tkzDefShiftPoint[A1](2,-1){A2}
    \tkzDrawSegment[red,thick,vector style](A1,A2)
    \tkzLabelSegment[red,above right](A1,A2){$\vv{a}$}

    \tkzDefPoint(9,1){B1}
    \tkzDefShiftPoint[B1](-6,3){B2}
    \tkzDrawSegment[red,thick,vector style](B1,B2)
    \tkzLabelSegment[red,above right](B1,B2){$\vv{b}$}

    \tkzDefPoint(3,0){C1}
    \tkzDefShiftPoint[C1](2,2){C2}
    \tkzDrawSegment[thick,vector style](C1,C2)
    \tkzLabelSegment[below right](C1,C2){$\vv{c}$}
  \end{tikzpicture}
\end{center}

\begin{example}
  \textbf{Beispiele:} Die Vektoren $\vv{a} = \vxy{2}{-1}$ und $\vv{b} = \vxy{-6}{3}$ sind kollinear, da:
  \[
     2 = \mathcolor{red}{-\frac{1}{3}\cdot} (-6) \qquad\qquad -1 = \mathcolor{red}{-\frac{1}{3}\cdot} 3
  \]

  Die Vektoren $\vv{a} = \vxy{2}{-1}$ und $\vv{c} = \vxy{2}{2}$ sind nicht parallel, da
  \[
    2 = \mathcolor{red}{1\cdot} 2 \qquad\qquad -1 = \mathcolor{red}{-\frac{1}{2}\cdot} 2
  \]
\end{example}

% ------------------------------------------------------------------------------
\subsection{Orthogonalität (senkrecht)}

Stehen zwei Vektoren senkrecht zueinander, so werden sie als \textbf{orthogonal} bezeichnet.

\begin{theorem}
  \textbf{Orthogonalität.} Zwei Vektoren $\vv{a} = \vxy{a_{x}}{a_{y}}$ und $\vv{b} = \vxy{b_{x}}{b_{y}}$ sind orthogonal (stehen senkrecht zueinander), wenn
  \[
    a_{x}\cdot b_{x}+a_{y}\cdot b_{y} = 0
  \]
\end{theorem}

\begin{center}
  \begin{tikzpicture}
    \tkzInit[xmin=0,xmax=12,ymin=0,ymax=4]
    \tkzAxeXY[]
    \tkzGrid[color=lightgray]

    \tkzDefPoint(1,3){A1}
    \tkzDefShiftPoint[A1](2,-3){A2}
    \tkzDrawSegment[red,thick,vector style](A1,A2)
    \tkzLabelSegment[red,below left](A1,A2){$\vv{a}$}

    \tkzDefPoint(3,0){B1}
    \tkzDefShiftPoint[B1](6,4){B2}
    \tkzDrawSegment[red,thick,vector style](B1,B2)
    \tkzLabelSegment[red,below right](B1,B2){$\vv{b}$}

    \tkzDefPoint(3,0){C1}
    \tkzDefShiftPoint[C1](1,4){C2}
    \tkzDrawSegment[thick,vector style](C1,C2)
    \tkzLabelSegment[below right](C1,C2){$\vv{c}$}

    \tkzMarkRightAngle[red,german,size=0.5](B2,B1,A1)
  \end{tikzpicture}
\end{center}

\begin{example}
  \textbf{Beispiele:} Die Vektoren $\vv{a} = \vxy{2}{-3}$ und $\vv{b} = \vxy{6}{4}$ sind orthogonal, da:
  \[
    2\cdot 6 + (-3)\cdot 4 = 12 - 12 = 0
  \]
  Die Vektoren $\vv{a} = \vxy{2}{-3}$ und $\vv{c} = \vxy{1}{4}$ sind nicht orthogonal, da
  \[
    2\cdot 1 + (-3)\cdot 4 = 2-12 = -10 \ne 0
  \]
\end{example}

Ist ein Vektor in Komponentenform gegeben, so können die zu ihm senkrechten Vektoren einfach ermittelt werden:

\begin{theorem}
  \textbf{Senkrechter Vektor.} Ist der Vektor $\vv{a} = \vxy{a_{x}}{a_{y}}$ gegeben, so stehen die Vektoren
  \[
    \vxy{-a_{y}}{a_{x}} \qquad\qquad \vxy{a_{y}}{-a_{x}}
  \]
  senkrecht auf $\vv{a}$ und haben die gleiche Länge wie $\vv{a}$. Die Komponenten der senkrechten Vektoren erhält man, indem die Komponenten von $\vv{a}$ vertauscht werden und eine Komponente negiert wird.
\end{theorem}
Das lässt sich einfach überprüfen:
\begin{align*}
  a_{x}\cdot(-a_{y})+a_{y}\cdot a_{x} = -a_{x}\cdot a_{y}+a_{y}\cdot a_{x} &= 0 \\
  a_{x}\cdot a_{y}+a_{y}\cdot(-a_{x}) = a_{x}\cdot a_{y}-a_{y}\cdot a_{x} &= 0
\end{align*}

\begin{center}
  \begin{tikzpicture}
    \tkzInit[xmin=0,xmax=10,ymin=0,ymax=5]
    \tkzAxeXY[]
    \tkzGrid[color=lightgray]

    \tkzDefPoint(5,3){A1}
    \tkzDefShiftPoint[A1](2,-3){A2}
    \tkzDrawSegment[thick,vector style](A1,A2)
    \tkzLabelSegment[below left](A1,A2){$\vv{a}$}

    \tkzDefShiftPoint[A1](3,2){B2}
    \tkzDrawSegment[red,thick,vector style](A1,B2)
    \tkzLabelSegment[red,above left](A1,B2){$\vv{b}$}

    \tkzDefShiftPoint[A1](-3,-2){C2}
    \tkzDrawSegment[red,thick,vector style](A1,C2)
    \tkzLabelSegment[red,above left](A1,C2){$-\vv{b}$}

    \tkzMarkRightAngle[red,german,size=0.5](C2,A1,A2)
  \end{tikzpicture}
\end{center}

\begin{example}
  \textbf{Beispiel:} Zum $\vv{a} = \vxy{2}{-3}$ stehen die folgenden beiden Vektoren senkrecht:
  \[
    \vv{b} = \vxy{3}{2} \qquad\qquad -\vv{b} = \vxy{-3}{-2}
  \]
\end{example}

  \newpage
\section{Geraden}

% ------------------------------------------------------------------------------
\subsection{Geradengleichung}
Eine Gerade $g$ kann durch die Vektorgleichung
\[
  g:\quad \vv{r} = \vv{p} + t\cdot\vv{a}
\]
beschrieben werden. Dabei ist $\vv{p} = \vv{OP}$ der Ortsvektor eines bestimmten
\textbf{Punktes} $P$ auf der Geraden und $\vv{a}$ ist die \textbf{Richtung} der
Geraden. Der \textbf{Parameter} $t$ ist eine beliebig wählbare, reelle Zahl.

Wird für $t$ eine Zahl eingesetzt, so ergibt sich der Ortsvektor
$\vv{r} = \vv{OR}$ eines Punktes $R$, welcher auf der Geraden liegt. Da $t$
beliebig gewählt werden kann, können so sämtliche auf der Gerade liegenden
Punkte ermittelt werden.
\begin{center}
  \begin{tikzpicture}
    \tkzInit[xmin=-0.5,xmax=10,ymin=-0.5,ymax=4]
    \tkzDrawXY
    \tkzDefPoint(0,0){O}
    \tkzDefPoint(4,3){P}
    \tkzDefPoint(6,2.5){A}
    \tkzDefPoint(0,4){G1}
    \tkzDefPoint(10,1.5){G2}
    \tkzDefPoint(8,2){R1}
    \tkzDefPoint(1,3.75){R2}

    \tkzDrawLine[](G1,G2)
    \tkzLabelSegment[pos=1,above](G1,G2){$g$}

    \tkzDrawSegment[very thick,red,vector style](O,P)
    \tkzLabelSegment[above left](O,P){$\vv{p}$}

    \tkzDrawSegment[very thick,red,vector style](P,A)
    \tkzLabelSegment[above](P,A){$\vv{a}$}

    \tkzDrawSegment[very thick,blue,vector style](A,R1)
    \tkzLabelSegment[above](A,R1){$2\vv{a}$}

    \tkzDrawSegment[very thick,blue,vector style](P,R2)
    \tkzLabelSegment[above](P,R2){$-1.5\vv{a}$}

    \tkzDrawPoints(O,P,R1,R2)
    \tkzLabelPoints[above left](O)
    \tkzLabelPoints[above](P)
    \tkzLabelPoint[above](R1){$R_{1}$}
    \tkzLabelPoint[above](R2){$R_{2}$}
  \end{tikzpicture}
\end{center}
\begin{example}
  \textbf{Beispiel:} Gegeben ist die Gerade
  \[
    g:\quad \vv{r} = \vxy{4,3}+t\cdot\vxy{2,-0.5}
  \]
  Durch Einsetzen von $2$ bzw. $-1.5$ für den Parameter $t$ können zwei
  Ortsvektoren von Punkten $R_{1}$ und $R_{2}$ auf der Geraden ermittelt werden.
  \begin{align*}
    \vv{r_{1}} &= \vxy{4,3}+2\cdot\vxy{2,-0.5} = \vxy{4+4,3-1} = \vxy{8,2} \qquad&&\Rightarrow\qquad \pxy[R_{1}]{8,2} \\
    \vv{r_{2}} &= \vxy{4,3}-1.5\cdot\vxy{2,-0.5} = \vxy{4-3,3+0.75} = \vxy{1,3.75} \qquad&&\Rightarrow\qquad \pxy[R_{2}]{1,3.75}
  \end{align*}
\end{example}

\newpage
% ------------------------------------------------------------------------------
\subsection{Gerade durch zwei Punkte}

Die Gleichung der Geraden durch die zwei Punkte $A$ und $B$ ergibt sich aus dem
Richtungsvektor $\vv{a} = \vv{AB}$ und dem Ortsvektors $\vv{p} = \vv{OA}$.
Alternativ kann auch der Ortsvektor von $B$ gewählt werden. Die Vektorgleichung
der Geraden durch $A$ und $B$ lautet also:
\[
  \vv{r} = \vv{OA}+t\cdot\vv{AB}
\]
\begin{example}
  \textbf{Beispiel:} Die Gleichung der Geraden durch $\pxy[A]{2,3}$ und
  $\pxy[B]{-4,2}$ soll bestimmt werden.
  \begin{align*}
    \vv{a} &= \vv{AB} = \vv{OB}-\vv{OA} = \vxy{-4,2}-\vxy{2,3} = \vxy{-6,-1} \\
    \vv{p} &= \vv{OA} = \vxy{2,3} \\
    \vv{r} &= \vv{p}+t\cdot \vv{a} = \vxy{2,3}+t\cdot \vxy{-6,-1}
  \end{align*}
\end{example}
% ------------------------------------------------------------------------------
\subsection{Parallele durch einen Punkt}


% ------------------------------------------------------------------------------
\subsection{Senkrechte durch einen Punkt}

% ------------------------------------------------------------------------------
\subsection{Schnittpunkt zweier Geraden}

  \newpage
\section{Anwendungen}

Mit Vektoren können geometrische Probleme algebraisch gelöst werden. Hier werden erste solche Probleme betrachtet.

% ------------------------------------------------------------------------------
\subsection{Strecke}
\textbf{Aufgabe:} Teilen Sie die Strecke $\overline{AB}$ in drei gleiche Teile.
\[
  A(2\mid 4) \qquad B(11\mid 1)
\]
Zunächst wird der Vektor $\vv{a} = \vv{AB}$ bestimmt. Der erste Teilpunkt $T_{1}$ erhält man, indem ein Drittel von $\vv{a}$ zum Ortsvektor von $A$ addiert wird. Der zweite Teilpunkt erhält man durch Addieren von zwei Dritteln von $\vv{a}$.
\[
  \vv{a} := \vv{AB} \qquad\qquad \vv{OT_{1}} = \vv{OA} + \frac{1}{3}\vv{a} \qquad\qquad \vv{OT_{2}} = \vv{OA} + \frac{2}{3}\vv{a}
\]

\begin{center}
  \begin{tikzpicture}
    \tkzInit[xmin=0,xmax=12,ymin=0,ymax=5]
    \tkzAxeXY[]
    \tkzGrid[color=lightgray]
    \tkzDefPoint(0,0){O}
    \tkzDefPoint(2,4){A}
    \tkzDefPoint(11,1){B}
    \tkzDefPoint(5,3){T1}
    \tkzDefPoint(8,2){T2}
    \tkzDrawSegment[](A,B)
    \tkzDrawSegment[-LaTeX](O,A)
    \tkzDrawSegment[-LaTeX](O,T1)
    \tkzDrawSegment[-LaTeX](O,T2)
    \tkzDrawSegment[red,thick,-LaTeX](A,T1)
    \tkzDrawSegment[red,thick,-LaTeX](T1,T2)
    \tkzDrawSegment[red,thick,-LaTeX](T2,B)

    \tkzLabelSegment[above left](O,A){$\vv{OA}$}
    \tkzLabelSegment[above left](O,T1){$\vv{OT_{1}}$}
    \tkzLabelSegment[below right](O,T2){$\vv{OT_{2}}$}
    \tkzLabelSegment[red,below left](A,T1){$\frac{1}{3}\vv{a}$}
    \tkzLabelSegment[red,below left](T1,T2){$\frac{1}{3}\vv{a}$}
    \tkzLabelSegment[red,below left](T2,B){$\frac{1}{3}\vv{a}$}

    \tkzDrawPoints(A,B)
    \tkzDrawPoints[red](T1,T2)
    \tkzLabelPoint[above right](A){$A$}
    \tkzLabelPoint[above right](B){$B$}
    \tkzLabelPoint[red,above right](T1){$T_{1}$}
    \tkzLabelPoint[red,above right](T2){$T_{2}$}
  \end{tikzpicture}
\end{center}
Damit ergeben sich folgende Ortsvektoren und Koordinaten für $T_{1}$ und $T_{2}$:
\begin{align*}
       \vv{a} &= \vv{AB} = \vxy{11}{1} - \vxy{2}{4} = \vxy{11-2}{1-4} = \vxy{9}{-3} \\[2mm]
  \vv{OT_{1}} &= \vxy{2}{4}+\frac{1}{3}\vxy{9}{-3} = \vxy{2+\frac{1}{3}9}{4+\frac{1}{3}(-3)} = \vxy{5}{3} \\[2mm]
  \vv{OT_{2}} &= \vxy{2}{4}+\frac{2}{3}\vxy{9}{-3} = \vxy{2+\frac{2}{3}9}{4+\frac{2}{3}(-3)} = \vxy{8}{2}
\end{align*}
Die Strecke wird also durch die Punkte $T_{1}(5\mid 3)$ und $T_{2}(8\mid 2)$ in drei gleiche Teile geteilt.

% ------------------------------------------------------------------------------
\subsection{Dreieck}
\textbf{Aufgabe:} Bestimmen Sie den Umfang des Dreiecks $\triangle ABC$ mit
\[
  A(3\mid 2) \qquad B(8\mid 1) \qquad C(7\mid 4)
\]
Zwischen den Punkten $A$, $B$ und $C$ können die Vektoren $\vv{AB}$, $\vv{BC}$ und $\vv{CA}$ definiert werden.
\begin{center}
  \begin{tikzpicture}
    \tkzInit[xmin=0,xmax=12,ymin=0,ymax=5]
    \tkzAxeXY[]
    \tkzGrid[color=lightgray]
    \tkzDefPoint(3,2){A}
    \tkzDefPoint(8,1){B}
    \tkzDefPoint(7,4){C}
    \tkzDrawSegment[red,thick,-LaTeX](A,B)
    \tkzDrawSegment[red,thick,-LaTeX](B,C)
    \tkzDrawSegment[red,thick,-LaTeX](C,A)
    \tkzLabelSegment[red,below left](A,B){$\vv{a}$}
    \tkzLabelSegment[red,above right](B,C){$\vv{b}$}
    \tkzLabelSegment[red,above left](C,A){$\vv{c}$}

    \tkzDrawPoints(A,B,C)
    \tkzLabelPoint[below left](A){$A$}
    \tkzLabelPoint[below right](B){$B$}
    \tkzLabelPoint[above](C){$C$}
  \end{tikzpicture}
\end{center}
Anschliessend werden die Längen der Vektoren ermittelt und addiert, um den Umfang des Dreiecks zu erhalten.
\begin{align*}
  \vv{a} &= \vv{AB} = \vxy{8-3}{1-2} = \vxy{5}{-1} & |\vv{a}| &= \sqrt{5^{2}+(-1)^{2}} = \sqrt{26} \approx 5.099 \\[2mm]
  \vv{b} &= \vv{BC} = \vxy{7-8}{4-1} = \vxy{-1}{3} & |\vv{b}| &= \sqrt{(-1)^{2}+3^{2}} = \sqrt{10} \approx 3.162 \\[2mm]
  \vv{c} &= \vv{CA} = \vxy{3-7}{2-4} = \vxy{-4}{-2} & |\vv{c}| &= \sqrt{(-4)^{2}+(-2)^{2}} = \sqrt{20} \approx 4.472
\end{align*}
Damit ergibt sich für den Umfang des Dreiecks:
\[
  U = |\vv{a}|+|\vv{b}|+|\vv{c}| \approx 5.099 + 3.162 + 4.472 \approx 12.733
\]

% ------------------------------------------------------------------------------
\subsection{Parallelogramm}

\textbf{Aufgabe:} Ermitteln Sie den fehlenden Eckpunkt $C$ und den Umfang des Parallelogramms $ABCD$ mit
\[
  A(3\mid 2) \qquad B(8\mid 1) \qquad D(6\mid 4)
\]
Zuerst werden die Seitenvektoren des Parallelogramms bestimmt:
\begin{align*}
  \vv{a} &= \vv{OB}-\vv{OA} = \vxy{8}{1}-\vxy{3}{2} = \vxy{8-3}{1-2} = \vxy{5}{-1} \\[2mm]
  \vv{b} &= \vv{OD}-\vv{OA} = \vxy{6}{4}-\vxy{3}{2} = \vxy{6-3}{4-2} = \vxy{3}{2}
\end{align*}
Der Punkt $C$ erhält man, indem der Punkt $B$ um den Vektor $\vv{b}$ verschiebt. Dazu wird $\vv{b}$ zum Ortsvektor $\vv{OB}$ addiert:
\[
  \vv{OC} = \vv{OB} + \vv{b} = \vxy{8}{1} + \vxy{3}{2} = \vxy{8+3}{1+2} = \vxy{11}{3}
\]
Also ist der vierte Punkt des Parallelogramms $C(11\mid 3)$.

\begin{center}
  \begin{tikzpicture}
    \tkzInit[xmin=0,xmax=12,ymin=0,ymax=5]
    \tkzAxeXY[]
    \tkzGrid[color=lightgray]
    \tkzDefPoint(0,0){O}
    \tkzDefPoint(3,2){A}
    \tkzDefPoint(8,1){B}
    \tkzDefPoint(11,3){C}
    \tkzDefPoint(6,4){D}

    \tkzDrawSegment[-LaTeX](O,A)
    \tkzLabelSegment[above left](O,A){$\vv{OA}$}

    \tkzDrawSegment[-LaTeX](O,B)
    \tkzLabelSegment[below right](O,B){$\vv{OB}$}

    \tkzDrawSegment[thick,-LaTeX](A,B)
    \tkzLabelSegment[below left](A,B){$\vv{a}$}

    \tkzDrawSegment[thick,-LaTeX](A,D)
    \tkzLabelSegment[above left](A,D){$\vv{b}$}

    \tkzDrawSegment[red,thick,-LaTeX](B,C)
    \tkzLabelSegment[red,below right](B,C){$\vv{b}$}

    \tkzDrawSegment[red,thick,dotted](C,D)

    \tkzDrawPoints(A,B,D)
    \tkzDrawPoints[red](C)
    \tkzLabelPoint[above left](A){$A$}
    \tkzLabelPoint[below right](B){$B$}
    \tkzLabelPoint[red,right](C){$C$}
    \tkzLabelPoint[above](D){$D$}
  \end{tikzpicture}
\end{center}

Der Umfang des Parallelogramms ergibt sich, indem die Längen von $\vv{a}$ und $\vv{b}$ addiert und verdoppelt werden:
\[
  U = 2\left(|\vv{a}| + |\vv{b}|\right) = 2\left(\sqrt{5^{2}+(-1)^2}+\sqrt{3^{2}+2^{2}}\right) \approx 2\cdot 8.704 \approx 12.310
\]

% ------------------------------------------------------------------------------
\subsection{Quadrat}

\textbf{Aufgabe:} Ermitteln Sie die fehlenden Eckpunkte $C$ und $D$ sowie die Fläche des Quadrats $ABCD$ mit
\[
  A(3\mid 2) \qquad B(6\mid 1)
\]
\begin{center}
  \begin{tikzpicture}
    \tkzInit[xmin=0,xmax=10,ymin=0,ymax=5]
    \tkzAxeXY[]
    \tkzGrid[color=lightgray]
    \tkzDefPoint(3,2){A}
    \tkzDefPoint(6,1){B}
    \tkzDefPoint(7,4){C}
    \tkzDefPoint(4,5){D}
    \tkzDrawSegment[thick,-LaTeX](A,B)
    \tkzLabelSegment[below left](A,B){$\vv{a}$}

    \tkzDrawSegment[red,thick,-LaTeX](A,D)
    \tkzLabelSegment[red,above left](A,D){$\vv{b}$}
    \tkzDrawSegment[red,thick,-LaTeX](B,C)
    \tkzLabelSegment[red,below right](B,C){$\vv{b}$}
    \tkzDrawSegment[red,thick,dotted](C,D)

    \tkzDrawPoints(A,B,D)
    \tkzDrawPoints[red](C)
    \tkzLabelPoint[left](A){$A$}
    \tkzLabelPoint[below](B){$B$}
    \tkzLabelPoint[red,right](C){$C$}
    \tkzLabelPoint[red,above](D){$D$}
    \tkzMarkRightAngle[red,german,size=0.5](B,A,D)
  \end{tikzpicture}
\end{center}
Zunächst wird der Vektor $\vv{a} = \vv{AB}$ ermittelt:
\[
  \vv{a} = \vxy{6}{1}-\vxy{3}{2} = \vxy{6-3}{1-2} = \vxy{3}{-1}
\]
Nun wird der zu $\vv{a}$ senkrechte Vektor $\vv{b}$ ermittelt:
\[
  \vv{b} = \vxy{1}{3}
\]
Nun werden $C$ und $D$ ermittelt:
\begin{align*}
  \vv{OC} &= \vv{OB}+\vv{b} = \vxy{6}{1} + \vxy{1}{3} = \vxy{6+1}{1+3} = \vxy{7}{4} \\[2mm]
  \vv{OD} &= \vv{OA}+\vv{b} = \vxy{3}{2} + \vxy{1}{3} = \vxy{3+1}{2+3} = \vxy{4}{5}
\end{align*}
Die gesuchten Punkte sind also $C(7\mid 4)$ und $D(4\mid 5)$. Die Fläche des Quadrats beträgt
\[
  A = |\vv{a}|^{2} = \left(\sqrt{3^{2}+1^{2}}\right)^{2} = 9+1 = 10
\]


\newpage
% ------------------------------------------------------------------------------
\subsection{Kreis}

\textbf{Aufgabe:} Berechnen Sie den Mittelpunkt und den Radius des Kreises, auf welchem die folgenden drei Punkte liegen:
\[
  A(3\mid 2) \qquad B(6\mid 5) \qquad C(7\mid 2)
\]
\begin{center}
  \begin{tikzpicture}
    \tkzInit[xmin=0,xmax=10,ymin=0,ymax=6]
    \tkzAxeXY[]
    \tkzGrid[color=lightgray]
    \tkzDefPoint(3,2){A}
    \tkzDefPoint(6,5){B}
    \tkzDefPoint(7,2){C}
    \tkzDefPoint(5,3){M}

    \tkzDrawCircle[thick,red](M,C)

    \tkzDrawSegment[red,thick,-LaTeX](A,M)
    \tkzLabelSegment[red,below right](A,M){$\vv{AM}$}

    \tkzDrawPoints(A,B,C)
    \tkzDrawPoints[red](M)
    \tkzLabelPoints[below left](A)
    \tkzLabelPoints[above right](B)
    \tkzLabelPoints[below right](C)
    \tkzLabelPoints[red,below right](M)
  \end{tikzpicture}
\end{center}
Sei $\vv{OA}$ der Ortsvektor des Punkts $A$ auf dem Kreis und $\vv{OM}$ der Ortsvektor des Mittelpunkts $M$ mit den folgenden Komponenten:
\[
  \vv{OA} = \vxy{3}{2} \qquad \vv{OM} = \vxy{x}{y}
\]
Der Vektor $\vv{AM} = \vv{OM}-\vv{OA}$ zeigt vom Punkt $P$ zum Kreismittelpunkt $M$. Seine Länge ist der Radius des Kreises $r$:
\[
  \left|\vv{OM}-\vv{OA}\right| = r
\]
In Komponentenschreibweise ergibt sich die folgende Gleichung:
\[
  \sqrt{\left(x-3\right)^{2}+\left(y-2\right)^{2}} = r
\]
Um die Wurzel zu eliminieren wird die Gleichung quadriert.
\[
  \left(x-3\right)^{2}+\left(y-2\right)^{2} = r^{2}
\]
Für die Punkte $B$ und $C$ ergeben sich zwei analoge Gleichungen. Diese ergeben zusammen ein Gleichungssystem mit drei Gleichungen und drei Unbekannten:
\begin{align*}
  (x-3)^{2}+(y-2)^{2} &= r^{2} && \circled{1} \\
  (x-6)^{2}+(y-5)^{2} &= r^{2} && \circled{2} \\
  (x-7)^{2}+(y-2)^{2} &= r^{2} && \circled{3}
  \intertext{Die Gleichungen werden ausmultipliziert:}
  x^{2}-6x+y^{2}-4y+13 &= r^{2} && \circled{1} \\
  x^{2}-12x+y^{2}-10y+61 &= r^{2} && \circled{2} \\
  x^{2}-14x+y^{2}-4y+53 &= r^{2} && \circled{3} \\
\end{align*}
Durch Gleichsetzen der Gleichungen werden $r$ und die Quadrate eliminiert und es entsteht ein lineares Gleichungssystem:
\begin{align*}
  -6x-4y+13 &= -12x-10y+61 && \circled{1}=\circled{2} \\
  -6x-4y+13 &= -14x-4y+53 && \circled{1}=\circled{3}
  \intertext{Nach Zusammenfassen ergibt sich:}
  6x+6y &= 48 \\
  8x &= 40
\end{align*}
Die Lösung des Gleichungssystems ist $x = 5$ und $y = 3$. Also hat der Mittelpunkt des Kreises die Koordinaten $M(5\mid 3)$.

Den Radius des Kreises ist die Länge des Vektors $\vv{AM}$:
\[
  r = \left|\vv{AM}\right| = \left|\vxy{5-2}{3-2}\right| = \left|\vxy{3}{1}\right| = \sqrt{3^{2}+1^{2}} = \sqrt{10} \approx 3.162
\]

\end{document}
