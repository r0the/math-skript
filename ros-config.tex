% ------------------------------------------------------------------------------
% LaTeX-Grundkonfiguration von Stefan Rothe
% ------------------------------------------------------------------------------

% 1 Konfiguration der Schriftarten
% --------------------------------
% um \ifXeTeX verwenden zu können
\usepackage{iftex}

\ifXeTeX
  % Setze PDF-Version auf 1.7
  \special{pdf:minorversion 7}

  % 1.1 Konfiguration der Schriftarten für XeTeX (empfohlen)
  % ----------------------------------------------------------
  \usepackage{fontspec}

  \IfFontExistsTF{Helvetica}{\setmainfont{Helvetica}}{%
    \IfFontExistsTF{Arial}{\setmainfont{Arial}}{}%
  }

  \IfFontExistsTF{Helvetica}{\setsansfont{Helvetica}}{%
    \IfFontExistsTF{Arial}{\setsansfont{Arial}}{}%
  }

  \IfFontExistsTF{Menlo}{\setmonofont[SizeFeatures={Size=10}]{Menlo}}{%
    \IfFontExistsTF{Courier New}{\setmonofont[SizeFeatures={Size=10}]{Courier New}}{}%
  }
\else
  % Setze PDF-Version auf 1.7
  \pdfminorversion=7

  % 1.2 Konfiguration der Schriftarten für PdfTeX
  % -----------------------------------------------

  % Unterstützung von UTF-8 (Unicode)
  \usepackage[utf8]{inputenc}

  % Unterstützung der modernen Zeichencodierung
  \usepackage[T1]{fontenc}

  % Moderne Schriftart verwenden
  \usepackage{lmodern}

  % Schriftart Helvetica verwenden
  \usepackage{helvet}

  % serifenfreie Schriftvariante verwenden
  \renewcommand{\familydefault}{\sfdefault}
\fi

% 2 wichtige Pakete laden und konfigurieren
% -----------------------------------------

% sprachspezifische Anpassungen
\usepackage[ngerman]{babel}

% Absatzabstände kontrollieren für nicht-KOMA-Klassen
\makeatletter
\@ifclassloaded{scrartcl}{}{\usepackage{parskip}}
\makeatother

% Aufzählungen
\usepackage{enumitem}

% mehrere Spalten
\usepackage{multicol}

% Rahmen
\usepackage{tcolorbox}

% Tabellen
\usepackage{booktabs}
\usepackage{tabularx}

% Grafiken (JPG, PNG, PDF)
\usepackage{graphicx}

% Links
\usepackage{hyperref}
\hypersetup{colorlinks=true,urlcolor=blue,linkcolor=black}

% 3 Mathematik
% ------------

% 3.1 Zahlen und Einheiten schön darstellen
% -----------------------------------------
\usepackage{siunitx}
\sisetup{%
  mode=match,
  exponent-product=\cdot,
  group-digits=integer,
  group-separator={\text{\textquotesingle}}
}

% 3.2 AMS-Mathematik
% ------------------

\usepackage[fleqn]{amsmath}
\usepackage{amssymb}
\usepackage{amsthm}

% eigene Operatoren
\DeclareMathOperator{\ggT}{ggT}
\DeclareMathOperator{\kgV}{kgV}
\DeclareMathOperator{\lb}{lb}

% 3.3 Punkte und Vektoren
% -----------------------
\usepackage[b]{esvect}

\makeatletter
% Komponentenschreibweise für Punkte
\NewDocumentCommand\coord@internal{mmmm}{%
\IfNoValueTF{#3}{\mathopen{}\left(#1/\mathopen{}#2\mathclose{}\right)\mathclose{}}
{\mathopen{}\left(#1/\mathopen{}#2\mathclose{}/\mathopen{}#3\mathclose{}\right)\mathclose{}}}
\NewDocumentCommand\coord{o>{\SplitArgument{3}{,}}m}{\IfNoValueF{#1}{#1}\coord@internal #2}

% Komponentenschreibweise für Vektoren für inline-Math
\NewDocumentCommand\tvec@internal{mmmm}{%
\IfNoValueTF{#3}{\left(\begin{smallmatrix}\scriptstyle #1\\\scriptstyle #2\end{smallmatrix}\right)}
{\left(\begin{smallmatrix}\scriptstyle #1\\\scriptstyle #2\\\scriptstyle #3\end{smallmatrix}\right)}}
\NewDocumentCommand\tvec{>{\SplitArgument{3}{,}}m}{\tvec@internal #1}

% Komponentenschreibweise für Vektoren für display-Math
\NewDocumentCommand\dvec@internal{mmmm}{%
\IfNoValueTF{#3}{\left(\begin{matrix}#1\\ #2\end{matrix}\right)}
{\left(\begin{matrix} #1\\ #2\\ #3\end{matrix}\right)}}
\NewDocumentCommand\dvec{>{\SplitArgument{3}{,}}m}{\dvec@internal #1}
\makeatother

% 3.4 Umgebung eqt für Äquivalenzumformungen von Gleichungen
% ----------------------------------------------------------
\makeatletter
\newcounter{@eqtrownumber}
\def\eqtequiv{\stepcounter{@eqtrownumber}\ifnum\value{@eqtrownumber}=1\else\Leftrightarrow\qquad\qquad\fi}
\makeatother
\NewDocumentEnvironment{eqt}{}{\begin{equation*}\begin{array}{@{\eqtequiv}>{\displaystyle}r@{\hspace{2mm}}>{\displaystyle}l@{\hspace{1cm}}|l}}{\end{array}\end{equation*}}
\preto\eqt{\setcounter{@eqtrownumber}{0}}

% 3.5 Lineare Gleichungssysteme
% -----------------------------
\usepackage{systeme}
\setsysteme{delim={|,|}}

% 3.6 Differentiale
% -----------------
\usepackage{derivative}

% 3.7 Umgebung für Beispiele
% --------------------------
% Definition eines neuen Theorem-Stils für Beispiele
\newtheoremstyle{example}
  {3pt}% Platz oberhalb des Beispiels
  {3pt}% Platz unterhalb des Beispiels
  {\itshape}% Schriftart des Beispiel-Textes
  {}% Abstand nach dem Beispiel-Kopf
  {\bfseries\itshape}% Schriftart des Beispiel-Kopfes
  {.}% Punktierung nach dem Beispiel-Kopf
  {.5em}% Abstand nach der Punktierung
  {\thmname{#1}\thmnumber{ #2}\thmnote{ (#3)}}% Kopfname spezifizieren

\theoremstyle{example}
\newtheorem*{example}{Beispiel}

% 4 Diagramme
% -----------

% 4.1 TikZ
% --------
% muss wegen der Konfiguration vor tikz eingebunden werden
% Mit table können Tabellenzellen eingefärbt werden
\usepackage[table]{xcolor}

% für Geometrie (TikZ-Erweiterung)
% damit wird auch tikz eingebunden
\usepackage{tkz-base}

% für Funktionsplots
\usepackage{tkz-fct}

% Konfiguration Darstellung von Tangenten in tkz-fct
\tkzfctset{tan style/.style={red,thick,>=}}

% für Geometrie
\usepackage{tkz-euclide}

% Kürzen bei Brüchen zeigen
\usepackage{cancel}

% Schriftliche Division
\usepackage{longdivision}
\longdivisionkeys{style=german}

% Bäume mit tikz
\usepackage{forest}

% 5 Informatik
% ------------

% 5.1 Quellcode
% -------------
\usepackage{listings}

\definecolor{pythonCommentColor}{HTML}{9F9F9F}
\definecolor{pythonIdentifierColor}{HTML}{02007C}
\definecolor{pythonKeywordColor}{HTML}{6B134A}
\definecolor{pythonNumberColor}{HTML}{9E4A1A}
\definecolor{pythonStringColor}{HTML}{215912}
\lstset{
  basicstyle=\ttfamily\small,
  breaklines,
  commentstyle={\color{pythonCommentColor}\itshape},
  frame=single,
  keywordstyle={\color{pythonKeywordColor}\bfseries},
  language=Python,
  numbers=left,
  numbersep=5pt,
  numberstyle=\scriptsize\color{black!70},
  showstringspaces=false,
  stringstyle={\color{pythonStringColor}},
}

% Material Design-Farben
% ----------------------
\definecolor{lightblue}{HTML}{BBDEFB} % MD Blue 100
\definecolor{lightred}{HTML}{FFCDD2} % MD Red 100
\definecolor{lightgrey}{HTML}{F5F5F5} % MD Gray 100
\definecolor{lightgreen}{HTML}{C8E6C9} % MD Green 100
\definecolor{lightorange}{HTML}{FFE0B2} % MD Orange 100

\definecolor{theoremcolor}{HTML}{FFEBEE} % MD Red 50
\definecolor{notecolor}{HTML}{FFECB3} % MD Amber 100

\definecolor{red}{HTML}{D50000} % MD Red A700
\definecolor{green}{HTML}{31843F} % MD Green A700
\definecolor{blue}{HTML}{2962FF} % MD Blue A700
\definecolor{teal}{HTML}{00BFA5} % MD Teal A700
\definecolor{cyan}{HTML}{00B8D4} % MD Cyan A700
\definecolor{yellow}{HTML}{EE8E0D} % WordPress Colors Yellow Fire 40%

\tikzset{dim style/.append style={purple,dashed}}
\tikzset{dim fence style/.append style={purple}}
\tikzset{mark angle style/.append style={german}}

% eigene Befehle
% --------------

\tikzset{circled/.style={shape=circle,draw,inner sep=2pt}}
\NewDocumentCommand\circled{m}{\tikz[baseline=(char.base)]{\node[circled] (char) {#1};}}

% Makro \result, um das Resultat von Berechnungen hervorzuheben.
\NewDocumentCommand\result{m}{\textcolor{red}{#1}}

% Makro \extra, um schwierige Zusatzaufgaben zu markieren
\NewDocumentCommand\extra{}{$\bigstar\quad$}


\NewTColorBox{note}{}{
  parbox=false,
  colback=notecolor,
  colframe=black,
  arc=0mm,
  before skip=2mm,
  left=1mm,
  right=1mm,
  boxrule=1pt,
}
\NewTColorBox{theorem}{}{
  parbox=false,
  colback=theoremcolor,
  colframe=black,
  arc=0mm,
  before skip=2mm,
  left=1mm,
  right=1mm,
  boxrule=1pt,
}
\NewTColorBox{instructions}{}{
  parbox=false,
  colback=notecolor,
  colframe=black,
  arc=0mm,
  before skip=2mm,
  left=1mm,
  right=1mm,
  boxrule=1pt,
}

\usepackage{fontawesome5}
\def\digital{\faLaptop{} }
\def\present{\faTrophy{} }

\def\rosconfig{1}
