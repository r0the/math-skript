\newpage
\section{Reelle Zahlen $\mathbb{R}$}

So wie uns die Subtraktion von den natürlichen Zahlen zu den ganzen Zahlen geführt hat und wir mit der Division auf die rationalen Zahlen gestossen sind, so führt das Wurzelziehen zu den reellen Zahlen. Denn leider ist die Wurzel aus einer rationalen Zahl nicht immer eine rationale Zahl. Das berühmteste Beispiel hierfür ist sicher $\sqrt{2}$.

% ----------------------------------------------------------------------------
\subsection{irrationale Zahlen}

Wir haben gesehen, dass die rationalen Zahlen auf der Zahlengerade dicht liegen. Das heisst, wir finden zwischen zwei rationalen Zahlen immer unendlich viele weitere.

Wir können aber auch Zahlen konstruieren, die nicht rational sind. Dazu müssen wir nur eine unendliche Dezimalentwicklung angeben, die nicht periodisch ist, beispielsweise:
\[
  0.1011011101111011111011111101\ldots
\]

Diese Zahl liegt natürlich auch auf der Zahlengerade, zwischen $0.1$ und $0.11$, sie ist aber nicht eine rationale Zahl, da sie nicht als Bruch darstellbar ist.

\textbf{Definition:} Eine Zahl auf der Zahlengeraden, welche nicht eine rationale Zahl ist, also \textbf{nicht} in der Form
\[
  \frac{z}{n} \qquad z \in \mathbb{Z},\quad n \in \mathbb{Z^{+}}
\]
geschrieben werden kann, heisst \textbf{irrationale} Zahl.

Werden alle rationalen und alle irrationalen Zahlen zusammengefasst, ergeben sich sämtliche Zahlen auf der Zahlengeraden.

\textbf{Definition:} Alle rationalen und irrationalen Zahlen zusammen werden \textbf{reelle Zahlen} genannt und mit $\mathbb{R}$ bezeichnet.

% ----------------------------------------------------------------------------
\subsection{Beweis der Irratonalität von Zahlen}

Es ist gar nicht so einfach zu beweisen, dass eine bestimmte Zahl irrational ist.

\begin{itemize}
  \item Dass $\sqrt{2}$ und weitere Wurzeln irrational sind, hat der Grieche Archytas von Tarent ca. 420 v. Chr. bewiesen.
  \item Dass die Kreiszahl $\pi$ irrational ist, hat Johann Heinrich Lambert 1761 bewiesen.
  \item Dass die eulersche Zahl $e$ irrational ist, hat Charles Hermite 1873 bewiesen.
  \item Bei anderen Zahlen wird die Irrationalität vermutet, ist aber bis heute nicht bewiesen, z.B. bei
  \[
    \pi + e \qquad\qquad \pi - e \qquad\qquad \pi^{\pi}
  \]
\end{itemize}

\newpage
% ----------------------------------------------------------------------------
\subsection{Irrationalität von Wurzeln}

Der Beweis, dass bestimmte Wurzeln irrational sind, ist von Euklid überliefert worden und ist relativ einfach nachvollziehbar, hier am Beispiel von $\sqrt{5}$

\begin{theorem}
  \textbf{Satz:} Die Wurzel von fünf ist keine rationale Zahl.
\end{theorem}

\textbf{Beweis:} Der Satz wird mit einem Widerspruchsbeweis bewiesen. Das bedeutet, dass das Gegenteil angenommen und gezeigt wird, dass diese Annahme zu einem Widerspruch führt.

Es wird also angenommen, dass $\sqrt{5}$ eine rationale Zahl ist. Jede rationale Zahl kann als vollständig gekürzten Bruch dargestellt werden:
\[
  \sqrt{5} = \frac{z}{n} \qquad z \in \mathbb{Z}, \quad n \in \mathbb{Z^{+}}
\]

Nun wird die Gleichung auf beiden Seiten quadriert:
\[
  5 = \left(\frac{z}{n}\right)^{2} = \frac{z^{2}}{n^{2}}
\]
Die Gleichung kann wie folgt geschrieben werden:
\[
  \frac{5}{1} = \frac{z^{2}}{n^{2}}
\]
Diese Gleichung hat auf beiden Seiten einen vollständig gekürzten Bruch, da $z$ und $n$ gemäss Annahme keine gemeinsamen Teiler haben. Zwei vollständig gekürzte Brüche sind gleich, wenn Zähler und Nenner gleich sind. Also folgt:
\[
  5 = z^{2} \qquad\text{und}\qquad 1 = n^{2}
\]
Es gibt aber keine ganze Zahl $z$, deren Quadrat $5$ ist. Damit ist ein Widerspruch aufgezeigt worden. Daraus folgt, dass die Annahme, dass $\sqrt{5}$ eine rationale Zahl ist, falsch sein muss.
\begin{note}
  \textbf{Anmerkung:} Um den Beweis richtig nachvollziehen zu können ist wichtig zu verstehen, was es bedeutet, dass der Bruch $\frac{n}{z}$ vollständig gekürzt ist. Dies bedeutet, dass $z$ und $n$ keine gemeinsamen Teiler haben, die noch gekürzt werden könnten. Insbesondere sind in der Primfaktorzerlegung alle Primfaktoren von $z$ und $n$ verschieden. Beim Quadrieren einer Zahl kommen keine neuen Primfaktoren hinzu, die vorhandenen Primfaktoren werden nur jeweils verdoppelt. Deshalb haben auch $z^{2}$ und $n^{2}$ keinen gemeinsamen Teiler und der Bruch $\frac{z^{2}}{n^{2}}$ ist ebenfalls ein gekürzter Bruch.
\end{note}

% ----------------------------------------------------------------------------
\subsection{Intervalle}

Manchmal soll ein Intervall auf der Zahlengeraden beschrieben werden, beispielsweise alle Zahlen zwischen $-1$ und $1$. Dabei muss angegeben werden, ob die Grenzen des Intervalls dazu gerechnet werden oder nicht.

Es werden die folgenden Begriffe und Schreibweisen verwendet:

\begin{center}
  \renewcommand{\arraystretch}{1.1}
  \begin{tabularx}{0.9\textwidth}{XXX}
    \textbf{Begriff} & \textbf{Schreibweise} & \textbf{Definition} \\
  \toprule
    geschlossenes Intervall & $[a, b]$ & $\{ x\in\mathbb{R} \;|\; a \leq x \leq b\}$ \\
  \midrule
    linksoffenes Intervall & $(a, b]$ oder $]a, b]$ & $\{ x\in\mathbb{R} \;|\; a< x \leq b\}$ \\
  \midrule
    rechtsoffenes Intervall & $[a, b)$ oder $[a, b[$ & $\{ x\in\mathbb{R} \;|\; a \leq x < b\}$ \\
  \midrule
    offenes Intervall & $(a, b)$ oder $]a, b[$ & $\{ x\in\mathbb{R} \;|\; a< x < b\}$ \\
  \bottomrule
  \end{tabularx}
\end{center}

Linksoffene und rechtsoffene Intervalle werden zusammen auch als halboffene Intervalle bezeichnet.

% ----------------------------------------------------------------------------
\subsection{Näherungen}

Irrationale Zahlen können wir nicht exakt als Dezimalzahl notieren. In diesem Fall geben wir eine Annäherung an und verwenden wir das Zeichen $\approx$ um zu verdeutlichen, dass die Angabe nicht exakt ist:
\[
  \sqrt{2} \approx 1.414 \qquad\qquad \pi \approx 3.141
\]
