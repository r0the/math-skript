\newpage
\section{Lineare Gleichungen}

Um eine einfache Gleichung mit einer Variable zu lösen, formen wir die Gleichung mit Äquivalenzumformungen um, bis die Variable isoliert auf einer Seite der Gleichung steht:
\[
  x = \square
\]

In dieser Situation können wir auf der anderen Seite den Wert der Variable ablesen. Nun müssen wir noch überprüfen, ob der Wert auch in der Definitionsmenge $\mathbb{D}$ enthalten ist.

\begin{theorem}
  Das \textbf{Vorgehen für das Lösen einer Gleichung} ist also:
  \begin{enumerate}
    \item Grundmenge $\mathbb{G}$ festlegen.
    \item Definitionsmenge $\mathbb{D}$ ermitteln und angeben.
    \item Gleichung nach der Variable auflösen.
    \item Überprüfen, ob der ermittelte Wert in der Definitionsmenge $\mathbb{D}$ liegt.
    \item Lösungsmenge $\mathbb{L}$ angeben.
  \end{enumerate}
\end{theorem}

% ------------------------------------------------------------------------------
\subsection{Lösungsstrategie}

\subsubsection{Ausmultiplizieren}

Wenn die Terme auf der linken oder rechten Seite der Gleichung als \textbf{Faktoren} vorliegen, sollten diese zunächst \textbf{ausmultipliziert} werden..

\begin{example}
  \textbf{Beispiel:}
  \[\begin{eqt}
     (x+1)(x+6) &= (x+3)^{2} & \text{ausmultiplizieren} \\
   x^{2}+6x+x+6 &= x^{2}+6x+9
  \end{eqt}\]
\end{example}

\subsubsection{Quadrate eliminieren}

Wenn auf beiden Seiten der Gleichung die Variable im Quadrat vorliegt, kann dies auf beiden Seiten der Gleichung subtrahiert werden.

\begin{example}
  \textbf{Beispiel:}
  \[\begin{eqt}
    x^{2}+6x+x+6 &= x^{2}+6x+9 & -x^{2} \\
          6x+x+6 &= 6x+9
  \end{eqt}\]
\end{example}

Falls das Quadrat der Variable nicht wegfällt, liegt eine \textbf{quadratische Gleichung} vor. Wie solche Gleichungen gelöst werden, schauen wir später an.

\subsubsection{Terme vereinfachen}

Durch Auflösen von Klammern und zusammenfassen einzelner Summanden können die Terme auf beiden Seiten der Gleichung so vereinfacht werden, dass nur noch je ein Summand mit $x$ und ein Summand als reine Zahl vorliegt:

\begin{example}
  \textbf{Beispiel:}
  \[\begin{eqt}
    3x+(1-2x) &= 10-(3x+5) & \text{vereinfachen} \\
          x+1 &= -3x+5
  \end{eqt}\]
\end{example}

\subsubsection{Variable isolieren}

Durch das Addieren eines geschickt gewählten Terms kann die Variable auf die eine und die Zahl auf die andere Seite der Gleichung gebracht werden.

\begin{example}
  \textbf{Beispiel:}
  \[\begin{eqt}
    x+1 &= -3x+5  & +3x-1 \\
     4x &= 4
  \end{eqt}\]
\end{example}

Anschliessend kann die Gleichung durch den Faktor vor der Variable dividiert werden:

\begin{example}
  \textbf{Beispiel:}
  \[\begin{eqt}
    4x &= 4 & :4 \\
     x &= 1
  \end{eqt}\]
  Die Lösungsmenge ist $\mathbb{L} = \{1\}$.
\end{example}

Nun liegt die Gleichung in der gewünschten Form vor und die mögliche Lösung kann abgelesen werden.

\newpage
% ------------------------------------------------------------------------------
\subsection{Spezialfälle}

Wir treffen auch Gleichungen an, welche nicht in die oben angegebene Form gebracht werden können. Das ist insbesondere dann der Fall, wenn die Variable auf beiden Seiten der Gleichung wegfällt.

Dann erhalten wir eine Gleichung ohne Variable, also eine Aussage, welche wahr oder falsch ist.

Wenn die Aussage wahr ist, wissen wir, dass wir sämtliche Werte der Definitionsmenge in die Gleichung einsetzen können. Somit ist die \textbf{Lösungsmenge} gleich der \textbf{Definitionsmenge}.

\begin{example}
  \textbf{Beispiel:} Hier fällt der Term $3x$ auf beiden Seiten der Gleichung weg. Die resultierende Aussage ist richtig. Somit ist die Lösungsmenge gleich der Definitionsmenge.
  \[\begin{eqt}
    3x-2 &= 3x-2 & -3x \\
      -2 &= -2
  \end{eqt}\]
  Die Definitionsmenge ist $\mathbb{D} = \mathbb{R}$. Die Lösungsmenge ist $\mathbb{L} = \mathbb{D} = \mathbb{R}$.
\end{example}

Wenn die Aussage falsch ist, dann kann es auch kein Wert für die Variable geben, welcher die Gleichung wahr machen würde. In diesem Fall ist die \textbf{Lösungsmenge} die \textbf{leere Menge}.

\begin{example}
  \textbf{Beispiel:} Hier fällt der Term $3x$ auf beiden Seiten der Gleichung weg. Die resultierende Aussage ist falsch. Somit gibt es keine Lösungen.
  \[\begin{eqt}
    3x+4 &= 3x+1 & -3x \\
       4 &= 1
  \end{eqt}\]
  Die Lösungsmenge ist $\mathbb{L} = \{\}$.
\end{example}
