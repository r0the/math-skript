\documentclass[parskip=half]{scrartcl}
% ------------------------------------------------------------------------------
% LaTeX-Grundkonfiguration von Stefan Rothe
% ------------------------------------------------------------------------------

% 1 Konfiguration der Schriftarten
% --------------------------------
% um \ifXeTeX verwenden zu können
\usepackage{iftex}

\ifXeTeX
  % Setze PDF-Version auf 1.7
  \special{pdf:minorversion 7}

  % 1.1 Konfiguration der Schriftarten für XeTeX (empfohlen)
  % ----------------------------------------------------------
  \usepackage{fontspec}

  \IfFontExistsTF{Helvetica}{\setmainfont{Helvetica}}{%
    \IfFontExistsTF{Arial}{\setmainfont{Arial}}{}%
  }

  \IfFontExistsTF{Helvetica}{\setsansfont{Helvetica}}{%
    \IfFontExistsTF{Arial}{\setsansfont{Arial}}{}%
  }

  \IfFontExistsTF{Menlo}{\setmonofont[SizeFeatures={Size=10}]{Menlo}}{%
    \IfFontExistsTF{Courier New}{\setmonofont[SizeFeatures={Size=10}]{Courier New}}{}%
  }
\else
  % Setze PDF-Version auf 1.7
  \pdfminorversion=7

  % 1.2 Konfiguration der Schriftarten für PdfTeX
  % -----------------------------------------------

  % Unterstützung von UTF-8 (Unicode)
  \usepackage[utf8]{inputenc}

  % Unterstützung der modernen Zeichencodierung
  \usepackage[T1]{fontenc}

  % Moderne Schriftart verwenden
  \usepackage{lmodern}

  % Schriftart Helvetica verwenden
  \usepackage{helvet}

  % serifenfreie Schriftvariante verwenden
  \renewcommand{\familydefault}{\sfdefault}
\fi

% 2 wichtige Pakete laden und konfigurieren
% -----------------------------------------

% sprachspezifische Anpassungen
\usepackage[ngerman]{babel}

% Absatzabstände kontrollieren für nicht-KOMA-Klassen
\makeatletter
\@ifclassloaded{scrartcl}{}{\usepackage{parskip}}
\makeatother

% Aufzählungen
\usepackage{enumitem}

% mehrere Spalten
\usepackage{multicol}

% Rahmen
\usepackage{tcolorbox}

% Tabellen
\usepackage{booktabs}
\usepackage{tabularx}

% Grafiken (JPG, PNG, PDF)
\usepackage{graphicx}

% Links
\usepackage{hyperref}
\hypersetup{colorlinks=true,urlcolor=blue,linkcolor=black}

% 3 Mathematik
% ------------

% 3.1 Zahlen und Einheiten schön darstellen
% -----------------------------------------
\usepackage{siunitx}
\sisetup{%
  mode=match,
  exponent-product=\cdot,
  group-digits=integer,
  group-separator={\text{\textquotesingle}}
}

% 3.2 AMS-Mathematik
% ------------------

\usepackage[fleqn]{amsmath}
\usepackage{amssymb}
\usepackage{amsthm}

% eigene Operatoren
\DeclareMathOperator{\ggT}{ggT}
\DeclareMathOperator{\kgV}{kgV}
\DeclareMathOperator{\lb}{lb}

% 3.3 Punkte und Vektoren
% -----------------------
\usepackage[b]{esvect}

\makeatletter
% Komponentenschreibweise für Punkte
\NewDocumentCommand\coord@internal{mmmm}{%
\IfNoValueTF{#3}{\mathopen{}\left(#1/\mathopen{}#2\mathclose{}\right)\mathclose{}}
{\mathopen{}\left(#1/\mathopen{}#2\mathclose{}/\mathopen{}#3\mathclose{}\right)\mathclose{}}}
\NewDocumentCommand\coord{o>{\SplitArgument{3}{,}}m}{\IfNoValueF{#1}{#1}\coord@internal #2}

% Komponentenschreibweise für Vektoren für inline-Math
\NewDocumentCommand\tvec@internal{mmmm}{%
\IfNoValueTF{#3}{\left(\begin{smallmatrix}\scriptstyle #1\\\scriptstyle #2\end{smallmatrix}\right)}
{\left(\begin{smallmatrix}\scriptstyle #1\\\scriptstyle #2\\\scriptstyle #3\end{smallmatrix}\right)}}
\NewDocumentCommand\tvec{>{\SplitArgument{3}{,}}m}{\tvec@internal #1}

% Komponentenschreibweise für Vektoren für display-Math
\NewDocumentCommand\dvec@internal{mmmm}{%
\IfNoValueTF{#3}{\left(\begin{matrix}#1\\ #2\end{matrix}\right)}
{\left(\begin{matrix} #1\\ #2\\ #3\end{matrix}\right)}}
\NewDocumentCommand\dvec{>{\SplitArgument{3}{,}}m}{\dvec@internal #1}
\makeatother

% 3.4 Umgebung eqt für Äquivalenzumformungen von Gleichungen
% ----------------------------------------------------------
\makeatletter
\newcounter{@eqtrownumber}
\def\eqtequiv{\stepcounter{@eqtrownumber}\ifnum\value{@eqtrownumber}=1\else\Leftrightarrow\qquad\qquad\fi}
\makeatother
\NewDocumentEnvironment{eqt}{}{\begin{equation*}\begin{array}{@{\eqtequiv}>{\displaystyle}r@{\hspace{2mm}}>{\displaystyle}l@{\hspace{1cm}}|l}}{\end{array}\end{equation*}}
\preto\eqt{\setcounter{@eqtrownumber}{0}}

% 3.5 Lineare Gleichungssysteme
% -----------------------------
\usepackage{systeme}
\setsysteme{delim={|,|}}

% 3.6 Differentiale
% -----------------
\usepackage{derivative}

% 3.7 Umgebung für Beispiele
% --------------------------
% Definition eines neuen Theorem-Stils für Beispiele
\newtheoremstyle{example}
  {3pt}% Platz oberhalb des Beispiels
  {3pt}% Platz unterhalb des Beispiels
  {\itshape}% Schriftart des Beispiel-Textes
  {}% Abstand nach dem Beispiel-Kopf
  {\bfseries\itshape}% Schriftart des Beispiel-Kopfes
  {.}% Punktierung nach dem Beispiel-Kopf
  {.5em}% Abstand nach der Punktierung
  {\thmname{#1}\thmnumber{ #2}\thmnote{ (#3)}}% Kopfname spezifizieren

\theoremstyle{example}
\newtheorem*{example}{Beispiel}

% 4 Diagramme
% -----------

% 4.1 TikZ
% --------
% muss wegen der Konfiguration vor tikz eingebunden werden
% Mit table können Tabellenzellen eingefärbt werden
\usepackage[table]{xcolor}

% für Geometrie (TikZ-Erweiterung)
% damit wird auch tikz eingebunden
\usepackage{tkz-base}

% für Funktionsplots
\usepackage{tkz-fct}

% Konfiguration Darstellung von Tangenten in tkz-fct
\tkzfctset{tan style/.style={red,thick,>=}}

% für Geometrie
\usepackage{tkz-euclide}

% Kürzen bei Brüchen zeigen
\usepackage{cancel}

% Schriftliche Division
\usepackage{longdivision}
\longdivisionkeys{style=german}

% Bäume mit tikz
\usepackage{forest}

% 5 Informatik
% ------------

% 5.1 Quellcode
% -------------
\usepackage{listings}

\definecolor{pythonCommentColor}{HTML}{9F9F9F}
\definecolor{pythonIdentifierColor}{HTML}{02007C}
\definecolor{pythonKeywordColor}{HTML}{6B134A}
\definecolor{pythonNumberColor}{HTML}{9E4A1A}
\definecolor{pythonStringColor}{HTML}{215912}
\lstset{
  basicstyle=\ttfamily\small,
  breaklines,
  commentstyle={\color{pythonCommentColor}\itshape},
  frame=single,
  keywordstyle={\color{pythonKeywordColor}\bfseries},
  language=Python,
  numbers=left,
  numbersep=5pt,
  numberstyle=\scriptsize\color{black!70},
  showstringspaces=false,
  stringstyle={\color{pythonStringColor}},
}

% Material Design-Farben
% ----------------------
\definecolor{lightblue}{HTML}{BBDEFB} % MD Blue 100
\definecolor{lightred}{HTML}{FFCDD2} % MD Red 100
\definecolor{lightgrey}{HTML}{F5F5F5} % MD Gray 100
\definecolor{lightgreen}{HTML}{C8E6C9} % MD Green 100
\definecolor{lightorange}{HTML}{FFE0B2} % MD Orange 100

\definecolor{theoremcolor}{HTML}{FFEBEE} % MD Red 50
\definecolor{notecolor}{HTML}{FFECB3} % MD Amber 100

\definecolor{red}{HTML}{D50000} % MD Red A700
\definecolor{green}{HTML}{31843F} % MD Green A700
\definecolor{blue}{HTML}{2962FF} % MD Blue A700
\definecolor{teal}{HTML}{00BFA5} % MD Teal A700
\definecolor{cyan}{HTML}{00B8D4} % MD Cyan A700
\definecolor{yellow}{HTML}{EE8E0D} % WordPress Colors Yellow Fire 40%

\tikzset{dim style/.append style={purple,dashed}}
\tikzset{dim fence style/.append style={purple}}
\tikzset{mark angle style/.append style={german}}

% eigene Befehle
% --------------

\tikzset{circled/.style={shape=circle,draw,inner sep=2pt}}
\NewDocumentCommand\circled{m}{\tikz[baseline=(char.base)]{\node[circled] (char) {#1};}}

% Makro \result, um das Resultat von Berechnungen hervorzuheben.
\NewDocumentCommand\result{m}{\textcolor{red}{#1}}

% Makro \extra, um schwierige Zusatzaufgaben zu markieren
\NewDocumentCommand\extra{}{$\bigstar\quad$}


\NewTColorBox{note}{}{
  parbox=false,
  colback=notecolor,
  colframe=black,
  arc=0mm,
  before skip=2mm,
  left=1mm,
  right=1mm,
  boxrule=1pt,
}
\NewTColorBox{theorem}{}{
  parbox=false,
  colback=theoremcolor,
  colframe=black,
  arc=0mm,
  before skip=2mm,
  left=1mm,
  right=1mm,
  boxrule=1pt,
}
\NewTColorBox{instructions}{}{
  parbox=false,
  colback=notecolor,
  colframe=black,
  arc=0mm,
  before skip=2mm,
  left=1mm,
  right=1mm,
  boxrule=1pt,
}

\usepackage{fontawesome5}
\def\digital{\faLaptop{} }
\def\present{\faTrophy{} }

\def\rosconfig{1}


\usepackage{scrlayer-scrpage}
\pagestyle{scrheadings}

\KOMAoption{DIV}{12}
\KOMAoption{toc}{listof}
\DeclareTOCStyleEntry[entryformat=\bfseries,beforeskip=2pt,linefill=\TOCLineLeaderFill]{tocline}{section}
\setcounter{tocdepth}{1}

\title{Gleichungen}
\author{Stefan Rothe\\
Laurenz Pantenburg}

\date{30.04.2025}

\newpairofpagestyles{firstpage}{%
  \cofoot{\textcopyright{} Gymnasium Kirchenfeld\\Dieses Skript steht unter einer Creative Commons Attribution 4.0 International-Lizenz.\\(CC BY 4.0)}
}

\makeatletter
\lohead{\@title}
\rohead{\@date}
\cofoot{\thepage}
\rofoot{}
\makeatother

\NewDocumentCommand{\balance}{m}{
\begin{tikzpicture}[
    scale=#1,
    very thick,
    balance style/.style={draw=blue, fill=lightblue},
    block style/.style={draw=red, fill=lightred}
  ]
  % Aufhängungen der Schalen
  \draw[blue,ultra thick] (-8,1.4) -- (-6,4);
  \draw[blue,ultra thick] (-4,1.4) -- (-6,4);
  \draw[blue,ultra thick] (8,1.4) -- (6,4);
  \draw[blue,ultra thick] (4,1.4) -- (6,4);

  % Waage - Balen
  \fill[balance style] (-1,0) -- (1,0) -- (0,4) -- cycle; % Dreieck-Basis
  \draw[balance style] (-6.2,3.8) rectangle (6.2,4.2);
  \fill[balance style] (-6,4) circle (0.1);
  \fill[balance style] (6,4) circle (0.1);
  \fill[balance style] (0,4) circle (0.1);

  % Blöcke in den Schalen
  \draw[block style] (-7,1.2) rectangle (-6,2.4);
  \draw[block style] (-6,1.2) rectangle (-5,2.4);
  \draw[block style] (7,1.2) rectangle (6,2.4);
  \draw[block style] (6,1.2) rectangle (5,2.4);

  % Schalen
  \fill[balance style] (-8,1) rectangle (-4,1.4);
  \fill[balance style] (8,1) rectangle (4,1.4);
\end{tikzpicture}
}

\begin{document}
  \maketitle
  \thispagestyle{firstpage}
  \begin{center}
    \balance{0.75}
    \vspace{5mm}

    \includegraphics[width=0.5\textwidth]{Erste Gleichung.png}
  \end{center}
  \tableofcontents
  \clearpage

  \newpage
\section{Begriffe}

% ------------------------------------------------------------------------------
\subsection{Definition}

Formal gesehen besteht eine Gleichung aus zwei Termen, die mit einem Gleichheitszeichen verbunden werden:
\[
  \square = \square
\]

% ------------------------------------------------------------------------------
\subsection{Geschichte}

Das Gleichheitszeichen, welches heute verwendet wird, wurde zum ersten Mal vom walisischen Mediziner und Mathematiker Robert Recorde um 1557 verwendet. Die Gleichung $14x+15=71$ hatte er in seinem Buch für Arithmetik so geschrieben:
\begin{center}
  \includegraphics[width=0.4\textwidth]{Erste Gleichung.png}
\end{center}

% ------------------------------------------------------------------------------
\subsection{Aussagen}

In der Mathematik ist eine \textbf{Aussage} eine Behauptung, die entweder \textbf{wahr} oder \textbf{falsch} ist.
\begin{example}
  \begin{itemize}[noitemsep]
    \item «Heute scheint die Sonne.»
    \item «Die Wurzel von 25 ist 3.»
    \item «Ich heisse Otto.»
  \end{itemize}
\end{example}
In der Formelsprache der Mathematik werden Aussagen als \textbf{Gleichungen} geschrieben, welche \textbf{keine Variablen} enthalten.
\begin{example}
  \begin{itemize}[noitemsep]
    \item Die Aussage $\sqrt{25} = 3$ ist falsch.
    \item Die Aussage $1+1 = 2$ ist wahr.
    \item Die Aussage $1+2 = 4$ ist falsch.
  \end{itemize}
\end{example}
In der Mathematik sind normalerweise nur \textbf{wahre} Aussagen interessant. Falsche Gleichungen können durch die Verwendung des Ungleichheitszeichens $\ne$ in eine wahren Aussage verwandelt werden.
\begin{example}
  \begin{itemize}[noitemsep]
    \item Die Aussage $\sqrt{25} \ne 3$ ist wahr.
    \item Die Aussage $1+1 = 2$ ist wahr.
    \item Die Aussage $1+2 \ne 4$ ist wahr.
  \end{itemize}
\end{example}

% ------------------------------------------------------------------------------
\subsection{Aussageformen}

Eine \textbf{Aussageform} ist eine Aussage, welche eine Lücke enthält. Ob die Aussage wahr oder falsch ist, hängt davon ab, was in die Lücke gesetzt wird.
\begin{example}
  \begin{itemize}[noitemsep]
    \item «Am \underline{\hspace{1cm}} scheint die Sonne.»
    \item «Die Wurzel von \underline{\hspace{1cm}} ist 3.»
    \item «Ich heisse \underline{\hspace{1cm}}.»
  \end{itemize}
\end{example}
In der Formelsprache der Mathematik werden Aussageformen als \textbf{Gleichungen} geschrieben, welche mindestens eine Variable enthalten. Ob diese wahr oder falsch sind, kann eigentlich erst beurteilt werden, wenn bestimmte Werte für die Variablen eingesetzt werden.
\begin{example}
  $\sqrt{x} = 3$
\end{example}
Wenn in einer Gleichung jede Variable durch eine Zahl ersetzt wird, entsteht eine Aussage, welche entweder wahr oder falsch ist:
\begin{example}
  Die Aussage $\sqrt{16} = 3$ ist falsch. Die Aussage $\sqrt{9} = 3$ ist wahr.
\end{example}
Es gibt aber auch Aussageformen, also Gleichungen mit Variablen, deren Wahrheitsgehalt für alle möglichen Werte der Variablen gleich ist.
\begin{example}
  \begin{itemize}[noitemsep]
    \item Die Gleichung $(a+b)^2= a^2+2ab+b^2$ ist für alle Werte $a,b\in\mathbb{R}$ wahr.
    \item Die Gleichung $x=x+1$ ist für alle Werte $x\in\mathbb{R}$ falsch.
  \end{itemize}
\end{example}

% ------------------------------------------------------------------------------
\subsection{Definitionsmenge}

Bei einer Aussageform müssen wir festlegen, welche Werte wir überhaupt in die Lücken einsetzen können, um eine sinnvolle Aussage zu erhalten. So macht die Aussage «Die Wurzel von Otto ist 3.» keinen Sinn.
\begin{example}
  \begin{itemize}[noitemsep]
    \item «Am \underline{\hspace{1cm}} scheint die Sonne.» $\rightarrow$ Setze einen Tag ein.
    \item «Die Wurzel von \underline{\hspace{1cm}} ist 3.» $\rightarrow$ Setze eine reelle Zahl ein.
    \item «Ich heisse \underline{\hspace{1cm}}.» $\rightarrow$ Setze einen Vornamen ein.
  \end{itemize}
\end{example}
Auch bei Gleichungen mit Variablen muss festgelegt werden, welche Zahlen überhaupt für die Variable eingesetzt werden dürfen. Wie bei den Termen heisst die Menge aller erlaubten Zahlen für eine Variable $x$ die \textbf{Definitionsmenge} von $x$ und wird mit $\mathbb{D}_{x}$ bezeichnet.
\begin{example}
  $\sqrt{x} = 3 \qquad\Rightarrow\qquad \mathbb{D}_{x} = \mathbb{R}_{0}^{+} \qquad\qquad
    \frac{1}{z+1} = \frac{1}{2} \qquad\Rightarrow\qquad \mathbb{D}_{z} = \mathbb{R}\setminus\{-1\}$
\end{example}

% ------------------------------------------------------------------------------
\subsection{Arten von Gleichungen}

\subsubsection{Identitäten}

Identitäten sind Gleichungen, welche immer wahr sind, egal welche Zahlen aus der Definitionsmengen für die Variablen eingesetzt werden. Identitäten können bewiesen werden, indem die linke Seite der Gleichung mit Hilfe von bekannten Regeln in die rechte Seite umgeformt wird.
\begin{example}
  Die folgenen Gleichungen sind Identitäten:
  \begin{itemize}
    \item Die erste binomische Formel $(a+b)^2= a^2+2ab+b^2$ für $a,b \in\mathbb{R}$.
    \item Das erste Potenzgesetz $a^k\cdot a^m = a^{k+m}$ für $a \in\mathbb{R}, k,m \in \mathbb{Z}$.
  \end{itemize}
\end{example}
Identitäten sind Hilfsmittel in der Mathematik. Sobald sie bewiesen sind, können wir sie einsetzen, um unsere eigenen Umformungen abzukürzen.

\subsubsection{Definitionen}

Mit einer Definition wird ein neuer Begriff eingeführt. Dabei bezieht sich die Definition auf schon Bekanntes. In der Mathematik und Wissenschaft können Definitionen als Gleichung geschrieben werden.
\begin{example}
  Die Kreiszahl Pi $\pi$ wird als Verhältnis des Umfang $U$ eines Kreises zu seinem Durchmesser $d$ definiert:
  \[
    \pi := \frac{U}{d}
  \]
  In der Physik wird die Arbeit $W$ als die benötigte Kraft $F$ mal der zurückgelegte Weg $s$ definiert:
  \[
    W := F\cdot s
  \]
\end{example}
Dabei wird für den neuen Begriff ein Symbol definiert. Damit klar ersichtlich ist, welches das neu definierte Symbol ist, wird manchmal das Definitions-Gleichheitszeichen $:=$ verwendet. Auf der anderen Seite des Gleichheitszeichen steht ein Term, welcher angibt, wie der neue Wert aus bekannten Werten berechnet wird.

\newpage
\subsubsection{Bestimmungsgleichungen}

Bestimmungsgleichungen sind Aussageformen, also Gleichungen mit mindestens einer Variable, welche nicht notwendigerweise für alle Werte aus der Definitionsmenge eine wahre Aussage ergeben.
\begin{example}
  Die folgende Gleichung wird zu einer wahren Aussage, wenn wir für $x$ der Wert $5$ oder $-5$ eingesetzt wird:
  \[
    x^2 = 25
  \]
\end{example}

% ------------------------------------------------------------------------------
\subsection{Gleichungen Lösen und Lösungsmenge}

Das Bestimmen der Werte aus der Definitionsmenge, für welche eine Gleichung zu einer wahren Aussage wird, wird das \textbf{Lösen der Gleichung} genannt. Die Menge aller Werte, welche die Gleichung zu einer wahren Aussage machen, heisst \textbf{Lösungsmenge} und wird mit dem Zeichen $\mathbb{L}$ abgekürzt.

\begin{example}
  Die Definitionsmenge ist $\mathbb{D} = \mathbb{R}$.
  \begin{align*}
     2x = 10 \qquad&\Rightarrow\qquad \mathbb{L} = \{5\} \\
    x^2 = 25 \qquad&\Rightarrow\qquad \mathbb{L} = \{-5; 5\} \\
    x^4 = -4 \qquad&\Rightarrow\qquad \mathbb{L} = \{\}
  \end{align*}
\end{example}

Um eine Gleichung zu lösen, wird die Gleichung so umgeformt, dass eine Variable \textbf{isoliert} auf einer Seite der Gleichung steht. Das bedeutet, dass die Gleichung in die folgende Form gebracht wird:
\[
  x = \square
\]
wobei $\square$ ein Term ist, der $x$ nicht enthält. Dann kann die Lösung abgelesen werden, indem der Wert des Terms bestimmt wird.

  \input{Äquivalenzumformungen}
  \newpage
\section{Polynome}

% ------------------------------------------------------------------------------
\subsection{Definition}

Polynome sind eine spezielle Kategorie von Termen. Ein Polynom besteht aus einer Summe von Termen, die wiederum aus einer Zahl und einer natürlichen Potzenz einer bestimmten Variable bestehen. Die allgemeine Form eines Polynoms sieht so aus:
\[
  a_{n}x^{n} + a_{n-1}x^{n-1} + \cdots + a_{2}x^{2} + a_{1}x + a_{0}
\]

Dabei ist $a_{k}$ die Zahl, welche mit der $k$-ten Potenz der Variable multipliziert werden. Die Zahlen $a_{k}$ werden \textbf{Koeffizienten} genannt. Die Summanden werden in absteigender Reihenfolge der Potenzen angeordnet. Summanden, bei welchen der Koeffizient gleich Null ist, werden weggelassen.

\begin{example}
  Die folgenden Terme sind Polynome:
  \[
    5 \qquad\qquad k+3 \qquad\qquad 5x^{2}-3x+25 \qquad\qquad z^{4}-\frac{1}{2}
  \]
  Diese Terme sind keine Polynome:
  \[
    \frac{1}{x} \qquad\qquad \sqrt{k} \qquad\qquad x\cdot y
  \]
\end{example}

Polynome sind eine sehr wichtige Kategorie von Termen, welche später bei Gleichungen und Funktionen wieder auftreten.

\subsection{Grad und spezielle Bezeichnungen}

Der \textbf{Grad} eines Polynoms ist die höchste Potenz der Variable, welche im Polynom vorkommt.

\begin{example}
  $x^{3}+2x$ ist ein Polynom dritten Grades, $x^{5}+x^{4}$ ist ein Polynom fünften Grades.
\end{example}

Für die Grade 0 bis 3 existieren spezielle Bezeichnungen:
\begin{center}
  \def\arraystretch{1.1}
  \newcolumntype{R}{>{\raggedleft\arraybackslash}X}
  \begin{tabularx}{0.9\textwidth}{XXR}
  \toprule
    \textbf{Grad} & \textbf{Bezeichnung} & \textbf{Beispiel} \\
  \midrule
    0 & konstant & $42$ \\
  \midrule
    1 & linear & $7x-23$ \\
  \midrule
    2 & quadratisch & $-5x^{2}+45x-92$ \\
  \midrule
    3 & kubisch & $x^{3}+x^{2}-50x-34$ \\
  \bottomrule
  \end{tabularx}
\end{center}

Wenn also beispielsweise von einem «quadratischen Term» oder einer «linearen Gleichung» die Rede ist, ist immer ein Polynom des entsprechenden Grades gemeint.

% ------------------------------------------------------------------------------
\subsection{Polynomgleichungen}


Eine Polynomgleichung ist eine Gleichung, bei welcher auf beiden Seiten des Gleichheitszeichens ein Polynom steht.

Eine Polynomgleichung kann so umgeformt werden, dass auf der einen Seite nur die Null steht:
\[
  a_{n}x^{n} + a_{n-1}x^{n-1} + \cdots + a_{2}x^{2} + a_{1}x + a_{0} = 0
\]

% ------------------------------------------------------------------------------
\subsection{Fundamentalsatz der Algebra}

Über die Anzahl Lösungen von Polynomgleichungen gibt es einen wichtigen Satz, also eine bewiesene Aussage:

\begin{theorem}
  \textbf{Fundamentalsatz der Algebra.} Eine Polynomgleichung $n$-ten Grades ($n > 0$) hat in den reellen Zahlen maximal $n$ Lösungen.
\end{theorem}

In den nächsten Kapiteln wird das Lösen von linearen und quadratischen Polynomgleichungen ausführlich betrachtet.

  \newpage
\section{Lineare Gleichungen}

Um eine einfache Gleichung mit einer Variable zu lösen, formen wir die Gleichung mit Äquivalenzumformungen um, bis die Variable isoliert auf einer Seite der Gleichung steht:
\[
  x = \square
\]

In dieser Situation können wir auf der anderen Seite den Wert der Variable ablesen. Nun müssen wir noch überprüfen, ob der Wert auch in der Definitionsmenge $\mathbb{D}$ enthalten ist.

\begin{theorem}
  Das \textbf{Vorgehen für das Lösen einer Gleichung} ist also:
  \begin{enumerate}
    \item Grundmenge $\mathbb{G}$ festlegen.
    \item Definitionsmenge $\mathbb{D}$ ermitteln und angeben.
    \item Gleichung nach der Variable auflösen.
    \item Überprüfen, ob der ermittelte Wert in der Definitionsmenge $\mathbb{D}$ liegt.
    \item Lösungsmenge $\mathbb{L}$ angeben.
  \end{enumerate}
\end{theorem}

% ------------------------------------------------------------------------------
\subsection{Lösungsstrategie}

\subsubsection{Ausmultiplizieren}

Wenn die Terme auf der linken oder rechten Seite der Gleichung als \textbf{Faktoren} vorliegen, sollten diese zunächst \textbf{ausmultipliziert} werden..

\begin{example}
  \begin{eqt}
     (x+1)(x+6) &= (x+3)^{2} & \text{ausmultiplizieren} \\
   x^{2}+6x+x+6 &= x^{2}+6x+9
  \end{eqt}
\end{example}

\subsubsection{Quadrate eliminieren}

Wenn auf beiden Seiten der Gleichung die Variable im Quadrat vorliegt, kann dies auf beiden Seiten der Gleichung subtrahiert werden.

\begin{example}
  \begin{eqt}
    x^{2}+6x+x+6 &= x^{2}+6x+9 & -x^{2} \\
          6x+x+6 &= 6x+9
  \end{eqt}
\end{example}

Falls das Quadrat der Variable nicht wegfällt, liegt eine \textbf{quadratische Gleichung} vor. Wie solche Gleichungen gelöst werden, schauen wir später an.

\subsubsection{Terme vereinfachen}

Durch Auflösen von Klammern und zusammenfassen einzelner Summanden können die Terme auf beiden Seiten der Gleichung so vereinfacht werden, dass nur noch je ein Summand mit $x$ und ein Summand als reine Zahl vorliegt:

\begin{example}
  \begin{eqt}
    3x+(1-2x) &= 10-(3x+5) & \text{vereinfachen} \\
          x+1 &= -3x+5
  \end{eqt}
\end{example}

\subsubsection{Variable isolieren}

Durch das Addieren eines geschickt gewählten Terms kann die Variable auf die eine und die Zahl auf die andere Seite der Gleichung gebracht werden.

\begin{example}
  \begin{eqt}
    x+1 &= -3x+5  & +3x-1 \\
     4x &= 4
  \end{eqt}
\end{example}

Anschliessend kann die Gleichung durch den Faktor vor der Variable dividiert werden:

\begin{example}
  \begin{eqt}
    4x &= 4 & :4 \\
     x &= 1
  \end{eqt}
  Die Lösungsmenge ist $\mathbb{L} = \{1\}$.
\end{example}

Nun liegt die Gleichung in der gewünschten Form vor und die mögliche Lösung kann abgelesen werden.

\newpage
% ------------------------------------------------------------------------------
\subsection{Spezialfälle}

Wir treffen auch Gleichungen an, welche nicht in die oben angegebene Form gebracht werden können. Das ist insbesondere dann der Fall, wenn die Variable auf beiden Seiten der Gleichung wegfällt.

Dann erhalten wir eine Gleichung ohne Variable, also eine Aussage, welche wahr oder falsch ist.

Wenn die Aussage wahr ist, wissen wir, dass wir sämtliche Werte der Definitionsmenge in die Gleichung einsetzen können. Somit ist die \textbf{Lösungsmenge} gleich der \textbf{Definitionsmenge}.

\begin{example}
  Hier fällt der Term $3x$ auf beiden Seiten der Gleichung weg. Die resultierende Aussage ist richtig. Somit ist die Lösungsmenge gleich der Definitionsmenge.
  \begin{eqt}
    3x-2 &= 3x-2 & -3x \\
      -2 &= -2
  \end{eqt}
  Die Definitionsmenge ist $\mathbb{D} = \mathbb{R}$. Die Lösungsmenge ist $\mathbb{L} = \mathbb{D} = \mathbb{R}$.
\end{example}

Wenn die Aussage falsch ist, dann kann es auch kein Wert für die Variable geben, welcher die Gleichung wahr machen würde. In diesem Fall ist die \textbf{Lösungsmenge} die \textbf{leere Menge}.

\begin{example}
  Hier fällt der Term $3x$ auf beiden Seiten der Gleichung weg. Die resultierende Aussage ist falsch. Somit gibt es keine Lösungen.
  \begin{eqt}
    3x+4 &= 3x+1 & -3x \\
       4 &= 1
  \end{eqt}
  Die Lösungsmenge ist $\mathbb{L} = \{\}$.
\end{example}

  \newpage
\section{Quadratische Gleichungen}

Eine quadratische Gleichung, also eine Polynomgleichung zweiten Grades, hat die Form
\[
  ax^{2}+bx+c = 0
\]

Die Koeffizienten quadratischer Gleichungen werden üblicherweise mit $a$, $b$ und $c$ bezeichnet anstelle von $a_{2}$, $a_{1}$ und $a_{0}$.

% ------------------------------------------------------------------------------
\subsection{Quadratische Gleichungen mit $c=0$}

Wenn bei einer quadratischen Gleichung der Parameter $c = 0$ ist, so hat sie die folgende Form:
\[
  ax^{2} + bx = 0
\]
Hier kann der Faktor $x$ ausgeklammert werden, damit hat die Gleichung folgende Form:
\[
  x\cdot(ax+b) = 0
\]
Um bei dieser Gleichung die Lösungen zu bestimmen, wird der Satz des Nullprodukts benötigt:

\begin{theorem}
  \textbf{Satz des Nullprodukts.} Wenn ein Produkt zweier Faktoren $a$ und $b$ gleich Null ist, dann muss einer der beiden Faktoren Null sein.
  \[
    a\cdot b = 0 \qquad\Rightarrow\qquad a = 0 \quad\text{oder}\quad b = 0
  \]
\end{theorem}

Aus diesem Satz folgt, dass die Gleichung $x\cdot(ax+b) = 0$ wahr wird, wenn entweder der Faktor $x$ oder der Faktor $xa+b$ gleich Null ist. Für diese beiden Varianten wird eine \textbf{Fallunterscheidung} gemacht.

\textbf{Fall $x = 0$}. Zunächst wird der Fall betrachtet, dass $x = 0$ ist. Dieser Fall ist trivial, da der Wert für $x$ schon feststeht.

\textbf{Fall $ax+b = 0$}. In diesem Fall liegt eine lineare Gleichung vor. Es ist bereits bekannt, wie eine solche Gleichung gelöst wird:
\begin{eqt}
  ax+b &= 0            & -b \\
    ax &= -b           & :a \\
     x &= -\frac{b}{a}
\end{eqt}
Somit ist die Lösungsmenge einer solchen quadratischen Gleichung
\[
  \mathbb{L} = \left\{0;-\frac{b}{a}\right\}
\]

\begin{example}
  \begin{eqt}
    3x^{2} + 6x &= 0 & \text{ausklammern} \\
        x(3x+6) &= 0
  \end{eqt}
  Fall 1: $x = 0$: trivial.

  Fall 2: $(3x+6) = 0$:
  \begin{eqt}
    3x+6 &= 0  & -6 \\
      3x &= -6 & :3 \\
       x &= -2
  \end{eqt}
  Die Lösungsmenge ist $\mathbb{L} = \{-2;0\}$.
\end{example}


% ------------------------------------------------------------------------------
\subsection{Quadratische Gleichungen mit $b=0$}
Wenn bei einer quadratischen Gleichung der Parameter $c = 0$ ist, so hat sie die folgende Form:
\[
ax^{2} + c = 0
\]
Hier kann der Faktor $c$ subtrahiert werden, damit hat die Gleichung folgende Form:
\[
ax^2 = -c
\]
Teilen wir jetzt auf beiden Seiten durch $a$, erhalten wir:
\[
x^2 = \frac{-c}{a}
\]
Um bei dieser Gleichung die Lösungen zu bestimmen, müssen wir die Gleichung \textbf{radizieren} (die Wurzel ziehen).
\begin{eqt}
       x^{2} &= -\frac{c}{a} & \sqrt{\phantom{x}} \\
x &= \pm\sqrt{-\frac{c}{a}}
\end{eqt}

\subsubsection{Radizieren einer Gleichung}
Das Radizieren (Wurzelziehen) ist im allgemeinen \textbf{keine} Äquivalenzumformung. Wir benötigen deshalb eine \textbf{Fallunterscheidung}.

\begin{example}
	Die Gleichungen
	\[ x^2 =-4 \\ x^2 =-25 \\ x^2= -100 \]
	haben \textbf{keine} Lösung, da es keine Zahl $x$ gibt, die mit sich selbst multipliziert eine negative Zahl ergibt. Ziehen wir auf beiden Seiten die Wurzel, erhalten wir:
	\[ x =\sqrt{-4} \\ x =\sqrt{-25} \\ x= \sqrt{-100} \]
	und die Wurzel einer negativen Zahl ist \textbf{nicht definiert}. Entsprechend gib es keine Lösung.
\end{example}
Haben wir eine Gleichung der allgemeinen Form
\[
x^2 = D,
\]
so hat die Gleichung \textbf{keine Lösung}, wenn $D<0$ ist. Die Lösungsmenge ist dann die leere Menge $\mathbb{L}= \{\}$.
\begin{example}
	Die Gleichung
	\[ x^2 = 0 \]
	hat \textbf{eine} Lösung, nämlich die Zahl 0. Ziehen wir auf beiden Seiten die Wurzel, erhalten wir:
	\[ x =\sqrt{0} = 0 \]
\end{example}
Haben wir eine Gleichung der allgemeinen Form
\[
x^2 = D,
\]
so hat die Gleichung \textbf{eine Lösung}, wenn $D=0$ ist. Die Lösungsmenge ist dann die leere Menge $\mathbb{L}= \{0\}$.

Der \textit{Normalfall} ist der Folgende:
\begin{example}
	Die Gleichungen
	\[ x^2 =4 \\ x^2 =25 \\ x^2= 100 \]
	haben \textbf{zwei} Lösungen, da es zwei Zahlen $\pm x$ gibt, welche die Gleichung erfüllen. Ziehen wir auf beiden Seiten die Wurzel, erhalten wir:
	\[ x =\pm \sqrt{4} = \pm 2 \\ x = \pm \sqrt{25} = \pm 5\\ x= \pm  \sqrt{100} = \pm 10 \]
	Das $x= \pm 2$ ist eine Kurzschreibweise für $x=2$ \textbf{oder} $x=-2$, denn beide Zahlen lösen die Gleichung $x^2=4$. Das sind die beiden Fälle, die wir unterscheiden müssen:
	  \begin{eqt}
		x^2 &= 4  & \sqrt{\square} \\
		\Leftrightarrow x &=2 \text{ oder }  x= -2
	\end{eqt}

\end{example}
Haben wir eine Gleichung der allgemeinen Form
\[
  x^2 = D,
\]
so hat die Gleichung \textbf{zwei Lösungen}, wenn $D>0$ ist. Die Lösungsmenge ist dann die leere Menge $\mathbb{L}= \{-\sqrt{D},\sqrt{D}\}$.

\begin{note}
	\textbf{Bemerkung:} Das Radizieren einer Gleichung ist genau dann eine Äquivalenzumformung, wenn eine Fallunterscheidung für $\pm \sqrt{\square}$ gemacht wird \textbf{und} jeweils der \textbf{gesamte Term} einer Seite radiziert wird.
	\begin{eqt}
		\text{Term 1} &= \text{Term 2}           & \sqrt{\phantom{x}} \\
	\Leftrightarrow	\sqrt{\text{Term 1}} &= \pm\sqrt{\text{Term 2}}
	\end{eqt}
\end{note}

Die obige Bemerkung ist auch der Grund, warum Sie Gleichungen der Form $ax^2+bx+c=0$ für $b\neq0$ durch einfaches Wurzelziehen \textbf{nicht} lösen können.
Betrachten wir das folgende Beispiel:
\begin{example}
	\[ x^2 + 2x +1 = 0 \Leftrightarrow \sqrt{ x^2 + 2x +1} = \pm \sqrt{0} \]
	\[
	\Leftrightarrow x^2 + 2x +1 = 0 \Leftrightarrow x^2 = - 2x -1 \Leftrightarrow x=  \pm \sqrt{-2x-1} \]
	\[
	\Leftrightarrow x^2 + 2x +1 = 0 \Leftrightarrow x^2+2x = -1 \Leftrightarrow  \sqrt{x^2+2x}=  \pm \sqrt{-1}   \]
	Egal, was wir versuchen: Wir können die Wurzel nicht aus einer Summer ziehen!
\end{example}

\begin{note}
	\textbf{Zusammenfassung:}
	Für Gleichungen der Form
	\begin{eqt}
		x^{2} &= D           & \sqrt{\phantom{x}} \\
		\sqrt{x^{2}} &= \pm\sqrt{D} &
	\end{eqt}
	gibt es unterschiedlich viele Lösungen, je nachdem welche Zahl unter der Wurzel steht:
	\begin{itemize}
		\item $D>0$: Ist $D$ grösser als Null, so hat die Gleichung wie oben erläutert die zwei Lösungen $\pm \sqrt{D}$, also die Lösungsmenge $\mathbb{L} = \{-\sqrt{D};\sqrt{D}\}$.
		\item $D=0$: Ist $D$ gleich Null so hat die Gleichung nur eine Lösung da $-0 = +0$, also die Lösungsmenge $\mathbb{L} = \{0\}$.
		\item $D<0$: Ist $D$ kleiner als Null, so hat die Gleichung keine Lösung, da die Wurzel einer negativen Zahl nicht definiert ist, also die Lösungsmenge $\mathbb{L} = \{\}$.
	\end{itemize}
	Der Term $D$ wird \textbf{Diskriminante} genannt (diskriminieren bedeutet unterscheiden).
\end{note}


\begin{example}
  \begin{eqt}
    3x^{2} - 48 &= 0     & +48 \\
         3x^{2} &= 48    & :3  \\
          x^{2} &= 16    & \sqrt{\phantom{x}} \\
              x &= \pm \sqrt{16}
  \end{eqt}
  $D=16 > 0$, also gibt es zwei Lösungen. Die Lösungsmenge ist $\mathbb{L} = \{-4;4\}$.

  \begin{eqt}
         3x^{2} &= 0    & :3  \\
          x^{2} &= 0    & \sqrt{\phantom{x}} \\
              x &= \pm\sqrt{0}
  \end{eqt}
  $D=0$, also gibt es eine Lösung. Die Lösungsmenge ist $\mathbb{L} = \{0\}$.

  \begin{eqt}
    3x^{2} + 48 &= 0     & -48 \\
         3x^{2} &= -48   & :3  \\
          x^{2} &= -16   & \sqrt{\phantom{x}} \\
              x &= \pm \sqrt{-16}
  \end{eqt}
  $D=-16 < 0$, also gibt es keine Lösung. Die Lösungsmenge ist $\mathbb{L} = \{\}$.
\end{example}

% ------------------------------------------------------------------------------
\subsection{Lösen mit binomischen Formeln}
\begin{example}
		\begin{eqt}
		x^2 +4x +4 &= 16 & \text{Faktorisieren 1. binomische Formel} \\
		(x+2)^{2} &= 16 & \sqrt{\phantom{x}} \\
		x+2 &= \pm\sqrt{16} & -2 \\
		x &= -2 \pm 4
	\end{eqt}
	Die Lösungsmenge lautet $\mathbb{L} = \{-6;4\}$.
		\begin{eqt}
		x^2 -2x +1 &= 25 & \text{Faktorisieren 2. binomische Formel} \\
		(x-1)^{2} &= 25 & \sqrt{\phantom{x}} \\
		x-1 &= \pm\sqrt{25} & +1 \\
		x &= 1\pm 5
	\end{eqt}
	Die Lösungsmenge lautet $\mathbb{L} = \{-4;6\}$.
	\begin{eqt}
		x^2  - 81 &= 0 & \text{Faktorisieren 3. binomische Formel} \\
		(x+9)(x-9) &= 0 & \text{Nullprodukt} \\
		x &= \pm 9
	\end{eqt}
	Nach dem Satz über das Nullprodukt ist entweder $x+9=0$ oder $x-9=0$. Die Lösungsmenge lautet $\mathbb{L} = \{-9;9\}$.
\end{example}
Die binomischen Formeln können dabei helfen, quadratische Gleichungen zu  lösen. Wenn eine quadratische Gleichung durch ein Binom dargestellt werden kann, so kann sie ebenfalls durch radizieren gelöst werden:
\begin{eqt}
  (x+k)^{2} &= D & \sqrt{\phantom{x}} \\
        x+k &= \pm\sqrt{D} & -k \\
          x &= -k\pm\sqrt{D}
\end{eqt}
Damit kann abhängig von $D$ die Lösungsmenge angegeben werden:
\begin{itemize}
  \item $D>0$: zwei Lösungen $-k\pm\sqrt{D}$, also $\quad\mathbb{L} = \left\{-k-\sqrt{D};-k+\sqrt{D}\right\}$
  \item $D=0$: eine Lösung $-k$, also $\mathbb{L} = \{-k\}$
  \item $D<0$: keine Lösung, also $\mathbb{L} = \{\}$
\end{itemize}



% --------------------------------------------------------------------------
\subsection{Quadratische Ergänzung}
Mit Hilfe eines Tricks kann jede quadratische Gleichung in ein Binom umgewandelt werden. Dieser Trick wird quadratische Ergänzung genannt.

\begin{example}
  $2x^{2}+12x+2 = 0$
  Zunächst liegt keine binomische Formel vor. Betrachten Sie das Beispiel und beschreiben Sie in eigenen Worten, wie der \textit{Trick} funktioniert.
  \begin{eqt}
    2x^{2}+12x+2 &= 0           & :2 \\
      x^{2}+6x+1 &= 0           & -1 \\
        x^{2}+6x &= -1          & +9 \\
    x^{2}+6x\mathcolor{red}{+9} &= -1\mathcolor{red}{+9} & \text{zusammenfassen} \\
      x^{2}+6x+9 &= 8           & \text{Binom} \\
       (x+3)^{2} &= 8           & \sqrt{\phantom{x}} \\
             x+3 &= \pm\sqrt{8} & -3 \\
               x &= -3\pm\sqrt{8}
  \end{eqt}
\end{example}

% ------------------------------------------------------------------------------
\subsection{Allgemeine Lösungsformel}

Mit Hilfe der quadratischen Ergänzung kann die allgemeine quadratische Gleichung gelöst werden. Diese hat folgende Form:
\[
  ax^{2}+bx+c = 0
\]
Die Gleichung wird zunächst durch $a$ dividiert und dann mit Hilfe der quadratischen Ergänzung in die Binom-Form gebracht und radiziert:
\begin{eqt}
                ax^{2}+bx+c &= 0 & :a \\[4mm]
  x^{2}+\frac{b}{a}x+\frac{c}{a} &= 0 & -\frac{c}{a} \\[4mm]
              x^{2}+\frac{b}{a}x &= -\frac{c}{a} & +\left(\frac{b}{a}\right)^{2} \\[4mm]
  x^{2}+\frac{b}{a}x\mathcolor{red}{+\left(\frac{b}{2a}\right)^{2}} &= -\frac{c}{a}\mathcolor{red}{+\left(\frac{b}{2a}\right)^{2}} & \text{links mit Binom faktorisieren} \\[4mm]
  \left(x+\frac{b}{2a}\right)^{2} &= -\frac{c}{a}+\left(\frac{b}{2a}\right)^{2} & \text{rechts vereinfachen} \\[4mm]
  \left(x+\frac{b}{2a}\right)^{2} &= -\frac{c\cdot 4a}{a\cdot 4a}+\frac{b^{2}}{4a^{2}} & \text{rechts Brüche addieren} \\[4mm]
  \left(x+\frac{b}{2a}\right)^{2} &= \frac{b^{2}-4ac}{4a^{2}} & \sqrt{\phantom{x}} \\[4mm]
    x+\frac{b}{2a} &= \pm\sqrt{\frac{b^{2}-4ac}{4a^{2}}} & \text{Wurzel vereinfachen} \\[4mm]
     x+\frac{b}{2a} &= \pm\sqrt{\frac{b^{2}-4ac}{4a^{2}}} & -\frac{b}{2a} \\[4mm]
    x &=  -\frac{b}{2a}\pm\frac{\sqrt{b^{2}-4ac}}{2a} & \text{Brüche addieren} \\[4mm]
    x &= \frac{-b\pm\sqrt{b^{2}-4ac}}{2a}
\end{eqt}
Der Term unter der Wurzel gibt an, wie viele Lösungen die Gleichung hat und wird \textbf{Diskriminante} $D$ genannt.
\[
  D = b^{2}-4ac
\]
Damit ist die quadratische Gleichung allgemein gelöst!

\begin{theorem}
  \textbf{Lösungen der quadratischen Gleichung}. Für die allgemeine quadratische Gleichung
  \[
     ax^{2}+bx+c = 0
  \]
  gibt die Diskriminante $D = b^{2}-4ac$ an, wie viele Lösungen vorhanden sind:
  \begin{itemize}
  \item $D>0$ Die Gleichung hat zwei Lösungen:
  \[
    x_{1/2} =\frac{-b\pm\sqrt{b^{2}-4ac}}{2a} \\ \mathbb{L}= \{x_1,x_2\}
  \]
  \item $D=0$ Die Gleichung hat eine Lösung:
  \[
   x= -\frac{b}{2a}  \\ \mathbb{L}= \{x\}
  \]
  \item $D<0$ Die Gleichung hat keine Lösung. $\mathbb{L}= \{\}$
  \end{itemize}
\end{theorem}

\subsection{Lösen durch Faktorisieren}
Gelingt es uns, den Term $ax^2+bx+c$ zu faktorisieren, können wir dank dem Satz über das Nullprodukt die Lösungen ziemlich schnell ablesen. Dazu müssen wir lediglich schauen, wann die Faktoren Null sind. Es lohnt sich oft, kurz nach einer Faktorisierung zu suchen, bevor man die allgemeine Lösungsformel anwendet.

\begin{example}
  \begin{enumerate}[label=(\alph*)]
  \item
    \begin{eqt}
      x^2-x-12 &= 0 \\
      (x-4)(x+3) &= 0 \\
      x-4 &= 0 \text{ oder } x+3=0 \\
      \mathbb{L}=\{-3;4\}
    \end{eqt}
  \item
  \begin{eqt}
    2^2+7x+6 &= 0 \\
    (2x+3)(x+2) &= 0 \\
    2x+3 &= 0 \text{ oder } x+2=0 \\
    \mathbb{L}=\left\{-2;\frac{-3}{2}\right\}
  \end{eqt}
  \item
    \begin{eqt}
      x^2-9x-20 &= 0 \\
      \Leftrightarrow (x-4)(x-5) &= 0 \\
      \Leftrightarrow x-4 &= 0 \text{ oder } x-5=0 \\
      \mathbb{L}=\{4;5\}
    \end{eqt}
  \end{enumerate}
\end{example}

\end{document}
