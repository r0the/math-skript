\documentclass[parskip=half]{scrartcl}
% ------------------------------------------------------------------------------
% LaTeX-Grundkonfiguration von Stefan Rothe
% ------------------------------------------------------------------------------

% 1 Konfiguration der Schriftarten
% --------------------------------
% um \ifXeTeX verwenden zu können
\usepackage{iftex}

\ifXeTeX
  % Setze PDF-Version auf 1.7
  \special{pdf:minorversion 7}

  % 1.1 Konfiguration der Schriftarten für XeTeX (empfohlen)
  % ----------------------------------------------------------
  \usepackage{fontspec}

  \IfFontExistsTF{Helvetica}{\setmainfont{Helvetica}}{%
    \IfFontExistsTF{Arial}{\setmainfont{Arial}}{}%
  }

  \IfFontExistsTF{Helvetica}{\setsansfont{Helvetica}}{%
    \IfFontExistsTF{Arial}{\setsansfont{Arial}}{}%
  }

  \IfFontExistsTF{Menlo}{\setmonofont[SizeFeatures={Size=10}]{Menlo}}{%
    \IfFontExistsTF{Courier New}{\setmonofont[SizeFeatures={Size=10}]{Courier New}}{}%
  }
\else
  % Setze PDF-Version auf 1.7
  \pdfminorversion=7

  % 1.2 Konfiguration der Schriftarten für PdfTeX
  % -----------------------------------------------

  % Unterstützung von UTF-8 (Unicode)
  \usepackage[utf8]{inputenc}

  % Unterstützung der modernen Zeichencodierung
  \usepackage[T1]{fontenc}

  % Moderne Schriftart verwenden
  \usepackage{lmodern}

  % Schriftart Helvetica verwenden
  \usepackage{helvet}

  % serifenfreie Schriftvariante verwenden
  \renewcommand{\familydefault}{\sfdefault}
\fi

% 2 wichtige Pakete laden und konfigurieren
% -----------------------------------------

% sprachspezifische Anpassungen
\usepackage[ngerman]{babel}

% Absatzabstände kontrollieren für nicht-KOMA-Klassen
\makeatletter
\@ifclassloaded{scrartcl}{}{\usepackage{parskip}}
\makeatother

% Aufzählungen
\usepackage{enumitem}

% mehrere Spalten
\usepackage{multicol}

% Rahmen
\usepackage{tcolorbox}

% Tabellen
\usepackage{booktabs}
\usepackage{tabularx}

% Grafiken (JPG, PNG, PDF)
\usepackage{graphicx}

% Links
\usepackage{hyperref}
\hypersetup{colorlinks=true,urlcolor=blue,linkcolor=black}

% 3 Mathematik
% ------------

% 3.1 Zahlen und Einheiten schön darstellen
% -----------------------------------------
\usepackage{siunitx}
\sisetup{%
  mode=match,
  exponent-product=\cdot,
  group-digits=integer,
  group-separator={\text{\textquotesingle}}
}

% 3.2 AMS-Mathematik
% ------------------

\usepackage[fleqn]{amsmath}
\usepackage{amssymb}
\usepackage{amsthm}

% eigene Operatoren
\DeclareMathOperator{\ggT}{ggT}
\DeclareMathOperator{\kgV}{kgV}
\DeclareMathOperator{\lb}{lb}

% 3.3 Punkte und Vektoren
% -----------------------
\usepackage[b]{esvect}

\makeatletter
% Komponentenschreibweise für Punkte
\NewDocumentCommand\coord@internal{mmmm}{%
\IfNoValueTF{#3}{\mathopen{}\left(#1/\mathopen{}#2\mathclose{}\right)\mathclose{}}
{\mathopen{}\left(#1/\mathopen{}#2\mathclose{}/\mathopen{}#3\mathclose{}\right)\mathclose{}}}
\NewDocumentCommand\coord{o>{\SplitArgument{3}{,}}m}{\IfNoValueF{#1}{#1}\coord@internal #2}

% Komponentenschreibweise für Vektoren für inline-Math
\NewDocumentCommand\tvec@internal{mmmm}{%
\IfNoValueTF{#3}{\left(\begin{smallmatrix}\scriptstyle #1\\\scriptstyle #2\end{smallmatrix}\right)}
{\left(\begin{smallmatrix}\scriptstyle #1\\\scriptstyle #2\\\scriptstyle #3\end{smallmatrix}\right)}}
\NewDocumentCommand\tvec{>{\SplitArgument{3}{,}}m}{\tvec@internal #1}

% Komponentenschreibweise für Vektoren für display-Math
\NewDocumentCommand\dvec@internal{mmmm}{%
\IfNoValueTF{#3}{\left(\begin{matrix}#1\\ #2\end{matrix}\right)}
{\left(\begin{matrix} #1\\ #2\\ #3\end{matrix}\right)}}
\NewDocumentCommand\dvec{>{\SplitArgument{3}{,}}m}{\dvec@internal #1}
\makeatother

% 3.4 Umgebung eqt für Äquivalenzumformungen von Gleichungen
% ----------------------------------------------------------
\makeatletter
\newcounter{@eqtrownumber}
\def\eqtequiv{\stepcounter{@eqtrownumber}\ifnum\value{@eqtrownumber}=1\else\Leftrightarrow\qquad\qquad\fi}
\makeatother
\NewDocumentEnvironment{eqt}{}{\begin{equation*}\begin{array}{@{\eqtequiv}>{\displaystyle}r@{\hspace{2mm}}>{\displaystyle}l@{\hspace{1cm}}|l}}{\end{array}\end{equation*}}
\preto\eqt{\setcounter{@eqtrownumber}{0}}

% 3.5 Lineare Gleichungssysteme
% -----------------------------
\usepackage{systeme}
\sysdelim||

% 3.6 Differentiale
% -----------------
\usepackage{derivative}

% 3.7 Umgebung für Beispiele
% --------------------------
% Definition eines neuen Theorem-Stils für Beispiele
\newtheoremstyle{example}
  {3pt}% Platz oberhalb des Beispiels
  {3pt}% Platz unterhalb des Beispiels
  {\itshape}% Schriftart des Beispiel-Textes
  {}% Abstand nach dem Beispiel-Kopf
  {\bfseries\itshape}% Schriftart des Beispiel-Kopfes
  {.}% Punktierung nach dem Beispiel-Kopf
  {.5em}% Abstand nach der Punktierung
  {\thmname{#1}\thmnumber{ #2}\thmnote{ (#3)}}% Kopfname spezifizieren

\theoremstyle{example}
\newtheorem*{example}{Beispiel}

% 4 Diagramme
% -----------

% 4.1 TikZ
% --------
% muss wegen der Konfiguration vor tikz eingebunden werden
% Mit table können Tabellenzellen eingefärbt werden
\usepackage[table]{xcolor}

% für Geometrie (TikZ-Erweiterung)
% damit wird auch tikz eingebunden
\usepackage{tkz-base}

% für Funktionsplots
\usepackage{tkz-fct}

% Konfiguration Darstellung von Tangenten in tkz-fct
\tkzfctset{tan style/.style={red,thick,>=}}

% für Geometrie
\usepackage{tkz-euclide}

% Kürzen bei Brüchen zeigen
\usepackage{cancel}

% Schriftliche Division
\usepackage{longdivision}
\longdivisionkeys{style=german}

% Bäume mit tikz
\usepackage{forest}

% 5 Informatik
% ------------

% 5.1 Quellcode
% -------------
\usepackage{listings}

\definecolor{pythonCommentColor}{HTML}{9F9F9F}
\definecolor{pythonIdentifierColor}{HTML}{02007C}
\definecolor{pythonKeywordColor}{HTML}{6B134A}
\definecolor{pythonNumberColor}{HTML}{9E4A1A}
\definecolor{pythonStringColor}{HTML}{215912}
\lstset{
  basicstyle=\ttfamily\small,
  breaklines,
  commentstyle={\color{pythonCommentColor}\itshape},
  frame=single,
  keywordstyle={\color{pythonKeywordColor}\bfseries},
  language=Python,
  numbers=left,
  numbersep=5pt,
  numberstyle=\scriptsize\color{black!70},
  showstringspaces=false,
  stringstyle={\color{pythonStringColor}},
}

% Material Design-Farben
% ----------------------
\definecolor{lightblue}{HTML}{BBDEFB} % MD Blue 100
\definecolor{lightred}{HTML}{FFCDD2} % MD Red 100
\definecolor{lightgrey}{HTML}{F5F5F5} % MD Gray 100
\definecolor{lightgreen}{HTML}{C8E6C9} % MD Green 100
\definecolor{lightorange}{HTML}{FFE0B2} % MD Orange 100

\definecolor{theoremcolor}{HTML}{FFEBEE} % MD Red 50
\definecolor{notecolor}{HTML}{FFECB3} % MD Amber 100

\definecolor{red}{HTML}{D50000} % MD Red A700
\definecolor{green}{HTML}{31843F} % MD Green A700
\definecolor{blue}{HTML}{2962FF} % MD Blue A700
\definecolor{teal}{HTML}{00BFA5} % MD Teal A700
\definecolor{cyan}{HTML}{00B8D4} % MD Cyan A700
\definecolor{yellow}{HTML}{EE8E0D} % WordPress Colors Yellow Fire 40%

\tikzset{dim style/.append style={purple,dashed}}
\tikzset{dim fence style/.append style={purple}}
\tikzset{mark angle style/.append style={german}}

% eigene Befehle
% --------------

\tikzset{circled/.style={shape=circle,draw,inner sep=2pt}}
\NewDocumentCommand\circled{m}{\tikz[baseline=(char.base)]{\node[circled] (char) {#1};}}

% Makro \result, um das Resultat von Berechnungen hervorzuheben.
\NewDocumentCommand\result{m}{\textcolor{red}{#1}}

% Makro \extra, um schwierige Zusatzaufgaben zu markieren
\NewDocumentCommand\extra{}{$\bigstar\quad$}


\NewTColorBox{note}{}{
  parbox=false,
  colback=notecolor,
  colframe=black,
  arc=0mm,
  before skip=2mm,
  left=1mm,
  right=1mm,
  boxrule=1pt,
}
\NewTColorBox{theorem}{}{
  parbox=false,
  colback=theoremcolor,
  colframe=black,
  arc=0mm,
  before skip=2mm,
  left=1mm,
  right=1mm,
  boxrule=1pt,
}
\NewTColorBox{instructions}{}{
  parbox=false,
  colback=notecolor,
  colframe=black,
  arc=0mm,
  before skip=2mm,
  left=1mm,
  right=1mm,
  boxrule=1pt,
}

\usepackage{fontawesome5}
\def\digital{\faLaptop{} }
\def\present{\faTrophy{} }

\def\rosconfig{1}


\usepackage{scrlayer-scrpage}
\pagestyle{scrheadings}

\KOMAoption{DIV}{12}
\KOMAoption{toc}{listof}
\DeclareTOCStyleEntry[entryformat=\bfseries,beforeskip=2pt,linefill=\TOCLineLeaderFill]{tocline}{section}
\setcounter{tocdepth}{1}

\title{Gleichungen}
\author{Stefan Rothe\\
Laurenz Pantenburg}

\date{20.11.2024}

\newpairofpagestyles{firstpage}{%
  \cofoot{\textcopyright{} Gymnasium Kirchenfeld\\Dieses Skript steht unter einer Creative Commons Attribution 4.0 International-Lizenz.\\(CC BY 4.0)}
}

\makeatletter
\lohead{\@title}
\rohead{\@date}
\cofoot{\thepage}
\rofoot{}
\makeatother

\NewDocumentCommand{\balance}{m}{
\begin{tikzpicture}[
    scale=#1,
    very thick,
    balance style/.style={draw=blue, fill=lightblue},
    block style/.style={draw=red, fill=lightred}
  ]
  % Aufhängungen der Schalen
  \draw[blue,ultra thick] (-8,1.4) -- (-6,4);
  \draw[blue,ultra thick] (-4,1.4) -- (-6,4);
  \draw[blue,ultra thick] (8,1.4) -- (6,4);
  \draw[blue,ultra thick] (4,1.4) -- (6,4);

  % Waage - Balen
  \fill[balance style] (-1,0) -- (1,0) -- (0,4) -- cycle; % Dreieck-Basis
  \draw[balance style] (-6.2,3.8) rectangle (6.2,4.2);
  \fill[balance style] (-6,4) circle (0.1);
  \fill[balance style] (6,4) circle (0.1);
  \fill[balance style] (0,4) circle (0.1);

  % Blöcke in den Schalen
  \draw[block style] (-7,1.2) rectangle (-6,2.4);
  \draw[block style] (-6,1.2) rectangle (-5,2.4);
  \draw[block style] (7,1.2) rectangle (6,2.4);
  \draw[block style] (6,1.2) rectangle (5,2.4);

  % Schalen
  \fill[balance style] (-8,1) rectangle (-4,1.4);
  \fill[balance style] (8,1) rectangle (4,1.4);
\end{tikzpicture}
}

\begin{document}
  \maketitle
  \thispagestyle{firstpage}
  \begin{center}
    \balance{0.75}
    \vspace{5mm}

    \includegraphics[width=0.5\textwidth]{Erste Gleichung.png}
  \end{center}
  \tableofcontents
  \clearpage

  \newpage
\section{Begriffe}

% ------------------------------------------------------------------------------
\subsection{Definition}

Formal gesehen besteht eine Gleichung aus zwei Termen, die mit einem Gleichheitszeichen verbunden werden:
\[
  \square = \square
\]

% ------------------------------------------------------------------------------
\subsection{Geschichte}

Das Gleichheitszeichen, welches heute verwendet wird, wurde zum ersten Mal vom walisischen Mediziner und Mathematiker Robert Recorde um 1557 verwendet. Die Gleichung $14x+15=71$ hatte er in seinem Buch für Arithmetik so geschrieben:
\begin{center}
  \includegraphics[width=0.4\textwidth]{Erste Gleichung.png}
\end{center}

% ------------------------------------------------------------------------------
\subsection{Aussagen}

In der Mathematik ist eine \textbf{Aussage} eine Behauptung, die entweder \textbf{wahr} oder \textbf{falsch} ist.
\begin{example}
  \begin{itemize}[noitemsep]
    \item «Heute scheint die Sonne.»
    \item «Die Wurzel von 25 ist 3.»
    \item «Ich heisse Otto.»
  \end{itemize}
\end{example}
In der Formelsprache der Mathematik werden Aussagen als \textbf{Gleichungen} geschrieben, welche \textbf{keine Variablen} enthalten.
\begin{example}
  \begin{itemize}[noitemsep]
    \item Die Aussage $\sqrt{25} = 3$ ist falsch.
    \item Die Aussage $1+1 = 2$ ist wahr.
    \item Die Aussage $1+2 = 4$ ist falsch.
  \end{itemize}
\end{example}
In der Mathematik sind normalerweise nur \textbf{wahre} Aussagen interessant. Falsche Gleichungen können durch die Verwendung des Ungleichheitszeichens $\ne$ in eine wahren Aussage verwandelt werden.
\begin{example}
  \begin{itemize}[noitemsep]
    \item Die Aussage $\sqrt{25} \ne 3$ ist wahr.
    \item Die Aussage $1+1 = 2$ ist wahr.
    \item Die Aussage $1+2 \ne 4$ ist wahr.
  \end{itemize}
\end{example}

% ------------------------------------------------------------------------------
\subsection{Aussageformen}

Eine \textbf{Aussageform} ist eine Aussage, welche eine Lücke enthält. Ob die Aussage wahr oder falsch ist, hängt davon ab, was in die Lücke gesetzt wird.
\begin{example}
  \begin{itemize}[noitemsep]
    \item «Am \underline{\hspace{1cm}} scheint die Sonne.»
    \item «Die Wurzel von \underline{\hspace{1cm}} ist 3.»
    \item «Ich heisse \underline{\hspace{1cm}}.»
  \end{itemize}
\end{example}
In der Formelsprache der Mathematik werden Aussageformen als \textbf{Gleichungen} geschrieben, welche mindestens eine Variable enthalten. Ob diese wahr oder falsch sind, kann eigentlich erst beurteilt werden, wenn bestimmte Werte für die Variablen eingesetzt werden.
\begin{example}
  $\sqrt{x} = 3$
\end{example}
Wenn in einer Gleichung jede Variable durch eine Zahl ersetzt wird, entsteht eine Aussage, welche entweder wahr oder falsch ist:
\begin{example}
  Die Aussage $\sqrt{16} = 3$ ist falsch. Die Aussage $\sqrt{9} = 3$ ist wahr.
\end{example}
Es gibt aber auch Aussageformen, also Gleichungen mit Variablen, deren Wahrheitsgehalt für alle möglichen Werte der Variablen gleich ist.
\begin{example}
  \begin{itemize}[noitemsep]
    \item Die Gleichung $(a+b)^2= a^2+2ab+b^2$ ist für alle Werte $a,b\in\mathbb{R}$ wahr.
    \item Die Gleichung $x=x+1$ ist für alle Werte $x\in\mathbb{R}$ falsch.
  \end{itemize}
\end{example}

% ------------------------------------------------------------------------------
\subsection{Definitionsmenge}

Bei einer Aussageform müssen wir festlegen, welche Werte wir überhaupt in die Lücken einsetzen können, um eine sinnvolle Aussage zu erhalten. So macht die Aussage «Die Wurzel von Otto ist 3.» keinen Sinn.
\begin{example}
  \begin{itemize}[noitemsep]
    \item «Am \underline{\hspace{1cm}} scheint die Sonne.» $\rightarrow$ Setze einen Tag ein.
    \item «Die Wurzel von \underline{\hspace{1cm}} ist 3.» $\rightarrow$ Setze eine reelle Zahl ein.
    \item «Ich heisse \underline{\hspace{1cm}}.» $\rightarrow$ Setze einen Vornamen ein.
  \end{itemize}
\end{example}
Auch bei Gleichungen mit Variablen muss festgelegt werden, welche Zahlen überhaupt für die Variable eingesetzt werden dürfen. Wie bei den Termen heisst die Menge aller erlaubten Zahlen für eine Variable $x$ die \textbf{Definitionsmenge} von $x$ und wird mit $\mathbb{D}_{x}$ bezeichnet.
\begin{example}
  $\sqrt{x} = 3 \qquad\Rightarrow\qquad \mathbb{D}_{x} = \mathbb{R}_{0}^{+} \qquad\qquad
    \frac{1}{z+1} = \frac{1}{2} \qquad\Rightarrow\qquad \mathbb{D}_{z} = \mathbb{R}\setminus\{-1\}$
\end{example}

% ------------------------------------------------------------------------------
\subsection{Arten von Gleichungen}

\subsubsection{Identitäten}

Identitäten sind Gleichungen, welche immer wahr sind, egal welche Zahlen aus der Definitionsmengen für die Variablen eingesetzt werden. Identitäten können bewiesen werden, indem die linke Seite der Gleichung mit Hilfe von bekannten Regeln in die rechte Seite umgeformt wird.
\begin{example}
  Die folgenen Gleichungen sind Identitäten:
  \begin{itemize}
    \item Die erste binomische Formel $(a+b)^2= a^2+2ab+b^2$ für $a,b \in\mathbb{R}$.
    \item Das erste Potenzgesetz $a^k\cdot a^m = a^{k+m}$ für $a \in\mathbb{R}, k,m \in \mathbb{Z}$.
  \end{itemize}
\end{example}
Identitäten sind Hilfsmittel in der Mathematik. Sobald sie bewiesen sind, können wir sie einsetzen, um unsere eigenen Umformungen abzukürzen.

\subsubsection{Definitionen}

Mit einer Definition wird ein neuer Begriff eingeführt. Dabei bezieht sich die Definition auf schon Bekanntes. In der Mathematik und Wissenschaft können Definitionen als Gleichung geschrieben werden.
\begin{example}
  Die Kreiszahl Pi $\pi$ wird als Verhältnis des Umfang $U$ eines Kreises zu seinem Durchmesser $d$ definiert:
  \[
    \pi := \frac{U}{d}
  \]
  In der Physik wird die Arbeit $W$ als die benötigte Kraft $F$ mal der zurückgelegte Weg $s$ definiert:
  \[
    W := F\cdot s
  \]
\end{example}
Dabei wird für den neuen Begriff ein Symbol definiert. Damit klar ersichtlich ist, welches das neu definierte Symbol ist, wird manchmal das Definitions-Gleichheitszeichen $:=$ verwendet. Auf der anderen Seite des Gleichheitszeichen steht ein Term, welcher angibt, wie der neue Wert aus bekannten Werten berechnet wird.

\newpage
\subsubsection{Bestimmungsgleichungen}

Bestimmungsgleichungen sind Aussageformen, also Gleichungen mit mindestens einer Variable, welche nicht notwendigerweise für alle Werte aus der Definitionsmenge eine wahre Aussage ergeben.
\begin{example}
  Die folgende Gleichung wird zu einer wahren Aussage, wenn wir für $x$ der Wert $5$ oder $-5$ eingesetzt wird:
  \[
    x^2 = 25
  \]
\end{example}

% ------------------------------------------------------------------------------
\subsection{Gleichungen Lösen und Lösungsmenge}

Das Bestimmen der Werte aus der Definitionsmenge, für welche eine Gleichung zu einer wahren Aussage wird, wird das \textbf{Lösen der Gleichung} genannt. Die Menge aller Werte, welche die Gleichung zu einer wahren Aussage machen, heisst \textbf{Lösungsmenge} und wird mit dem Zeichen $\mathbb{L}$ abgekürzt.

\begin{example}
  Die Definitionsmenge ist $\mathbb{D} = \mathbb{R}$.
  \begin{align*}
     2x = 10 \qquad&\Rightarrow\qquad \mathbb{L} = \{5\} \\
    x^2 = 25 \qquad&\Rightarrow\qquad \mathbb{L} = \{-5; 5\} \\
    x^4 = -4 \qquad&\Rightarrow\qquad \mathbb{L} = \{\}
  \end{align*}
\end{example}

Um eine Gleichung zu lösen, wird die Gleichung so umgeformt, dass eine Variable \textbf{isoliert} auf einer Seite der Gleichung steht. Das bedeutet, dass die Gleichung in die folgende Form gebracht wird:
\[
  x = \square
\]
wobei $\square$ ein Term ist, der $x$ nicht enthält. Dann kann die Lösung abgelesen werden, indem der Wert des Terms bestimmt wird.

  \newpage
\section{Äquivalenzumformungen}

% ------------------------------------------------------------------------------
\subsection{Begriff}
Gleichungen dürfen nur nach bestimmten Regeln umgeformt werden, sodass die resultierende Gleichung \textbf{äquivalent} zur Ausgangsgleichung ist.

Eine Gleichung kann man sich als Balkenwaage vorstellen. Wenn die Gleichung eine wahre Aussage ist, sind die beiden Seiten ausbalanciert. Es sind nur solche Veränderungen erlaubt, welche die Balance der Waage erhalten.
\begin{center}
  \includegraphics[height=4cm]{Waage.pdf}
\end{center}

Anhand der wage kann man sich überlegen, welche Änderungen erlaubt sind:

\begin{itemize}
  \item eine Seite umschichten
  \item auf beiden Seiten das Gleiche hinzufügen oder entfernen
  \item beide Seiten verdoppeln oder halbieren
  \item beide Seiten vertauschen
\end{itemize}

Genau so können auch Gleichungen umgeformt werden:

\begin{itemize}
  \item Die Terme auf jeder Seite der Gleichung dürfen umgeformt werden (siehe Termumformungen).
  \item Auf beiden Seiten einer Gleichung kann der gleiche Term addiert oder subtrahiert werden.
  \item Beide Seiten einer Gleichung können mit dem gleichen Term multipliziert oder dividiert werden.
  \item Die beiden Seiten einer Gleichung dürfen vertauscht werden.
\end{itemize}

% ------------------------------------------------------------------------------
\subsection{Umformungsprotokoll}

Damit die an einer Gleichung vorgenommen Umformungen nachvollzogen werden können, wird ein \textbf{Umformungsprotokoll} geführt. Dazu wird rechts der Gleichung, durch eine vertikale Linie abgetrennt, jeweils der Umformungsschritt angegeben.
\[\def\arraystretch{2}\begin{eqt}
  2x+6 &= 0  & -6 \\
    2x &= -6 & :2 \\
     x &= -\frac{6}{2} & \text{kürzen} \\
     x &= -3
\end{eqt}\]

% ------------------------------------------------------------------------------
\subsection{Terme umformen}

Auf jeder Seite der Gleichung befindet sich ein Term. Diese dürfen mit den bekannten Regeln umgeformt werden.

\begin{example}
  \textbf{Beispiel:} Hier werden die Terme auf beiden Seiten zunächst ausmultipliziert. Danach werden auf der linken Seite die Summanden zusammengefasst.
  \[\begin{eqt}
      (2x-1)(3+2x) &= 4x(x-5)    & \text{ausmultiplizieren} \\
    6x+4x^{2}-3-2x &= 4x^{2}-20x & \text{zusammenfassen} \\
       4x^{2}+4x-3 &= 4x^{2}-20x
  \end{eqt}\]
\end{example}

% ------------------------------------------------------------------------------
\subsection{Addieren oder Subtrahieren eines Terms}

Auf beiden Seiten der Gleichung darf der gleiche Term addiert oder subtrahiert werden. Dabei wird der Term auf beiden Seiten der Gleichung an den vorhandenen Term angefügt. Anschliessend können die Summanden auf beiden Seiten zusammengefasst werden.

\begin{example}
  \textbf{Beispiel:} Hier wird auf beiden Seiten der Gleichung $4x^{2}$ und $4x$ subtrahiert. Damit wird $x^{2}$ vollständig aus der Gleichung entfernt und $x$ auf die rechte Seite der Gleichung gebracht.
  \[\begin{eqt}
    4x^{2}+4x-3 &= 4x^{2}-20x & -4x^{2}-4x \\
    4x^{2}+4x-3\mathcolor{red}{-x^{2}-4x} &= 4x^{2}-20x\mathcolor{red}{-x^{2}-4x} & \text{zusammenfassen} \\
             -3 &= -24x
  \end{eqt}\]
\end{example}

% ------------------------------------------------------------------------------
\subsection{Multiplizieren oder Dividieren mit einem Term}

Beide Seiten der Gleichung dürfen mit dem gleichen Term multipliziert oder dividiert werden. Dabei muss dieser \textbf{ungleich Null} sein. Dabei wird \textbf{jeder Summand} auf beiden Seiten der Gleichung mit dem Term multipliziert beziehungsweise durch ihn dividiert.

\begin{example}
  \textbf{Beispiel:} Hier werden beide Seiten der Gleichung mit dem kgV der Nenner multipliziert, um Brüche zu entfernen.
  \[\begin{eqt}
    \frac{x}{4}+\frac{1}{5} &= \frac{x}{2}+\frac{x}{6}                 & \cdot \kgV(4,5,2,6) = 60 \\[4mm]
    \frac{\mathcolor{red}{60\cdot} x}{4}+\frac{\mathcolor{red}{60\cdot} 1}{5} &= \frac{\mathcolor{red}{60\cdot} x}{2}+\frac{\mathcolor{red}{60\cdot} x}{6} & \text{kürzen} \\[3mm]
    15x+12 &= 30x+10x
  \end{eqt}\]
\end{example}


\begin{example}
  \textbf{Beispiel:} Hier werden beide Seiten der Gleichung mit $-1$ multipliziert, um die negativen Vorzeichen zu entfernen.
  Anschliessend werden beide Seiten der Gleichung durch $24$ dividiert, um die Variable $x$ vollständig zu isolieren.
  \[\begin{eqt}
    -3 &= -24x & \cdot(-1) \\
    \textcolor{red}{(-1)\cdot}(-3) &= \textcolor{red}{(-1)\cdot}(-24x)\ & \text{Klammern ausrechnen} \\
             3 &= 24x
  \end{eqt}\]
\end{example}

Ist auf einer Seite der Gleichung ein Mehrfaches der Variable vorhanden, so kann die Gleichung durch diesen Faktor dividiert werden, um die Variable zu isolieren. Dabei wird jeder Summand auf beiden Seiten der Gleichung durch den Faktor dividiert. Anschliessend kann gekürzt werden:

\begin{example}
  \textbf{Beispiel:} Hier werden beide Seiten der Gleichung durch $24$ dividiert, um die Variable $x$ vollständig zu isolieren.
  \[\begin{eqt}
    3 &= 24x  & :24 \\[3mm]
    \frac{3}{\textcolor{red}{24}} &= \frac{24x}{\textcolor{red}{24}} & \text{kürzen} \\[4mm]
    \frac{1}{8} &= x
  \end{eqt}\]
\end{example}

  \newpage
\section{Polynome}

\subsection{Definition}

Polynome sind eine spezielle Kategorie von Termen. Ein Polynom besteht aus einer Summe von Termen, die wiederum aus einer Zahl und einer natürlichen Potzenz einer bestimmten Variable bestehen. Die allgemeine Form eines Polynoms sieht so aus:
\[
  a_{n}x^{n} + a_{n-1}x^{n-1} + \cdots + a_{2}x^{2} + a_{1}x + a_{0}
\]
Dabei ist $a_{k}$ die Zahl, welche mit der $k$-ten Potenz der Variable multipliziert werden. Die Zahlen $a_{k}$ werden \textbf{Koeffizienten} genannt. Die Summanden werden in absteigender Reihenfolge der Potenzen angeordnet. Summanden, bei welchen der Koeffizient gleich Null ist, werden weggelassen.

\begin{example}
  Die folgenden Terme sind Polynome:
  \[
    5 \qquad\qquad k+3 \qquad\qquad 5x^{2}-3x+25 \qquad\qquad z^{4}-\frac{1}{2}
  \]
  Diese Terme sind keine Polynome:
  \[
    \frac{1}{x} \qquad\qquad \sqrt{k} \qquad\qquad x\cdot y
  \]
\end{example}

Polynome sind eine sehr wichtige Kategorie von Termen, welche später bei Gleichungen und Funktionen wieder auftreten.

\subsection{Grad und spezielle Bezeichnungen}

Der \textbf{Grad} eines Polynoms ist die höchste Potenz der Variable, welche im Polynom vorkommt.

\begin{example}
  $x^{3}+2x$ ist ein Polynom dritten Grades, $x^{5}+x^{4}$ ist ein Polynom fünften Grades.
\end{example}

Für die Grade 0 bis 3 existieren spezielle Bezeichnungen:
\begin{center}
  \def\arraystretch{1.1}
  \newcolumntype{R}{>{\raggedleft\arraybackslash}X}
  \begin{tabularx}{0.9\textwidth}{XXR}
  \toprule
    \textbf{Grad} & \textbf{Bezeichnung} & \textbf{Beispiel} \\
  \midrule
    0 & konstant & $42$ \\
  \midrule
    1 & linear & $7x-23$ \\
  \midrule
    2 & quadratisch & $-5x^{2}+45x-92$ \\
  \midrule
    3 & kubisch & $x^{3}+x^{2}-50x-34$ \\
  \bottomrule
  \end{tabularx}
\end{center}

Wenn also beispielsweise von einem «quadratischen Term» oder einer «linearen Gleichung» die Rede ist, ist immer ein Polynom des entsprechenden Grades gemeint.

  \newpage
\section{Lineare Gleichungen}

Um eine einfache Gleichung mit einer Variable zu lösen, formen wir die Gleichung mit Äquivalenzumformungen um, bis die Variable isoliert auf einer Seite der Gleichung steht:
\[
  x = \square
\]

In dieser Situation können wir auf der anderen Seite den Wert der Variable ablesen. Nun müssen wir noch überprüfen, ob der Wert auch in der Definitionsmenge $\mathbb{D}$ enthalten ist.

\begin{theorem}
  Das \textbf{Vorgehen für das Lösen einer Gleichung} ist also:
  \begin{enumerate}
    \item Grundmenge $\mathbb{G}$ festlegen.
    \item Definitionsmenge $\mathbb{D}$ ermitteln und angeben.
    \item Gleichung nach der Variable auflösen.
    \item Überprüfen, ob der ermittelte Wert in der Definitionsmenge $\mathbb{D}$ liegt.
    \item Lösungsmenge $\mathbb{L}$ angeben.
  \end{enumerate}
\end{theorem}

% ------------------------------------------------------------------------------
\subsection{Lösungsstrategie}

\subsubsection{Ausmultiplizieren}

Wenn die Terme auf der linken oder rechten Seite der Gleichung als \textbf{Faktoren} vorliegen, sollten diese zunächst \textbf{ausmultipliziert} werden..

\begin{example}
  \textbf{Beispiel:}
  \[\begin{eqt}
     (x+1)(x+6) &= (x+3)^{2} & \text{ausmultiplizieren} \\
   x^{2}+6x+x+6 &= x^{2}+6x+9
  \end{eqt}\]
\end{example}

\subsubsection{Quadrate eliminieren}

Wenn auf beiden Seiten der Gleichung die Variable im Quadrat vorliegt, kann dies auf beiden Seiten der Gleichung subtrahiert werden.

\begin{example}
  \textbf{Beispiel:}
  \[\begin{eqt}
    x^{2}+6x+x+6 &= x^{2}+6x+9 & -x^{2} \\
          6x+x+6 &= 6x+9
  \end{eqt}\]
\end{example}

Falls das Quadrat der Variable nicht wegfällt, liegt eine \textbf{quadratische Gleichung} vor. Wie solche Gleichungen gelöst werden, schauen wir später an.

\subsubsection{Terme vereinfachen}

Durch Auflösen von Klammern und zusammenfassen einzelner Summanden können die Terme auf beiden Seiten der Gleichung so vereinfacht werden, dass nur noch je ein Summand mit $x$ und ein Summand als reine Zahl vorliegt:

\begin{example}
  \textbf{Beispiel:}
  \[\begin{eqt}
    3x+(1-2x) &= 10-(3x+5) & \text{vereinfachen} \\
          x+1 &= -3x+5
  \end{eqt}\]
\end{example}

\subsubsection{Variable isolieren}

Durch das Addieren eines geschickt gewählten Terms kann die Variable auf die eine und die Zahl auf die andere Seite der Gleichung gebracht werden.

\begin{example}
  \textbf{Beispiel:}
  \[\begin{eqt}
    x+1 &= -3x+5  & +3x-1 \\
     4x &= 4
  \end{eqt}\]
\end{example}

Anschliessend kann die Gleichung durch den Faktor vor der Variable dividiert werden:

\begin{example}
  \textbf{Beispiel:}
  \[\begin{eqt}
    4x &= 4 & :4 \\
     x &= 1
  \end{eqt}\]
  Die Lösungsmenge ist $\mathbb{L} = \{1\}$.
\end{example}

Nun liegt die Gleichung in der gewünschten Form vor und die mögliche Lösung kann abgelesen werden.

\newpage
% ------------------------------------------------------------------------------
\subsection{Spezialfälle}

Wir treffen auch Gleichungen an, welche nicht in die oben angegebene Form gebracht werden können. Das ist insbesondere dann der Fall, wenn die Variable auf beiden Seiten der Gleichung wegfällt.

Dann erhalten wir eine Gleichung ohne Variable, also eine Aussage, welche wahr oder falsch ist.

Wenn die Aussage wahr ist, wissen wir, dass wir sämtliche Werte der Definitionsmenge in die Gleichung einsetzen können. Somit ist die \textbf{Lösungsmenge} gleich der \textbf{Definitionsmenge}.

\begin{example}
  \textbf{Beispiel:} Hier fällt der Term $3x$ auf beiden Seiten der Gleichung weg. Die resultierende Aussage ist richtig. Somit ist die Lösungsmenge gleich der Definitionsmenge.
  \[\begin{eqt}
    3x-2 &= 3x-2 & -3x \\
      -2 &= -2
  \end{eqt}\]
  Die Definitionsmenge ist $\mathbb{D} = \mathbb{R}$. Die Lösungsmenge ist $\mathbb{L} = \mathbb{D} = \mathbb{R}$.
\end{example}

Wenn die Aussage falsch ist, dann kann es auch kein Wert für die Variable geben, welcher die Gleichung wahr machen würde. In diesem Fall ist die \textbf{Lösungsmenge} die \textbf{leere Menge}.

\begin{example}
  \textbf{Beispiel:} Hier fällt der Term $3x$ auf beiden Seiten der Gleichung weg. Die resultierende Aussage ist falsch. Somit gibt es keine Lösungen.
  \[\begin{eqt}
    3x+4 &= 3x+1 & -3x \\
       4 &= 1
  \end{eqt}\]
  Die Lösungsmenge ist $\mathbb{L} = \{\}$.
\end{example}

  \newpage
\section{Quadratische Gleichungen}

Eine quadratische Gleichung, also eine Polynomgleichung zweiten Grades, hat die Form
\[
  ax^{2}+bx+c = 0
\]

Die Koeffizienten quadratischer Gleichungen werden üblicherweise mit $a$, $b$ und $c$ bezeichnet anstelle von $a_{2}$, $a_{1}$ und $a{0}$.

% ------------------------------------------------------------------------------
\subsection{Quadratische Gleichungen mit $c=0$}

Wenn bei einer quadratischen Gleichung der Parameter $c = 0$ ist, so hat sie die folgende Form:
\[
  ax^{2} + bx = 0
\]
Hier kann der Faktor $x$ ausgeklammert werden, damit hat die Gleichung folgende Form:
\[
  x\cdot(ax+b) = 0
\]
Um bei dieser Gleichung die Lösungen zu bestimmen, wird der Satz des Nullprodukts benötigt:

\begin{theorem}
  \textbf{Satz des Nullprodukts.} Wenn ein Produkt zweier Faktoren $a$ und $b$ gleich Null ist, dann muss einer der beiden Faktoren Null sein.
  \[
    a\cdot b = 0 \qquad\Rightarrow\qquad a = 0 \quad\text{oder}\quad b = 0
  \]
\end{theorem}

Aus diesem Satz folgt, dass die Gleichung $x\cdot(ax+b) = 0$ wahr wird, wenn entweder der Faktor $x$ oder der Faktor $xa+b$ gleich Null ist. Für diese beiden Varianten wird eine \textbf{Fallunterscheidung} gemacht.

\textbf{Fall $x = 0$}. Zunächst wird der Fall betrachtet, dass $x = 0$ ist. Dieser Fall ist trivial, da der Wert für $x$ schon feststeht.

\textbf{Fall $ax+b = 0$}. In diesem Fall liegt eine lineare Gleichung vor. Es ist bereits bekannt, wie eine solche Gleichung gelöst wird:
\[\begin{eqt}
  ax+b &= 0            & -b \\
    ax &= -b           & :a \\
     x &= -\frac{b}{a}
\end{eqt}\]
Somit ist die Lösungsmenge einer solchen quadratischen Gleichung
\[
  \mathbb{L} = \left\{0;-\frac{b}{a}\right\}
\]

\begin{example}
  \textbf{Beispiel:}
  \[\begin{eqt}
    3x^{2} + 6x &= 0 & \text{ausklammern} \\
        x(3x+6) &= 0
  \end{eqt}\]
  Fall 1: $x = 0$: trivial.

  Fall 2: $(3x+6) = 0$:
  \[\begin{eqt}
    3x+6 &= 0  & -6 \\
      3x &= -6 & :3 \\
       x &= -2
  \end{eqt}\]
  Die Lösungsmenge ist $\mathbb{L} = \{-2;0\}$.
\end{example}


% ------------------------------------------------------------------------------
\subsection{Radizieren einer Gleichung}

Um eine Gleichung zu lösen, in welcher die Variable $x$ im Quadrat auftritt, muss die Gleichung radiziert werden. Das bedeutet, dass auf beiden Seiten der Gleichung die Wurzel gezogen wird. Dazu muss sinnvollerweise auf der einen (hier linken) Seite der Gleichung ein Term als Quadrat vorliegen.

Das radizieren oder Ziehen der Wurzel wird mit einem Wurzelzeichen protokolliert. Die Wurzel wird auf beiden Seiten über dem ganzen Term gezogen.
\[\begin{eqt}
           \square^{2} &= \square           & \sqrt{\phantom{x}} \\
    \sqrt{\square^{2}} &= \pm\sqrt{\square} & \\
               \square &= \pm\sqrt{\square}
\end{eqt}\]
Dabei muss berücksichtigt werden, dass der quadrierte Term auch eine negativen Wert besitzen kann. Dies wird in der Gleichung mit einem Plusminuszeichen $\pm$ vor der Wurzel ausgedrückt.

Ein Plusminuszeichen in einem Term bedeutet, dass der Term zwei mögliche Werte hat, je nachdem ob für das Plusminuszeichen ein Plus oder ein Minus eingfügt wird. Damit kann das Radizieren einer Gleichung wie folgt ausgedrückt werden:

\begin{example}
  \textbf{Beispiel:}
  \[\begin{eqt}
           x^{2} &= 25        & \sqrt{\phantom{x}} \\
    \sqrt{x^{2}} &= \pm\sqrt{25} & \\
               x &= \pm 5
  \end{eqt}\]
  Die Gleichung hat die Lösungsmenge $\mathbb{L} = \{-5;5\}$.
\end{example}

Anschliessend können die Wurzeln mit den bekannten Wurzelgesetzen vereinfacht oder ausgerechnet werden. Im oben stehenden Beispiel ist so die Lösung $x = 5$ ermittelt worden.

\[\begin{eqt}
         x^{2} &= 25        & \sqrt{\phantom{x}} \\
  \sqrt{x^{2}} &= \pm\sqrt{25} & \\
             x &= \pm 5
\end{eqt}\]

% ------------------------------------------------------------------------------
\subsection{Anzahl Lösungen}
Das nach dem radizieren einer Gleichung eine Wurzel in der Gleichung vorkommen, muss berücksichtigt werden, dass Wurzeln von negativen Zahlen nicht definiert sind.
\[\begin{eqt}
         x^{2} &= D           & \sqrt{\phantom{x}} \\
  \sqrt{x^{2}} &= \pm\sqrt{D} &
\end{eqt}\]
Je nachdem welche Zahl unter der Wurzel steht, hat die oben stehende Gleichung eine unterschiedliche Anzahl Lösungen. Es muss daher eine \textbf{Fallunterscheidung} vorgenommen werden:
\begin{itemize}
\item $D>0$: Ist $D$ grösser als Null, so hat die Gleichung wie oben erläutert die zwei Lösungen $\pm D$, also die Lösungsmenge $\mathbb{L} = \{-D;D\}$.
\item $D=0$: Ist $D$ gleich Null so hat die Gleichung nur eine Lösung da $-D = D$, also die Lösungsmenge $\mathbb{L} = \{0\}$.
\item $D<0$: Ist $D$ kleiner als Null, so hat die Gleichung keine Lösung, da die Wurzel einer negativen Zahl nicht definiert ist, also die Lösungsmenge $\mathbb{L} = \{\}$.
\end{itemize}
Der Term $D$ wird \textbf{Diskriminante} genannt.

% ------------------------------------------------------------------------------
\subsection{Quadratische Gleichungen mit $b=0$}

Zunächst wird das Lösen von quadratischen Gleichungen mit $b = 0$ betrachtet. Diese werden auch \textbf{reinquadratische} Gleichungen, da die Variable $x$ nur als Quadrat vorkommt. Sie haben also die Form
\[
  ax^{2}+c = 0
\]
Eine reinquadratische Gleichung wird gelöst, indem zunächst $x^{2}$ isoliert wird. Dazu wird zunächst der Konstante Koeffizient $c$ auf die rechte Seite der Gleichung gebracht, anschliessend wird die Gleichung durch den Koeffizienten $a$ dividiert. Als dritter Schritt wird die Gleichung radiziert.
\[\def\arraystretch{2}\begin{eqt}
  ax^{2} + c &= 0            & -c \\
      ax^{2} &= -c           & :a \\
       x^{2} &= -\frac{c}{a} & \sqrt{\phantom{x}} \\
           x &= \pm\sqrt{-\frac{c}{a}}
\end{eqt}\]

Schliesslich wird mit $D = -\frac{a}{c}$ gemäss Fallunterscheidung die Anzahl Lösungen ermittelt und die Lösungsmenge angegeben:
\begin{itemize}
\item $D > 0$: zwei Lösungen $\pm\sqrt{-\frac{c}{a}}$ also $\quad\mathbb{L} = \left\{-\sqrt{-\frac{c}{a}}; \sqrt{-\frac{c}{a}}\right\}$
\item $D = 0:$ Null als einzige Lösung, also $\quad\mathbb{L} = \{0\}$
\item $D < 0:$ keine Lösung, also $\quad\mathbb{L} = \{\}$
\end{itemize}

\begin{example}
  \textbf{Beispiele:}
  \[\begin{eqt}
    3x^{2} - 48 &= 0     & +48 \\
         3x^{2} &= 48    & :3  \\
          x^{2} &= 16    & \sqrt{\phantom{x}} \\
              x &= \pm \sqrt{16}
  \end{eqt}\]
  $D=16 > 0$, also gibt es zwei Lösungen. Die Lösungsmenge ist $\mathbb{L} = \{-4;4\}$.

  \[\begin{eqt}
         3x^{2} &= 0    & :3  \\
          x^{2} &= 0    & \sqrt{\phantom{x}} \\
              x &= \pm\sqrt{0}
  \end{eqt}\]
  $D=0$, also gibt es eine Lösung. Die Lösungsmenge ist $\mathbb{L} = \{0\}$.

  \[\begin{eqt}
    3x^{2} + 48 &= 0     & -48 \\
         3x^{2} &= -48   & :3  \\
          x^{2} &= -16   & \sqrt{\phantom{x}} \\
              x &= \pm \sqrt{-16}
  \end{eqt}\]
  $D=-16 < 0$, also gibt es keine Lösung. Die Lösungsmenge ist $\mathbb{L} = \{\}$.
\end{example}

% ------------------------------------------------------------------------------
\subsection{Quadratische Gleichungen mit Binom}

Nun wird eine erst Form von allgmeinen quadratischen Gleichungen betrachtet. Wenn eine quadratische Gleichung in der folgenden Form Binom dargestellt werden kann, so kann sie ebenfalls durch radizieren gelöst werden:
\[\begin{eqt}
  (x+k)^{2} &= D & \sqrt{\phantom{x}} \\
        x+k &= \pm\sqrt{D} & -k \\
          x &= -k\pm\sqrt{D}
\end{eqt}\]
Damit kann abhängig von $D$ die Lösungsmenge angegeben werden:
\begin{itemize}
  \item $D>0$: zwei Lösungen $-k\pm\sqrt{D}$, also $\quad\mathbb{L} = \left\{-k-\sqrt{D};-k+\sqrt{D}\right\}$
  \item $D=0$: eine Lösung $-k$, also $\mathbb{L} = \{-k\}$
  \item $D<0$: keine Lösung, also $\mathbb{L} = \{\}$
\end{itemize}

\begin{example}
  \textbf{Beispiel:}
  \[\begin{eqt}
    (x-1)^{2} &= 25 & \sqrt{\phantom{x}} \\
          x-1 &= \pm\sqrt{25} & +1 \\
            x &= -1\pm 5
  \end{eqt}\]
  Die Lösungsmenge lautet $\mathbb{L} = \{-6;4\}$.
\end{example}

% --------------------------------------------------------------------------
\subsection{Quadratische Ergänzung}

Kann eine quadratische Gleichung nicht direkt in ein Binom umgewandelt werden, wird ein Term addiert, sodass das geht. Dieser Trick wird quadratische Ergänzung genannt.

\begin{example}
  \textbf{Beispiel:} $2x^{2}+12x+2 = 0$
  Zunächst wird die Gleichung durch den Koeffizienten von $x^{2}$ dividiert. Damit die binomische Formel angewendet werden kann, müsste der letzte Summand $9$ sein. Dies kann erreicht werden, indem $8$ zur Gleichung addiert wird:
  \[\begin{eqt}
    2x^{2}+12x+2 &= 0           & :2 \\
      x^{2}+6x+1 &= 0           & -1 \\
        x^{2}+6x &= -1          & +9 \\
    x^{2}+6x\mathcolor{red}{+9} &= -1\mathcolor{red}{+9} & \text{zusammenfassen} \\
      x^{2}+6x+9 &= 8           & \text{Binom} \\
       (x+3)^{2} &= 8           & \sqrt{\phantom{x}} \\
             x+3 &= \pm\sqrt{8} & -3 \\
               x &= -3\pm\sqrt{8}
  \end{eqt}\]
\end{example}

% ------------------------------------------------------------------------------
\subsection{Allgemeine Lösungsformel}

Mit Hilfe der quadratischen Ergänzung kann die allgemeine quadratische Gleichung gelöst werden. Diese hat folgende Form:
\[
  ax^{2}+bx+c = 0
\]
Die Gleichung wird zunächst durch $a$ dividiert und dann mit Hilfe der quadratischen Ergänzung in die Binom-Form gebracht und radiziert:
\[\begin{eqt}
                ax^{2}+bx+c &= 0 & :a \\[4mm]
  x^{2}+\frac{b}{a}x+\frac{c}{a} &= 0 & -\frac{c}{a} \\[4mm]
              x^{2}+\frac{b}{a}x &= -\frac{c}{a} & +\left(\frac{b}{a}\right)^{2} \\[4mm]
  x^{2}+\frac{b}{a}x\mathcolor{red}{+\left(\frac{b}{2a}\right)^{2}} &= -\frac{c}{a}\mathcolor{red}{+\left(\frac{b}{2a}\right)^{2}} & \text{links mit Binom faktorisieren} \\[4mm]
  \left(x+\frac{b}{2a}\right)^{2} &= -\frac{c}{a}+\left(\frac{b}{2a}\right)^{2} & \text{rechts vereinfachen} \\[4mm]
  \left(x+\frac{b}{2a}\right)^{2} &= -\frac{c\cdot 4a}{a\cdot 4a}+\frac{b^{2}}{4a^{2}} & \text{rechts Brüche addieren} \\[4mm]
  \left(x+\frac{b}{2a}\right)^{2} &= \frac{b^{2}-4ac}{4a^{2}} & \sqrt{\phantom{x}} \\[4mm]
    x+\frac{b}{2a} &= \pm\sqrt{\frac{b^{2}-4ac}{4a^{2}}} & \text{Wurzel vereinfachen} \\[4mm]
    x+\frac{b}{2a} &= \pm\frac{\sqrt{b^{2}-4ac}}{2a}
\end{eqt}\]
Der letzte Schritt ist erlaubt, da im Nenner ein Quadrat vorhanden ist. Der Term unter der Wurzel gibt an, wie viele Lösungen die Gleichung hat. Damit verdient dieser Term ein eigenes Formelzeichen $D$ und eine eigene Bezeichnung: Er wird \textbf{Diskriminante} genannt.
\[
  D = b^{2}-4ac
\]
Um die Lösungen zu erhalten, muss aber $x$ noch vollständig isoliert werden:
\[\begin{eqt}
  x+\frac{b}{2a} &= \pm\sqrt{\frac{b^{2}-4ac}{4a^{2}}} & -\frac{b}{2a} \\[4mm]
  x &=  -\frac{b}{2a}\pm\frac{\sqrt{b^{2}-4ac}}{2a} & \text{Brüche addieren} \\[4mm]
  x &= \frac{-b\pm\sqrt{b^{2}-4ac}}{2a}
\end{eqt}\]

Damit ist die quadratische Gleichung allgemein gelöst:

\begin{theorem}
  \textbf{Lösungen der quadratischen Gleichung}. Für die allgemeine quadratische Gleichung
  \[
    ax^{2}+bx+c = 0
  \]
  gibt die Diskriminante $D = b^{2}-4ac$ an, wie viele Lösungen vorhanden sind:
  \begin{itemize}
  \item $D>0$ Die Gleichung hat zwei Lösungen:
  \[
    \frac{-b\pm\sqrt{b^{2}-4ac}}{2a}
  \]
  \item $D=0$ Die Gleichung hat eine Lösung:
  \[
    -\frac{b}{2a}
  \]
  \item $D<0$ Die Gleichung hat keine Lösung.
  \end{itemize}
\end{theorem}

\end{document}
