\newpage
\section{Quadratische Gleichungen}

Eine quadratische Gleichung, also eine Polynomgleichung zweiten Grades, hat die Form
\[
  ax^{2}+bx+c = 0
\]

Die Koeffizienten quadratischer Gleichungen werden üblicherweise mit $a$, $b$ und $c$ bezeichnet anstelle von $a_{2}$, $a_{1}$ und $a_{0}$.

% ------------------------------------------------------------------------------
\subsection{Quadratische Gleichungen mit $c=0$}

Wenn bei einer quadratischen Gleichung der Parameter $c = 0$ ist, so hat sie die folgende Form:
\[
  ax^{2} + bx = 0
\]
Hier kann der Faktor $x$ ausgeklammert werden, damit hat die Gleichung folgende Form:
\[
  x\cdot(ax+b) = 0
\]
Um bei dieser Gleichung die Lösungen zu bestimmen, wird der Satz des Nullprodukts benötigt:

\begin{theorem}
  \textbf{Satz des Nullprodukts.} Wenn ein Produkt zweier Faktoren $a$ und $b$ gleich Null ist, dann muss einer der beiden Faktoren Null sein.
  \[
    a\cdot b = 0 \qquad\Rightarrow\qquad a = 0 \quad\text{oder}\quad b = 0
  \]
\end{theorem}

Aus diesem Satz folgt, dass die Gleichung $x\cdot(ax+b) = 0$ wahr wird, wenn entweder der Faktor $x$ oder der Faktor $xa+b$ gleich Null ist. Für diese beiden Varianten wird eine \textbf{Fallunterscheidung} gemacht.

\textbf{Fall $x = 0$}. Zunächst wird der Fall betrachtet, dass $x = 0$ ist. Dieser Fall ist trivial, da der Wert für $x$ schon feststeht.

\textbf{Fall $ax+b = 0$}. In diesem Fall liegt eine lineare Gleichung vor. Es ist bereits bekannt, wie eine solche Gleichung gelöst wird:
\begin{eqt}
  ax+b &= 0            & -b \\
    ax &= -b           & :a \\
     x &= -\frac{b}{a}
\end{eqt}
Somit ist die Lösungsmenge einer solchen quadratischen Gleichung
\[
  \mathbb{L} = \left\{0;-\frac{b}{a}\right\}
\]

\begin{example}
  \begin{eqt}
    3x^{2} + 6x &= 0 & \text{ausklammern} \\
        x(3x+6) &= 0
  \end{eqt}
  Fall 1: $x = 0$: trivial.

  Fall 2: $(3x+6) = 0$:
  \begin{eqt}
    3x+6 &= 0  & -6 \\
      3x &= -6 & :3 \\
       x &= -2
  \end{eqt}
  Die Lösungsmenge ist $\mathbb{L} = \{-2;0\}$.
\end{example}


% ------------------------------------------------------------------------------
\subsection{Quadratische Gleichungen mit $b=0$}
Wenn bei einer quadratischen Gleichung der Parameter $c = 0$ ist, so hat sie die folgende Form:
\[
ax^{2} + c = 0
\]
Hier kann der Faktor $c$ subtrahiert werden, damit hat die Gleichung folgende Form:
\[
ax^2 = -c
\]
Teilen wir jetzt auf beiden Seiten durch $a$, erhalten wir:
\[
x^2 = \frac{-c}{a}
\]
Um bei dieser Gleichung die Lösungen zu bestimmen, müssen wir die Gleichung \textbf{radizieren} (die Wurzel ziehen).
\begin{eqt}
       x^{2} &= -\frac{c}{a} & \sqrt{\phantom{x}} \\
x &= \pm\sqrt{-\frac{c}{a}}
\end{eqt}

\subsubsection{Radizieren einer Gleichung}
Das Radizieren (Wurzelziehen) ist im allgemeinen \textbf{keine} Äquivalenzumformung. Wir benötigen deshalb eine \textbf{Fallunterscheidung}.

\begin{example}
	Die Gleichungen
	\[ x^2 =-4 \\ x^2 =-25 \\ x^2= -100 \]
	haben \textbf{keine} Lösung, da es keine Zahl $x$ gibt, die mit sich selbst multipliziert eine negative Zahl ergibt. Ziehen wir auf beiden Seiten die Wurzel, erhalten wir:
	\[ x =\sqrt{-4} \\ x =\sqrt{-25} \\ x= \sqrt{-100} \]
	und die Wurzel einer negativen Zahl ist \textbf{nicht definiert}. Entsprechend gib es keine Lösung.
\end{example}
Haben wir eine Gleichung der allgemeinen Form
\[
x^2 = D,
\]
so hat die Gleichung \textbf{keine Lösung}, wenn $D<0$ ist. Die Lösungsmenge ist dann die leere Menge $\mathbb{L}= \{\}$.
\begin{example}
	Die Gleichung
	\[ x^2 = 0 \]
	hat \textbf{eine} Lösung, nämlich die Zahl 0. Ziehen wir auf beiden Seiten die Wurzel, erhalten wir:
	\[ x =\sqrt{0} = 0 \]
\end{example}
Haben wir eine Gleichung der allgemeinen Form
\[
x^2 = D,
\]
so hat die Gleichung \textbf{eine Lösung}, wenn $D=0$ ist. Die Lösungsmenge ist dann die leere Menge $\mathbb{L}= \{0\}$.

Der \textit{Normalfall} ist der Folgende:
\begin{example}
	Die Gleichungen
	\[ x^2 =4 \\ x^2 =25 \\ x^2= 100 \]
	haben \textbf{zwei} Lösungen, da es zwei Zahlen $\pm x$ gibt, welche die Gleichung erfüllen. Ziehen wir auf beiden Seiten die Wurzel, erhalten wir:
	\[ x =\pm \sqrt{4} = \pm 2 \\ x = \pm \sqrt{25} = \pm 5\\ x= \pm  \sqrt{100} = \pm 10 \]
	Das $x= \pm 2$ ist eine Kurzschreibweise für $x=2$ \textbf{oder} $x=-2$, denn beide Zahlen lösen die Gleichung $x^2=4$. Das sind die beiden Fälle, die wir unterscheiden müssen:
	  \begin{eqt}
		x^2 &= 4  & \sqrt{\square} \\
		\Leftrightarrow x &=2 \text{ oder }  x= -2
	\end{eqt}

\end{example}
Haben wir eine Gleichung der allgemeinen Form
\[
  x^2 = D,
\]
so hat die Gleichung \textbf{zwei Lösungen}, wenn $D>0$ ist. Die Lösungsmenge ist dann die leere Menge $\mathbb{L}= \{-\sqrt{D},\sqrt{D}\}$.

\begin{note}
	\textbf{Bemerkung:} Das Radizieren einer Gleichung ist genau dann eine Äquivalenzumformung, wenn eine Fallunterscheidung für $\pm \sqrt{\square}$ gemacht wird \textbf{und} jeweils der \textbf{gesamte Term} einer Seite radiziert wird.
	\begin{eqt}
		\text{Term 1} &= \text{Term 2}           & \sqrt{\phantom{x}} \\
	\Leftrightarrow	\sqrt{\text{Term 1}} &= \pm\sqrt{\text{Term 2}}
	\end{eqt}
\end{note}

Die obige Bemerkung ist auch der Grund, warum Sie Gleichungen der Form $ax^2+bx+c=0$ für $b\neq0$ durch einfaches Wurzelziehen \textbf{nicht} lösen können.
Betrachten wir das folgende Beispiel:
\begin{example}
	\[ x^2 + 2x +1 = 0 \Leftrightarrow \sqrt{ x^2 + 2x +1} = \pm \sqrt{0} \]
	\[
	\Leftrightarrow x^2 + 2x +1 = 0 \Leftrightarrow x^2 = - 2x -1 \Leftrightarrow x=  \pm \sqrt{-2x-1} \]
	\[
	\Leftrightarrow x^2 + 2x +1 = 0 \Leftrightarrow x^2+2x = -1 \Leftrightarrow  \sqrt{x^2+2x}=  \pm \sqrt{-1}   \]
	Egal, was wir versuchen: Wir können die Wurzel nicht aus einer Summer ziehen!
\end{example}

\begin{note}
	\textbf{Zusammenfassung:}
	Für Gleichungen der Form
	\begin{eqt}
		x^{2} &= D           & \sqrt{\phantom{x}} \\
		\sqrt{x^{2}} &= \pm\sqrt{D} &
	\end{eqt}
	gibt es unterschiedlich viele Lösungen, je nachdem welche Zahl unter der Wurzel steht:
	\begin{itemize}
		\item $D>0$: Ist $D$ grösser als Null, so hat die Gleichung wie oben erläutert die zwei Lösungen $\pm \sqrt{D}$, also die Lösungsmenge $\mathbb{L} = \{-\sqrt{D};\sqrt{D}\}$.
		\item $D=0$: Ist $D$ gleich Null so hat die Gleichung nur eine Lösung da $-0 = +0$, also die Lösungsmenge $\mathbb{L} = \{0\}$.
		\item $D<0$: Ist $D$ kleiner als Null, so hat die Gleichung keine Lösung, da die Wurzel einer negativen Zahl nicht definiert ist, also die Lösungsmenge $\mathbb{L} = \{\}$.
	\end{itemize}
	Der Term $D$ wird \textbf{Diskriminante} genannt (diskriminieren bedeutet unterscheiden).
\end{note}


\begin{example}
  \begin{eqt}
    3x^{2} - 48 &= 0     & +48 \\
         3x^{2} &= 48    & :3  \\
          x^{2} &= 16    & \sqrt{\phantom{x}} \\
              x &= \pm \sqrt{16}
  \end{eqt}
  $D=16 > 0$, also gibt es zwei Lösungen. Die Lösungsmenge ist $\mathbb{L} = \{-4;4\}$.

  \begin{eqt}
         3x^{2} &= 0    & :3  \\
          x^{2} &= 0    & \sqrt{\phantom{x}} \\
              x &= \pm\sqrt{0}
  \end{eqt}
  $D=0$, also gibt es eine Lösung. Die Lösungsmenge ist $\mathbb{L} = \{0\}$.

  \begin{eqt}
    3x^{2} + 48 &= 0     & -48 \\
         3x^{2} &= -48   & :3  \\
          x^{2} &= -16   & \sqrt{\phantom{x}} \\
              x &= \pm \sqrt{-16}
  \end{eqt}
  $D=-16 < 0$, also gibt es keine Lösung. Die Lösungsmenge ist $\mathbb{L} = \{\}$.
\end{example}

% ------------------------------------------------------------------------------
\subsection{Lösen mit binomischen Formeln}
\begin{example}
		\begin{eqt}
		x^2 +4x +4 &= 16 & \text{Faktorisieren 1. binomische Formel} \\
		(x+2)^{2} &= 16 & \sqrt{\phantom{x}} \\
		x+2 &= \pm\sqrt{16} & -2 \\
		x &= -2 \pm 4
	\end{eqt}
	Die Lösungsmenge lautet $\mathbb{L} = \{-6;4\}$.
		\begin{eqt}
		x^2 -2x +1 &= 25 & \text{Faktorisieren 2. binomische Formel} \\
		(x-1)^{2} &= 25 & \sqrt{\phantom{x}} \\
		x-1 &= \pm\sqrt{25} & +1 \\
		x &= 1\pm 5
	\end{eqt}
	Die Lösungsmenge lautet $\mathbb{L} = \{-4;6\}$.
	\begin{eqt}
		x^2  - 81 &= 0 & \text{Faktorisieren 3. binomische Formel} \\
		(x+9)(x-9) &= 0 & \text{Nullprodukt} \\
		x &= \pm 9
	\end{eqt}
	Nach dem Satz über das Nullprodukt ist entweder $x+9=0$ oder $x-9=0$. Die Lösungsmenge lautet $\mathbb{L} = \{-9;9\}$.
\end{example}
Die binomischen Formeln können dabei helfen, quadratische Gleichungen zu  lösen. Wenn eine quadratische Gleichung durch ein Binom dargestellt werden kann, so kann sie ebenfalls durch radizieren gelöst werden:
\begin{eqt}
  (x+k)^{2} &= D & \sqrt{\phantom{x}} \\
        x+k &= \pm\sqrt{D} & -k \\
          x &= -k\pm\sqrt{D}
\end{eqt}
Damit kann abhängig von $D$ die Lösungsmenge angegeben werden:
\begin{itemize}
  \item $D>0$: zwei Lösungen $-k\pm\sqrt{D}$, also $\quad\mathbb{L} = \left\{-k-\sqrt{D};-k+\sqrt{D}\right\}$
  \item $D=0$: eine Lösung $-k$, also $\mathbb{L} = \{-k\}$
  \item $D<0$: keine Lösung, also $\mathbb{L} = \{\}$
\end{itemize}



% --------------------------------------------------------------------------
\subsection{Quadratische Ergänzung}
Mit Hilfe eines Tricks kann jede quadratische Gleichung in ein Binom umgewandelt werden. Dieser Trick wird quadratische Ergänzung genannt.

\begin{example}
  $2x^{2}+12x+2 = 0$
  Zunächst liegt keine binomische Formel vor. Betrachten Sie das Beispiel und beschreiben Sie in eigenen Worten, wie der \textit{Trick} funktioniert.
  \begin{eqt}
    2x^{2}+12x+2 &= 0           & :2 \\
      x^{2}+6x+1 &= 0           & -1 \\
        x^{2}+6x &= -1          & +9 \\
    x^{2}+6x\mathcolor{red}{+9} &= -1\mathcolor{red}{+9} & \text{zusammenfassen} \\
      x^{2}+6x+9 &= 8           & \text{Binom} \\
       (x+3)^{2} &= 8           & \sqrt{\phantom{x}} \\
             x+3 &= \pm\sqrt{8} & -3 \\
               x &= -3\pm\sqrt{8}
  \end{eqt}
\end{example}

% ------------------------------------------------------------------------------
\subsection{Allgemeine Lösungsformel}

Mit Hilfe der quadratischen Ergänzung kann die allgemeine quadratische Gleichung gelöst werden. Diese hat folgende Form:
\[
  ax^{2}+bx+c = 0
\]
Die Gleichung wird zunächst durch $a$ dividiert und dann mit Hilfe der quadratischen Ergänzung in die Binom-Form gebracht und radiziert:
\begin{eqt}
                ax^{2}+bx+c &= 0 & :a \\[4mm]
  x^{2}+\frac{b}{a}x+\frac{c}{a} &= 0 & -\frac{c}{a} \\[4mm]
              x^{2}+\frac{b}{a}x &= -\frac{c}{a} & +\left(\frac{b}{a}\right)^{2} \\[4mm]
  x^{2}+\frac{b}{a}x\mathcolor{red}{+\left(\frac{b}{2a}\right)^{2}} &= -\frac{c}{a}\mathcolor{red}{+\left(\frac{b}{2a}\right)^{2}} & \text{links mit Binom faktorisieren} \\[4mm]
  \left(x+\frac{b}{2a}\right)^{2} &= -\frac{c}{a}+\left(\frac{b}{2a}\right)^{2} & \text{rechts vereinfachen} \\[4mm]
  \left(x+\frac{b}{2a}\right)^{2} &= -\frac{c\cdot 4a}{a\cdot 4a}+\frac{b^{2}}{4a^{2}} & \text{rechts Brüche addieren} \\[4mm]
  \left(x+\frac{b}{2a}\right)^{2} &= \frac{b^{2}-4ac}{4a^{2}} & \sqrt{\phantom{x}} \\[4mm]
    x+\frac{b}{2a} &= \pm\sqrt{\frac{b^{2}-4ac}{4a^{2}}} & \text{Wurzel vereinfachen} \\[4mm]
     x+\frac{b}{2a} &= \pm\sqrt{\frac{b^{2}-4ac}{4a^{2}}} & -\frac{b}{2a} \\[4mm]
    x &=  -\frac{b}{2a}\pm\frac{\sqrt{b^{2}-4ac}}{2a} & \text{Brüche addieren} \\[4mm]
    x &= \frac{-b\pm\sqrt{b^{2}-4ac}}{2a}
\end{eqt}
Der Term unter der Wurzel gibt an, wie viele Lösungen die Gleichung hat und wird \textbf{Diskriminante} $D$ genannt.
\[
  D = b^{2}-4ac
\]
Damit ist die quadratische Gleichung allgemein gelöst!

\begin{theorem}
  \textbf{Lösungen der quadratischen Gleichung}. Für die allgemeine quadratische Gleichung
  \[
     ax^{2}+bx+c = 0
  \]
  gibt die Diskriminante $D = b^{2}-4ac$ an, wie viele Lösungen vorhanden sind:
  \begin{itemize}
  \item $D>0$ Die Gleichung hat zwei Lösungen:
  \[
    x_{1/2} =\frac{-b\pm\sqrt{b^{2}-4ac}}{2a} \\ \mathbb{L}= \{x_1,x_2\}
  \]
  \item $D=0$ Die Gleichung hat eine Lösung:
  \[
   x= -\frac{b}{2a}  \\ \mathbb{L}= \{x\}
  \]
  \item $D<0$ Die Gleichung hat keine Lösung. $\mathbb{L}= \{\}$
  \end{itemize}
\end{theorem}

\subsection{Lösen durch Faktorisieren}
Gelingt es uns, den Term $ax^2+bx+c$ zu faktorisieren, können wir dank dem Satz über das Nullprodukt die Lösungen ziemlich schnell ablesen. Dazu müssen wir lediglich schauen, wann die Faktoren Null sind. Es lohnt sich oft, kurz nach einer Faktorisierung zu suchen, bevor man die allgemeine Lösungsformel anwendet.

\begin{example}
  \begin{enumerate}[label=(\alph*)]
  \item
    \begin{eqt}
      x^2-x-12 &= 0 \\
      (x-4)(x+3) &= 0 \\
      x-4 &= 0 \text{ oder } x+3=0 \\
      \mathbb{L}=\{-3;4\}
    \end{eqt}
  \item
  \begin{eqt}
    2^2+7x+6 &= 0 \\
    (2x+3)(x+2) &= 0 \\
    2x+3 &= 0 \text{ oder } x+2=0 \\
    \mathbb{L}=\left\{-2;\frac{-3}{2}\right\}
  \end{eqt}
  \item
    \begin{eqt}
      x^2-9x-20 &= 0 \\
      \Leftrightarrow (x-4)(x-5) &= 0 \\
      \Leftrightarrow x-4 &= 0 \text{ oder } x-5=0 \\
      \mathbb{L}=\{4;5\}
    \end{eqt}
  \end{enumerate}
\end{example}
